So far, we have only considered methods to sample static averages, which concern quantities at thermodynamic equilibrium.
Such techniques yield information about the bulk macroscopic properties of the system, for which there is no discernable macroscopic evolution.
We now turn to the next natural question, which is to consider systems in which there is such an evolution,
 which typically arises from a perturbation of the equilibrium dynamics, either by the introduction of a non-gradient forcing term, 
 or a modification of the fluctuation-dissipation part for which the fluctuation-dissipation relation \eqref{eq:general_fd_relation} is not verified.
 We will not consider the latter case, which is relevant for instance to the modelling of heat transport within an atomic system, to concentrate on the first case.
 In general, systems undergoing such a perturbation of the dynamics will reach a new steady state in which there is a net flux in some observable.
 Mathematically, this translate into the existence of a response observable $R$ which has zero average with respect to the canonical measure $\mu$, but which has a positive average with respect to the perturbation steady-state.
 A natural question is that of the sensitivity of the system to the perturbation: one way to quantify this is to modulate the strength of the perturbation by a positive real parameter $\eta>0$, and,
  assuming that the average response is asymptotically linear as $\eta\to 0$, to compute the linear coefficient linking $\eta$ and the average response. 
  This quantity is called a \textit{transport coefficient}, and we will dedicate the next two chapters to different methods of computing them using molecular simulation.

\section{Non-equilibrium molecular dynamics}
We first consider the most natural method, which is to directly apply a forcing term which is not the gradient of a periodic function. 
The general framework is that of a classical Langevin dynamics

\begin{equation}
    \label{eq:general_nemd_dynamics}
    \left\{\begin{aligned}
        \text dq_t&=M^{-1}p_t\dt,\\
        \text dp_t&= -\nabla V(q_t)\dt -\gamma M^{-1}p_t\dt+\sqrt{\frac{2\gamma}\beta}\text dW_t +\eta F(q_t)\dif t,
    \end{aligned}\right.
\end{equation}
perturbed by the configuration-dependent forcing term $F$, and where the strength of the perturbation is modulated by the parameter $\eta>0$.
Its generator is given by the operator 
\begin{equation}
    \label{eq:nemd_generator}
    \cL_{\gamma,F}=\cL_\gamma +\eta F\cdot \nabla_p.
\end{equation}
Given a response of interest $R$, the transport coefficient is given by the following definition:

\begin{equation}
    \label{eq:transport_coefficient}
    \rho_{R,F}=\underset{\eta \to 0}{\lim}\, \frac{\mathbb{E}[R]}{\eta},
\end{equation}
where $\mathbb{E}_\eta$ denotes the expectation with respect to the steady-state probability distribution. 
If the context makes $R$ clear from the knowledge of $F$, we will drop it from the notation and simply write $\rho_F$ for the transport coefficient.
For this definition to make sense, one has to show that the steady-state with respect to which we take the expectation is well-defined.
Since the steady-state is defined by the fact that it is invariant with respect to the dynamics \eqref{eq:general_nemd_dynamics}, 
this translate into the fact that it is the solution to a stationary Fokker-Planck equation
\begin{equation}
    \label{eq:nemd_fp_equation}
    \cL_{\gamma,F}^\dagger \rho = 0.
\end{equation}
Using analytic properties of $\cL_{\gamma,F}$ such as hypoellipticity, one can hopefully infer existence, uniqueness and regularity of the solution distribution.
The steady-state measure is a high-dimensional measure on phase space for which a closed form is generally unavailable.
 Thus one has to resort to ergodic averages under the dynamics \eqref{eq:general_nemd_dynamics} to compute averages with respect to $\E_\eta$.
 This poses another theoretical difficulty, that of showing that the steady-state measure is ergodic for the dynamics.

 \begin{remark}[Size of the linear regime]
 \end{remark}

\subsection{Mobility}
\subsection{Shear viscosity}
In this section, we describe a NEMD method to compute the shear viscosity in a fluid, from \cite{JS12}
We index the state as \[\left(q_{i\alpha}\right)_{\substack{\leq i\leq N\\1\leq \alpha\leq d}},\qquad \left(p_{i\alpha}\right)_{\substack{\leq i\leq N\\1\leq \alpha\leq d}}.\]
The Thevenin dynamics we considered writes, for $i=1\dots N$,
\begin{equation}
    \label{eq:thevenin_sv_dynamics}
    \left\{\begin{aligned}
        \dif q_{ij,t}&=\frac{p_{ij,t}}m\dif t,&1\leq j\leq d,\\
        \dif p_{i1,t}&=-\nabla_{q_{i1}}V(q_t)\dif t-\gamma_1\frac{p_{i1,t}}m+\sqrt{\frac{2\gamma_1}\beta}\dif W_t^{i1}+\xi F(q_{i2,t}),&\\
        \dif p_{ij,t}&=-\nabla_{q_{ij}}V(q_t)\dif t-\gamma \frac{p_{ij,t}}m+\sqrt{\frac{2\gamma}\beta}\dif W_t^{ij},&2\leq j\leq d.
    \end{aligned}\right.
\end{equation}
This equation describes a system undergoing standard Langevin evolution, with the longitudinal momentum coordinates 
\[\left(p_{i1}\right)_{1\leq i\leq N}\]
subject to a forcing $F$ which is constant in time, but depends on the corresponding transverse position coordinates
\[\left(q_{i2}\right)_{1\leq i\leq N}.\]
Additionally we allow the longitudinal fluctuation dissipation term to have a distinct friction coefficient $\gamma_1$, which is a free parameter of the model.

\begin{equation}
    \label{eq:velocity_profile}
    u_x(Y)\defeq \underset{\epsilon \to 0}{\lim} \, \underset{\xi \to 0}{\lim} \frac{L_y\E_\xi \left[\sum_{i=1}^N p_{i1}\chi_\epsilon(q_{i2}-Y)\right]}{\xi mN}
\end{equation}

\begin{equation}
    \label{eq:shear_viscosity_relation_diffeq}
    \eta u_x''(Y)+\gamma_1 \rho u_x(Y)=\rho F(Y)
\end{equation}

\section{The Green-Kubo method}
\subsection{Mobility}
\subsection{Shear viscosity}