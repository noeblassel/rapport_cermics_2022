\section{Non-equilibrium dynamics}
In the previous chapter, we investigated various numerical strategies to sample states from thermodynamic ensembles.
The object now is to go beyond the computation of average observables, to consider dynamical behavior of molecular systems.
One way to define question we may ask is how the system responds to small perturbations of the equilibrium.
For instance we may think of applying a small non-gradient force $\eta F$ to the equilibrium force $-\nabla V(q)$, which amounts to considering the following equation:
\begin{equation}
    \label{eq:nemd_langevin}
    \left\{\begin{aligned}
        \dif q_t &= M^{-1}p_t\dif t\\
        \dif p_t &= -\nabla V(q_t)\dif t +\eta F\dif t -\gamma M^{-1}p_t\dif t +\sqrt{\frac{2\gamma}{\beta}}\dif W_t
    \end{aligned}\right.
\end{equation}
We can think of this as effectively tilting the potential landscape so that we expect the steady-state of this perturbed dynamics to feature a measurable flux of particles in the direction $F$,
which we measure by looking at the average velocity in the direction $F$,
\begin{equation}
    \label{eq:mobility}
    \E_\eta\left[F\cdot \left(M^{-1}p\right)\right],
\end{equation}
where $\E_\eta$ denotes the average with respect to the steady-state for the dynamics \eqref{eq:nemd_langevin}, which we take to be a probability measure on $\mathcal E$,
 for which we have no closed form, in contrast to the equilibrium setting.
\section{Linear response theory}
\section{Numerical schemes}