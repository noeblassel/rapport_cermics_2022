\section{Introduction}
So far, the non-equilibrium setting we considered was one in which we perturbed the dynamics by a constant forcing,
 and measured the linear relationship between the magnitude of the forcing and the induced response.
An alternative idea would be to consider fixing the response, and measuring the magnitude of the forcing needed to induce it. This is the core of the Norton method.


\section{Norton dynamics for mobility}
Let us fix a direction defined by a vector $F\in \R^{dN}$. The response we consider is the velocity in the direction $F$, which is the observable \[F\cdot \left(M^{-1} p\right).\]
Again, we consider perturbed Langevin dynamics, but this time the magnitude of the non-gradient force $\eta$ is replaced by a fluctuating magnitude $\Lambda_t$, which is determined in order to ensure that the response
\[v=F\cdot\left(M^{-1}p\right)=p\cdot\left(M^{-1}F\right)\]
is a constant.

\subsection{The dynamics}
Thus, we consider the dynamics
\begin{equation}
    \label{eq:norton_method}
    \left\{\begin{aligned}
        \dif q_t&=\left(M^{-1}p_t\right)\dif t\\
        \dif p_t&=-\nabla V(q_t)\dif t -\gamma\left(M^{-1}p_t\right)\dif t+\sqrt{\frac{2\gamma}\beta}\dif W_t+F\dif \Lambda_t,
    \end{aligned}\right.
\end{equation}
where $\Lambda_t$ is a real-valued stochastic process. We make the tacit assumption that for fixed $v$, the process $\Lambda_t$ is well-defined and that ergodic averages
\[\frac1 T\int_0^T \dif\Lambda_t=\frac{\Lambda_T-\Lambda_0}T\]
converge almost surely to some average value $R_v$ under an invariant measure for the dynamics. We can easily write an equation for $\Lambda_t$, as follows.
Our aim is to determine
\[\underset{v\to 0}\lim \frac{v}{R_v}\]
Using Itô's formula on the response, we get
\begin{equation}
    \label{eq:formula_response}
\dif \left(\left(M^{-1}F\right)\cdot p\right)_t=\left(M^{-1}F\right)\cdot\left(-\nabla V(q_t)\dif t -\gamma M^{-1}p_t\dif t+\sqrt{\frac{2\gamma}\beta}\dif W_t\right)+\left( \left(M^{-1}F\right)\cdot F\right)\dif\Lambda_t=0,
\end{equation}
which gives $\Lambda_t$ as an Itô process,
\begin{equation}
    \label{eq:norton_multiplier_ito_decomposition}
    \dif \Lambda_t=\left( \left(M^{-1}F\right)\cdot F\right)^{-1}\left(M^{-1}F\right)\cdot\left(\nabla V(q_t)\dif t +\gamma M^{-1}p_t\dif t-\sqrt{\frac{2\gamma}\beta}\dif W_t\right).
\end{equation}
At this point, let us introduce the linear maps
\begin{equation}
    \label{eq:norton_projector}
    \begin{aligned}
        &P_{M,F}\,u=\frac{\left(M^{-1}F\right)\cdot u}{\left(M^{-1}F\right)\cdot F}F\qquad\forall\,u\in\R^{dN}\\
        &P_{M,F}^\perp=\Id-P_{M,F}
    \end{aligned}
\end{equation}
which are simply the orthogonal projectors onto the subspace spanned by $F$ and its orthogonal complement, with respect to the weighted scalar product
\begin{equation}
    \label{eq:norton_weighted_scalar_product}
    \left\langle x,y\right\rangle_{M}=\left\langle M^{-1}x,y\right\rangle.
\end{equation}
Using this projector and substituting $\dif \Lambda_t$ for its value, we get the equation for the augmented dynamics $(q_t,p_t,\Lambda_t)\in \mathcal E\times \R$.
\begin{equation}
    \label{eq:norton_final_sde}
    \left\{\begin{aligned}
        \dif q_t &= M^{-1}p_t\dif t\\
        \dif p_t &= P_{M,F}^\perp\left(-\nabla V(q_t)\dif t -\gamma M^{-1}p_t\dif t +\sqrt{\frac{2\gamma}\beta}\dif W_t\right)\\
        \dif \Lambda_t &=\frac{M^{-1}F}{\left(M^{-1}F\right)\cdot F}\cdot\left(\nabla V(q_t)\dif t +\gamma M^{-1}p_t\dif t-\sqrt{\frac{2\gamma}\beta}\dif W_t\right)
    \end{aligned}\right.
\end{equation}
Notice this equation is almost the same as for the Langevin dynamics.
The kinetic part is projected onto the $F$-subspace with respect to $\langle \cdot,\cdot\rangle_M$, and $\dif\Lambda_t$ is minus the coordinate in the $F$ direction of the standard Langevin equation for $\dif p_t$, again with respect to $\langle \cdot ,\cdot \rangle_M$.
The Brownian motions $W_t$ are the same in the last two equations, hence $p_t$ and $\Lambda_t$ are coupled processes.[GENERATOR FOR THE FULL DYNAMICS?]

\begin{equation}
    \label{eq:norton_generator}
---
\end{equation}

\subsection{Analytic calculations for the kinetic dynamics}

The kinetic part of the dynamics writes as the combination of a ballistic Hamiltonian evolution
\begin{equation}
    \label{eq:elem_hamiltonian_equation}
    \left\{\begin{aligned}
        \dif q_t&=0\\
        \dif p_t&=-P_{M,F}^\perp\nabla V(q_t)
    \end{aligned}\right.\implies\left\{\begin{aligned}
        q_t&=q_0\\
        p_t&=p_0-tP_{M,F}^\perp\nabla V(q_0)
    \end{aligned}\right.
\end{equation}
and a fluctuation-dissipation part,
\begin{equation}
    \label{eq:projected_OU}
    \dif p_t=-\gamma P_{M,F}^\perp M^{-1}p_t\dif t-\sqrt{\frac{2\gamma}{\beta}}P_{M,F}^\perp\dif W_t,
\end{equation}
which can also be analytically solved.
Indeed, applying Itô's formula to the rescaled process $\e^{\gamma P_{M,F}^\perp M^{-1}t}p_t$ yields
\[\mathrm{e}^{\gamma P_{M,F}^\perp M^{-1}t}\left(\gamma P_{M,F}^\perp M^{-1}p_t\dif t - \gamma P_{M,F}^\perp M^{-1}p_t\dif t-\sqrt{\frac{2\gamma}{\beta}} P_{M,F}^\perp\dif W_t\right),\]
hence we get 
\begin{equation}
    \label{eq:projected_ou_analytical_form}
    \begin{aligned}
    p_t&=\mathrm{e}^{-\gamma P_{M,F}^\perp M^{-1}t}p_0-\sqrt{\frac{2\gamma}{\beta}}\int_0^t\mathrm{e}^{\gamma P_{M,F}^\perp M^{-1}(s-t)} P_{M,F}^\perp\dif W_s\\
    &=\mathrm{e}^{-\gamma P_{M,F}^\perp M^{-1}t}p_0-\sqrt{\frac{2\gamma}{\beta}}\left(\int_0^t\e^{-\gamma P_{M,F}^\perp M^{-1}s}P_{M,F}^\perp\e^{-\gamma P_{M,F}^\perp M^{-1}s} \dif s\right)^{\frac12}G,
    \end{aligned}
\end{equation}
where the last equality holds in law if $G$ is a standard $dN$-dimensional Gaussian, using Itô's isometry and the symmetry of $\e^{\gamma P_{M,F}^\perp M^{-1}t}$ for all $t$.
The matrix exponentials and integrals can be computed, however there is no concise expression like \eqref{eq:propagators} because $P_{M,F}^\perp M^{-1}$ is not invertible.

In the case of a homogeneous system with $M=\alpha\Id$ (in which case we can assume by a judicious choice of mass unit that $M=\Id$), a more concise expression can be found, using the identity
\[\e^{r\Pi}=\sum_{k=0}^\infty \frac{r^k}{k!}\Pi^k=\Id+\left(\sum_{k=1}^\infty\frac{r^k}{k!}\right)\Pi=\Id+\left(\e^r-1\right)\Pi.\]
This identity follows, for $\Pi$ a linear projector, from applying the projector identity \[\Pi=\Pi^2=\Pi^3=\dotsm.\]
We write the following derivation in the case of a general projector $\Pi$, and conclude by taking the case $\Pi=P_{M,F}^\perp$. We also note $\Pi^\perp=\Id-\Pi$.

\begin{equation}
    \begin{aligned}
        \mathrm{e}^{-\gamma \Pi t}p_0-\sqrt{\frac{2\gamma}{\beta}} \Pi\left(\int_0^t\e^{2\gamma s \Pi }\dif s\right)^{\frac12}G&=\left(\Id+\left(\e^{-\gamma t}-1\right)\Pi\right)p_0-\sqrt{\frac{2\gamma}{\beta}}\Pi\left(\int_0^t\left(\Id+\left(\e^{-2\gamma s}-1\right)\Pi\right)\dif s\right)^\frac{1}{2}G\\
        &=\left(\Pi^\perp +\e^{-\gamma t}\Pi\right)p_0-\sqrt{\frac{2\gamma}{\beta}}\Pi\left(t\Pi^\perp+\frac{1-\e^{-2\gamma t}}{2\gamma}\Pi\right)^{\frac 12}G.
    \end{aligned}
\end{equation}
Now, observe that $\R^{dN}$ writes as the sum of two orthogonal spaces, each of which is an eigenspace for both $\Pi$ and $\Pi^\perp$, and correspond respectively to the eigenvalues $0$ and $1$ (this correspondence being flipped upon passing from $\Pi$ to $\Pi^\perp$). By an easy spectral calculation, we also have $\Pi^\alpha=\Pi$ for any $\alpha>0$,
which allows to conclude that 
\[(a\Pi^\perp+b\Pi)^\alpha=a^\alpha\Pi^\perp+b^\alpha\Pi\qquad \forall\,a,b,\alpha>0,\]
and finally
\begin{equation}
    \label{eq:norton_ou_solved_no_mass}
    \begin{aligned}
    \mathrm{e}^{-\gamma \Pi t}p_0-\sqrt{\frac{2\gamma}{\beta}} \Pi\left(\int_0^t\e^{2\gamma s \Pi }\dif s\right)^{\frac12}G&=\Pi^\perp p_0+\e^{-\gamma t}\Pi p_0-\sqrt{\frac{2\gamma t}{\beta}}\Pi\Pi^\perp G-\sqrt{\frac{1-\e^{-2\gamma t}}{\beta}}\Pi G\\
    &=\Pi^\perp p_0+\e^{-\gamma t}\Pi p_0-\sqrt{\frac{1-\e^{-2\gamma t}}{\beta}}\Pi G,
    \end{aligned}
\end{equation}
where we use $\Pi\Pi^\perp=0$ for the last equality.
Comparing equation \eqref{eq:norton_ou_solved_no_mass} and equation \eqref{eq:propagators} yields a clear interpretation of the action of the fluctuation-dissipation term on the dynamics: there is no effect in the direction tangent to $F$, and a standard Ornstein-Uhlenbeck process applies to the dynamics projected onto the subspace orthogonal to $F$ (note $\Pi G$ can then be viewed as a $(dN-1)$-dimensional standard Gaussian by isotropy).
We also note that a similar computation may be performed in the case where both $M$ and $P_{M,F}$ are simultaneously diagonalizable, for instance if $M$ is a diagonal matrix and $F$ has a single non-zero component.

\section{Splitting schemes}
Consider the Norton dynamics on the state variable $(q_t,p_t)$. Its generator is given by the operator

\begin{equation}
    \label{eq:norton_generator}
    \cL_F\varphi(q,p)=(M^{-1}p)\cdot \nabla_q\varphi(q,p)-(P_{M,F}^\perp\nabla V(q)-\gamma P_{M,F}^\perp M^{-1}p)\cdot\nabla_p\varphi(q,p)-\frac{\gamma}\beta\mathrm{Tr}\left(P_{M,F}^\perp\nabla_p^2\varphi(q,p)\right),
\end{equation}
which we rewrite as 
\begin{equation}
    \label{eq:norton_generator_splitting}
    \cL_F=A+B_F+\gamma C_F,\qquad \text{with} \qquad
    \left\{\begin{aligned}
        B_F\varphi(q,p)&=\left(P_{M,F}^\perp\nabla V(q)\right)\cdot \nabla_p\varphi(q,p)\\
        C_F\varphi(q,p)&=-\left(P_{M,F}^\perp M^{-1}p\right)\cdot\nabla_p\varphi(q,p)+\frac1\beta\mathrm{Tr}\left(P_{M,F}^\perp\nabla_p^2\varphi(q,p)\right).
    \end{aligned}\right.
\end{equation}
The $A$ and $B_F$ operators are the generators of exactly integrable dynamics (namely ballistic Hamiltonian dynamics), and $C_F$ generates an analytically solvable diffusion process given in \eqref{eq:projected_ou_analytical_form}. Hence we can use a splitting scheme exactly as we did in the case of the Langevin dynamics, as discussed in section \ref{section:splitting_schemes_langevin}. 
We will use the exact same terminology in this context, referring to a scheme by the order in which the elementary dynamics are integrated.
For instance, the BAO scheme associated with the Norton dynamics for $F$ is the following. We fix a timestep $\Dt>0$.

\begin{example}[Norton BAO scheme]
    \begin{equation}
        \left\{
            \begin{aligned}
                p^{n+\frac12}&=p^n-\Dt P_{M,F}^\perp \nabla V(q^n)\\
                q^{n+1}&=q^n+\Dt M^{-1}p^{n+\frac12}\\
                p^{n+1}&=\mathrm{e}^{-\gamma P_{M,F}^\perp M^{-1}\Dt}p^{n+\frac12}-\sqrt{\frac{2\gamma}{\beta}}\left(\int_0^\Dt\e^{-\gamma P_{M,F}^\perp M^{-1}s}P_{M,F}^\perp\e^{-\gamma P_{M,F}^\perp M^{-1}s} \dif s\right)^{\frac12}G^n
            \end{aligned}\right.
    \end{equation}
\end{example}
The $\gamma C_F$-step is cumbersome to write in full, but it is always of the form 
\[p^{n+1}=R_\Dt p^n + S_\Dt G^n\]
for fixed matrices $R_\Dt$ and $S_\Dt$ which only have to be computed once.

\section{Estimating transport coefficients}
To estimate the transport coefficient $\rho_F$ using this method, we need to estimate the average forcing. Recalling the SDE \eqref{eq:norton_final_sde} for $\dif \Lambda$, the martingale term involving the Brownian motion cancels out under steady-state averaging.
Hence we can estimate the average forcing using the following ergodic average:
\begin{equation}
    \label{eq:norton_ergodic_forcing}
    \frac{1}T\int_0^T\frac{M^{-1}F}{\left(M^{-1}F\right)\cdot F}\cdot\left(\nabla V(q_t)+\gamma M^{-1}p_t\right)\dif t.
\end{equation}
In the case of a mass matrix $M=m\Id$ and $|F|=1$, which we now assume on to lighten notation, this becomes
\begin{equation}
    \label{eq:norton_ergodic_forcing_homogeneous}
    \frac{1}{T}\int_0^T F\cdot\left(\nabla V(q_t)\right)\dif t + \gamma v,
\end{equation}
using the expression for the response, so that the finite difference estimator of $\rho_F$ is given by
\begin{equation}
    \label{eq:norton_transport estimator}
    \widehat{\rho}_{F,T}=\frac{v}{\gamma v+\frac{1}{T}\int_0^T F\cdot\left(\nabla V(q_t)\right)\dif t}.
\end{equation}
In practice, we approximate the continuous ergodic average in the denominator by discrete averages.

\section{Norton dynamics for shear viscosity}

\subsection{The Thevenin method}
We index the state as \[\left(q_{ij}\right)_{\substack{\leq i\leq N\\1\leq j\leq d}},\qquad \left(p_{ij}\right)_{\substack{\leq i\leq N\\1\leq j\leq d}}.\]
The Thevenin dynamics we considered writes, for $i=1\dots N$,
\begin{equation}
    \label{eq:thevenin_sv_dynamics}
    \left\{\begin{aligned}
        \dif q_{ij,t}&=\frac{p_{ij,t}}m\dif t,&1\leq j\leq d,\\
        \dif p_{i1,t}&=-\nabla_{q_{i1,t}}V(q_t)\dif t-\gamma_1\frac{p_{i1,t}}m+\sqrt{\frac{2\gamma_1}\beta}\dif W_t^{i1}+\xi F(q_{i2,t}),&\\
        \dif p_{ij,t}&=-\nabla_{q_{ij,t}}V(q_t)\dif t-\gamma \frac{p_{ij,t}}m+\sqrt{\frac{2\gamma}\beta}\dif W_t^{ij},&2\leq j\leq d.
    \end{aligned}\right.
\end{equation}
This equation describes a system undergoing standard Langevin evolution, with the longitudinal momentum coordinates 
\[\left(p_{i1}\right)_{1\leq i\leq N}\]
subject to a forcing $F$ which is constant in time, but depends on the corresponding transverse position coordinates
\[\left(q_{i2}\right)_{1\leq i\leq N}.\]
Additionally we allow the longitudinal fluctuation dissipation term to have a distinct friction coefficient $\gamma_1$, which is a free parameter of the model.

\subsection{A Norton method}
We derive a corresponding Norton method by constraining the velocity profile in the $x$ direction to be constant, and proportional to $F$.
Unlike the mobility case where the constraint can be enforced by a single Lagrange multiplier, fixing the whole velocity profile requires a corresponding forcing intensity profile.
Formally, we write
\begin{equation}
    \left\{\begin{aligned}
        \dif q_{ij,t}&=\frac{p_{ij,t}}m\dif t,&1\leq j\leq d,\\
        \dif p_{i1,t}&=-\nabla_{q_{i1,t}}V(q_t)\dif t-\gamma_1\frac{p_{i1,t}}m+\sqrt{\frac{2\gamma_1}\beta}\dif W_t^{i1}+\dif\xi_t\left(q_{i2,t}\right) F\left(q_{i2,t}\right),&\\
        \dif p_{ij,t}&=-\nabla_{q_{ij,t}}V(q_t)\dif t-\gamma \frac{p_{ij,t}}m+\sqrt{\frac{2\gamma}\beta}\dif W_t^{ij},&2\leq j\leq d.
    \end{aligned}\right.
\end{equation}
where $\xi$ is now a process indexed by both the time and the transverse coordinate in $L_y \mathbb T$.
As in the case of the mobility, we can restate the dynamics with no reference to $\xi$, by enforcing the constraint
\begin{equation}
    \label{eq:norton_shear_viscosity_constraint}
    \frac{p_{i1,t}}m=vF\left(q_{i2,t}\right)\qquad \forall\, 1\leq i\leq N.
\end{equation}
By Ito's lemma, since $q$ is a bounded variation process, we get
\[\dif{p_{i1,t}}=mvF'\left(q_{i2,t}\right)\dif q_{i2,t}=vF'\left(q_{i2,t}\right)p_{i2,t}\dif t.\]
This allows for a formal rewriting 
\begin{equation}\label{eq:norton_sv_multiplier}\dif\xi_t\left(q_{i2,t}\right)= \frac 1{F\left(q_{i2,t}\right)}\left(\nabla_{q_{i1,t}}V(q_t)\dif t+\gamma_1\frac{p_{i1,t}}m-\sqrt{\frac{2\gamma_1}\beta}\dif W_t^{i1}\right)=\gamma_1v+\frac 1{F\left(q_{i2,t}\right)}\left(\nabla_{q_{i1,t}}V(q_t)\dif t-\sqrt{\frac{2\gamma_1}\beta}\dif W_t^{i1}\right)\end{equation}
of the forcing term. We finally can enforce the constraint to obtain an expression for the Norton dynamics.
\begin{equation}
    \label{eq:norton_sv_dynamics}
    \left\{\begin{aligned}
        \dif q_{ij,t}&=\frac{p_{ij,t}}m\dif t,&1\leq j\leq d,\\
        \dif p_{i1,t}&=vF'\left(q_{i2,t}\right)p_{i2,t}\dif t,&\\
        \dif p_{ij,t}&=-\nabla_{q_{ij,t}}V(q_t)\dif t-\gamma \frac{p_{ij,t}}m+\sqrt{\frac{2\gamma}\beta}\dif W_t^{ij},&2\leq j\leq d.
    \end{aligned}\right.
\end{equation}

We expect that it is possible to recover the velocity profile obtained from Thevenin dynamics from the Norton dynamics in a manner analogous to the mobility case. Formally, we would like the process $\xi$ to be well-defined, and ergodic averages
\[\frac 1T\int_0^T \dif \xi_s(Y)=\frac{\xi_T(Y)-\xi_0(Y)}T\]
to converge to some well defined 
At this point, let us remark that, unlike the mobility case where an explicit SDE for the forcing term was easily derived, we do not obtain an expression for $\xi_t(y)$, which we would expect under some form of stochastic PDE.
However, equation \eqref{eq:norton_sv_multiplier} allows us to numerically estimate the response profile under a steady state for the Norton dynamics, by using a discretization procedure analogous to the one described in (reference article).
To this effect, we introduce a partition of the transverse domain 
\[\left[0,L_y\right)=\bigcup_{0\leq s\leq k-1}\left[s\frac{L_y}k,(s+1)\frac{L_y}k\right)\]
into $k$ bins. We can estimate the normalized average response profile over the $s$-th bin is given by the following estimator:
\begin{equation}
    \label{eq:norton_sv_response_estimator}
    \frac{k}{N_{\mathrm{iter}}N}\sum_{n=1}^{N_{\mathrm{iter}}}\sum_{i=1}^N\1_{s\frac{L_y}k\leq q_{i2}^n\leq (s+1)\frac{L_y}k}\frac{F(q_{i2}^n)}{\nabla_{q_{i1}}V(q^n)+\gamma_1vF(q_{i2}^n)},
\end{equation}
where the $(q^n,p^n)$ are iterates of a discrete process approximating the continuous dynamics \eqref{eq:norton_sv_dynamics}.
Just as in the case of the mobility, we omit the Brownian term which vanishe
\subsection{Numerical integration}
\subsection{Numerical results}