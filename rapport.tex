\documentclass[a4paper,11pt,twoside,leqno]{report}

\usepackage[utf8]{inputenc} %Ligne PS

\newlength{\minipagewidth}
\setlength{\minipagewidth}{\textwidth}
\setlength{\fboxsep}{3mm}

\usepackage{amsmath,amsthm} 
\usepackage{amssymb,mathrsfs} 
\usepackage{bbm}
\usepackage{a4wide} 
\usepackage{graphicx}
\usepackage{physics}
\usepackage{xcolor,subfigure} 
\usepackage{enumerate}
\usepackage[normalem]{ulem}
\usepackage{csquotes}
\usepackage{cancel}
\usepackage{hyperref}

%----------biblio-----------
\usepackage{biblatex}
\addbibresource[]{bibliography.bib}

%------- styles ---------
\newtheorem{theorem}{Theorem}
\newtheorem{lemma}{Lemma}
\newtheorem{assumption}{Hypothesis}
\newtheorem{definition}{Definition}
\newtheorem{prop}{Proposition}
\newtheorem{remark}{Remark}
\newtheorem{corollary}{Corollary}
\newtheorem{example}{Example}
\newtheorem{conjecture}{Conjecture}
\newtheorem{algorithm}{Algorithm}
%----- definitions ----------
\DeclareMathAlphabet{\mathpzc}{OT1}{pzc}{m}{it}
\newcommand{\dps}{\displaystyle } 
\newcommand{\rme}{\mathrm{e}}
\newcommand{\ri}{\mathrm{i}} 
\newcommand{\cL}{\mathcal{L}}
% \newcommand{\cLs}{\mathcal{L}_\mathrm{s}}
% \newcommand{\cLa}{\mathcal{L}_\mathrm{a}}
\newcommand{\cC}{\mathcal{C}}
\newcommand{\cLs}{\mathcal{S}}
\newcommand{\cLa}{\mathcal{A}}
\newcommand{\Schur}{\mathfrak{S}_0}
\newcommand{\Schurb}{\mathfrak{S}_1}
\newcommand{\cLham}{\mathcal{L}_{\rm ham}}
\newcommand{\cLFD}{\mathcal{L}_{\rm FD}}
\newcommand{\cB}{\mathcal{B}}
\newcommand{\cM}{\mathcal{M}}
\newcommand{\cX}{\mathcal{X}}
\newcommand{\cD}{\mathcal{D}}
\newcommand{\eps}{\varepsilon}
\newcommand{\R}{\mathbb{R}}
\newcommand{\E}{\mathbb{E}}
\newcommand{\Id}{\mathrm{Id}} 
\newcommand{\Ran}{\mathrm{Ran}}
\newcommand{\cR}{\mathcal{R}}
\newcommand{\cH}{\mathcal{H}}
\newcommand{\cK}{\mathcal{K}}
\newcommand{\invAs}{\left[A^{-1}\right]_\mathrm{s}}
\newcommand{\subplus}{\textnormal{\texttt{+}}}
\renewcommand{\leq}{\leqslant}
\renewcommand{\geq}{\geqslant}
\renewcommand{\le}{\leqslant}
\renewcommand{\ge}{\geqslant}
\newcommand{\dt}{\mathrm{d}t}
\newcommand{\dx}{\mathrm{d}x}
%\renewcommand{\dp}{\text{d}p}
\newcommand{\dq}{\mathrm{d}q}
\newcommand{\dif}{\mathrm{d}}
\newcommand{\defeq}{\mathrel{\mathop:}=}
\newcommand{\eqdef}{\mathrel={\mathop:}}
\newcommand{\1}{\mathbbm{1}}
\newcommand{\e}{\mathrm{e}}
\newcommand{\Dt}{{\Delta t}}
\newcommand{\iid}{{\textit{i.i.d.} }}
\DeclareMathOperator*{\argmax}{argmax}
\DeclareMathOperator*{\argmin}{argmin}
\newcommand{\ind}{\mathrel{\perp\!\!\!\perp}}

%------------------------------------
\title { 
  %\vspace{-3cm}\includegraphics{output_enpc.ps} \vspace{1cm} \\
  \Large{\textbf{\underline{Sorbonne Université}\\}}
  \vspace{0,7cm}
  \large{\textbf{2021-2022\\}}
  \vspace{1,5cm} \huge{\textbf{Stage de Master 2\\}}
  \vspace{0,7cm}\Large{\textbf{Noé Blassel\\}}
  \vspace{1cm}
  \huge{\textbf{(Non)-equilibrium molecular dynamics and a Norton method for the estimation of transport coefficients\\}}
  \vspace{2cm}
  \large{\textbf{Projet réalisé  en collaboration avec le CERMICS\\ 
  Ecole Nationale des Ponts et Chaussées \\6 et 8 avenue Blaise Pascal\\
  Cité Descartes - Champs sur Marne\\
77455 Marne la Vallée Cedex 2 \\}}
\vspace{2cm}
\Large{\textbf{Tuteur : Gabriel Stoltz}}
}


\begin{document}

\maketitle

This document is the report for an internship which took place from February 1st to July 31st 2022, at the CERMICS laboratory in the École des Ponts. This is a research unit in applied mathematics,
comprising of teams working on problems arising from probability theory and optimization, as well as, more directly relevant to us, material science and molecular simulations.
It was conducted under the advisement of Gabriel Stoltz, and had two explicitly stated aims. 

One was to engage with Julia and its molecular simulation ecosystem by implementing various methods from equilibrium and non-equilibrium molecular dynamics inside the Molly package.
Eventually, this approach resulted in a one-week stay with Molly's main author, to integrate some of the implementations developed during the course of the internship inside Molly.

A second aim, more prospective from the scientific point of view, was to advance the understanding of the Norton method, 
which is a novel method for the computation of transport coefficients from molecular simulations, based on a dual approach from the standard non-equilibrium method, whereby the response is fixed and the average forcing needed to induce it is measured, instead of the usual reverse situation.
As of the end of this internship, we have proposed a numerical integration strategy for a class of Norton dynamics, and applied it to the cases of mobility and shear viscosity computations. 
Our numerical results suggest several avenues for future theoretical work, which will be continued in the PhD work of Shiva Darshan, starting from the fall of 2022.

The report is divided into five chapters and one appendix. The first chapter is dedicated to a basic introduction to some concepts in statistical mechanics which are relevant to molecular simulation.
The second and third chapter are dedicated to equilibrium averages, with the second's focus on a presentation of the different numerical methods involved, and the third's on answering a question that arose when examining's Molly native Langevin integrator,
which coincidentally was also under investigation at the same time by a team of theoretical chemists, Bettina Keller and Stefanie Kieninger. It is essentially compiled from the written communication we sent to them, presenting our understanding of the BAOA scheme.
The fourth and fifth chapters are dedicated to the non-equilibrium setting, with the fourth centered on a discussion of the standard methods (Thévenin and Green-Kubo), and the fifth on the presentation of the Norton method.
All chapters are supplemented with numerical examples, which are destined to illustrate some theoretical properties, or the viability of a given numerical method.
We conclude the report by a short appendix which highlight some of the thinking that went into the choice of Molly as a molecular simulation package, as well as pointing the reader to relevant source code.

Every example on a realistic system was implemented within Molly, and thus we wish to thank Joe Greener for creating this very flexible and pleasant to work with package,
 as well as for inviting us to stay for a fascinating week in Cambridge.
We also take advantage of this short introduction to thank Gabriel Stoltz for trusting us with this subject, for his precious advice, and more generally for introducing us to the fun and mathematically rich subject that is molecular simulation.
\newpage

\begin{center}
  \textbf{Notational conventions}
\end{center}
We convene that the gradient of a differentiable function $\varphi : \R^n \mapsto \R$ is a column vector-valued function
$$\nabla \varphi : \R^n \mapsto \R^n \defeq \R^{n\times 1}$$
Notationally,
$$\nabla = \begin{pmatrix} \partial x_1 \\ \vdots \\ \partial x_n\end{pmatrix}$$
So that the Hessian operator writes
$$\nabla^2 \defeq \nabla\nabla^\intercal = \begin{pmatrix}
\partial{x_1}\partial_{x_1} & \dotsm & \partial{x_1}\partial{x_n} \\
 \vdots & \ddots & \vdots \\
 \partial{x_n}\partial{x_1} & \dotsm & \partial{x_n}\partial{x_1}    
\end{pmatrix}$$
And for $f=(f_1,\dotsm, f_n)^\intercal: \R^n \mapsto \R^n$

\begin{equation}
  \label{eq:jacobian}
  \begin{aligned}
    \nabla f=\begin{pmatrix}
        \nabla^\intercal f_1 \\ \vdots \\ \nabla^\intercal f_n
    \end{pmatrix}=(\nabla \otimes f)^\intercal, && 
    \operatorname{div} f=\partial_{x_1}f_1+\dotsm+\partial_{x_n}f_n=\nabla^\intercal f
    \end{aligned}
\end{equation}


are respectively the Jacobian matrix and divergence of $f$.
\tableofcontents

\chapter{Introduction to molecular dynamics}
\section{The microscopic description of atomic systems}
Molecular dynamics, and computational statistical physics at large, aim at simulating on the computer the behavior of physical systems.
The hope is that one can infer quantities and properties of real-life interest from observing the results of numerical simulations, which may be relevant to understand the material properties of many-particle systems, or the nature of interactions in complex systems such as those found in biology.
Computational simulations can thus act as surrogate experiments in cases where experimental setups are hard to achieve, or measurements are impossible.
They can also be seen as surrogate tests of theoretical models, as they allow to test the validity of a mathematical description by comparing numerical predictions to experimental data. Molecular dynamics, in particular, is concerned with simulating atomic systems, most often (and as we shall systematically do) using a classical description.


We consider a system of $N$ particles evolving in $d$-dimensional space. The classical description contends that the \textit{state} of a system is the datum of the positions and momenta of every particle in the system. We can interpret this as the statement that, given full knowledge of the positions and momenta at some initial time, and of the forces at play, one can deduce exactly the positions and momenta at any future time.

\begin{definition}[Phase space]

    We describe the positions and momenta of the atoms as vectors
    $$ q=(q_{1,1},\dots,q_{1,d},\dots ,q_{N,1},\dots,q_{N,d})^\intercal \in \R^{dN},$$
    $$ p=(p_{1,1},\dots,p_{1,d},\dots ,p_{N,1},\dots,p_{N,d})^\intercal \in \R^{dN},$$

    where $q_i \defeq (q_{i,1},\dots, q_{i,d})^\intercal$ is the position vector of the $i$-th particle, and similarly for $p$. 
    In practice, it is often the case that each $q_i$ is restricted to some $d$-dimensional manifold $\cD$, called the configuration space. For our purposes, we will always take $\cD=\R^{d}$ or $\cD=(L\mathbb T)^{d}$, where $L>0$ is some size parameter. Phase space, then, is the set of possible microscopic states of the system, that is, the set
    $$\mathcal E=\cD^{N} \times \R ^{dN}$$

    Trajectories through phase space, that is functions

    $$\left\{\begin{aligned} \R_+  &\mapsto \mathcal E \\ 
                            t  &\mapsto  (p_t,q_t)\end{aligned}\right.,$$
    can be seen as describing time evolutions of the system, objects which will be central to our study.
\end{definition}

It is not clear \textit{a priori} why we should choose momenta to describe the kinetic quality of the system, rather than velocities. However it is of no importance since we can change from one description to the other via the relation
$$v=M^{-1}p,$$
where $M\in \R^{dN \times dN}$ is a diagonal matrix recording the masses of each particle ($d$ times per particle), and $v$ is the velocity vector.


In order to describe the evolution of the system's state, one must specify a dynamical law. This is done by giving a function

\begin{equation*}
    \left\{ \begin{aligned} \cD &\mapsto \R \\
                            q &\mapsto V(q) \end{aligned} \right. ,
\end{equation*}

$V$ whose gradient in the $i$-th particle's coordinates 
$$ \nabla_{q_i} V \defeq (\partial_{q_{i,d}},\dots , \partial_{q_{i,d}})^\intercal $$
gives minus the force vector acting on the $i$-th particle. In the case where $\cD = (L\mathbb T)^{dN}$, it will be convenient to think of $V$ as a function from $\R^{dN}$ to $\R$ which is $C^1$ and $L$-periodic in each direction.

$V$ is called the potential, and, as it encodes the dynamics of the system, it is of paramount importance. The time evolution of the system, then, is described by Newton's second law:

$$\frac{\text{d} p}{\dt}=-\nabla V(q)$$.

It will be convenient for our analysis to use of reformulation of Newton's equations, based on the Hamiltonian of a system.

\begin{definition}[Hamiltonian]
    The Hamiltonian of a classical system is its total energy, which is the sum of a kinetic energy term depending only on the momenta and a potential energy term depending only on the positions.

    \begin{equation}
        H(q,p)=\frac12p^\intercal M^{-1}p+V(q)
    \end{equation}
\end{definition}

Using the Hamiltonian, we can rewrite the classical equations of motion as

\begin{equation}
\label{Hamiltonian dynamics}
\begin{cases}
    \text d q_t=M^{-1}p_t\dt=\nabla_p H(q_t,p_t)\dt\\
    \text d p_t=-\nabla V(q_t)\dt=-\nabla_q H(q_t,p_t)\dt
\end{cases},
\end{equation}

The potential is the most important part of the microscopic description, and accordingly, the main problem in establishing a physical model of this kind is to determine potential functions which adequately capture the dynamic behavior of a given system. 
The choice of a classical description automatically implies a degree of approximation, since behavior arising from the laws of quantum mechanics, which may be relevant at a microscopic level, are described by Newton's law.
 Furthermore, if the aim is to simulate such systems numerically, computational constraints imply that some compromise has to be reached between theoretical accuracy and computational cost. 
 If, for small systems, it may be possible to simulate all atomic interactions, for larger or more complex systems, it is often to use potential functions which are both cheap from a computational point of view and empirically shown to be accurate enough for the purpose of a simulation.
 
 Our main numerical example will be the system given by the following potential, which is of this empirical form, and which is often used to describe the microscopic behavior of chemically inert fluids, such as Argon.

 \begin{example}[The Lennard-Jones fluid]
    We fix $L>0$, $d=3$, and $N$ the number of particles. The Lennard-Jones fluid is the classical system given by the potential

    \begin{equation}
        \label{Lennard-Jones potential}
        V_{\mathrm{LJ}}(q)=\sum_{i=1}^N\sum_{j<i} 4\varepsilon \left( \left(\frac{|q_i-q_j|}{\sigma}\right)^{-12} - \left(\frac{|q_i-q_j|}{\sigma}\right)^{-6} \right).        
    \end{equation}
     
    Note that $V_{\mathrm{LJ}}$ is given by a sum over pairs of particles,

    $$V_{\mathrm{LJ}}(q)=\sum_{1\leq i < j \leq N} v(|q_i-q_j|),$$

    where $v$ is a radial function

    $$v(r)=4\varepsilon \left( \left( \frac{\sigma}{r}\right)^{12}-\left(\frac{\sigma}{r} \right)^6\right).$$

    $\varepsilon$, an energy, and $\sigma$, a length, are shape parameters which respectively control the depth of the potential well of $v$ and the equilibrium distance $2^{1/6}\sigma$.
    As seen on Figure \ref{fig:lennard_jones}, the potential combines two effects. At small interparticular distances, the dominant term is in $r^{-12}$, which translates into a strongly repulsive force between close pairs of particles, and makes individual particles essentially impenetrable.
    At long range, the dominant term is in $-r^6$, which translates into a weakly attractive force between distant particles. Contrary to the repulsive term, which is empirical, this scaling has a theoretical origin in the Van der Waals forces.
    From a computational standpoint, the fact that $v$ is an even function of $r$ allows one to compute the normalized force while sparing the expense of computing a square root, while the identity $r^{12}=(r^6)^2$ allows further economy.
    The shape parameters $\sigma$ and $\varepsilon$ must be chosen empirically to describe the behavior of a particular atomic species. For Argon, values of reference are: $\sigma=$ , $\varepsilon=$.
 \end{example}

 \begin{figure}[htbp]
    \begin{center}
      \includegraphics[width=0.7\linewidth]{figures/chapter1/lennard_jones.pdf}
      \caption{ \label{fig:lennard_jones}
        The pair potential $v$, with lengths and energy given in reduced units. The equilibrium interparticular distance is indicated by the vertical dotted line.
      }
    \end{center}
  \end{figure}

\section{Reduced units}

\section{Statistical ensembles}
The microscopic description is interesting from a theoretical standpoint, but it fails to be relevant when attempting to describe the behavior of atomic systems with a macroscopic number of particles, of the order of Avogadro's number ($6.02 \times 10^{23} $).
Besides the technical impossibility of measuring to a high accuracy the configuration of such systems, and that of recording the information required to track it (coincidentally, the total amount of digitally stored information on Earth is estimated to be $10^{23}$ bytes as of 2022), it is also the case that knowledge of a system at this level of detail is unnecessary to describe the quantities which are relevant to our macroscopic experience.
In the instance of a gas at thermal equilibrium, examples of relevant quantities are total energy, pressure, temperature, density, which, while of course resulting from the internal state of the system, are independent of the minutiae of individual atomic motions: loosely speaking, one may describe the macroscopic state of a system by only a handful of macroscopic variables, loosing track of the myriad of microscopic degrees of freedom.
An important point is that for a given macroscopic state, there are many microscopic configurations which are compatible with our observations. This motivates defining the macroscopic state of a system as a probability distribution over phase space, which we may interpret as assigning to each microscopic configuration a likelihood that this configuration underlies the macroscopic state.

This does not tell one how to choose the distribution over microscopic states. However, it seems reasonable to assign positive probabilities to states compatible with the macroscopic constraints, and in such a way as to make the weakest possible assumptions on this microscopic state, or in other words contain the least amount of information about the system, given the macroscopic constraints. The mathematical translation of this idea is given by the principle of maximal entropy. Given a class of probability distributions compatible with the macroscopic constraints, define the macroscopic state as the one which maximizes the entropy, which is defined for a probability distribution $\rho$ by

\begin{equation}
    \label{entropy}
    \mathfrak S(\rho)=-\int_{\mathcal E} \rho(x)\ln(\rho(x))\dx.
\end{equation}

The specification of a probability distribution over states is called a thermodynamic ensemble. We will be considering two ensembles:

\begin{example}[Microcanonical ensemble]

    The microcanonical ensemble is the suitable model for an isolated system in thermodynamic equilibrium, evolving under Hamiltonian dynamics. The number of particles $N$, the volume $V=L^3$, and the energy $E$ is fixed. We will alternatively refer to the microcanonical ensemble as the NVE ensemble.
     Because the constant energy condition constrains the compatible microstates to level sets of $H$, which in general will be negligible subsets of $\mathcal E$, some care must be taken in defining the microcanonical measure, since one cannot express the macroscopic constraints by a family of probability densities.
      However, under suitable assumptions on $V$, one can define the microcanonical measure as a weak limit of uniform distributions over level \textquote{shells} of $H$:
    $$\int_\mathcal{E} \varphi \text d\mu_{NVE} \defeq \underset{\varepsilon \to 0}{\mathrm{lim}} \frac{1}{|S(E,\varepsilon)|}\int_{S(E,\varepsilon)} \varphi(q,p) \text d q \text d p,$$
    where 
    $$S(E,\varepsilon) = \{ (q,p) \in \mathcal E |\ H(q,p) \in [E-\varepsilon,E+\varepsilon]\}.$$
    
    This is consistent with the fact that, for a set $A$ with finite Lebesgue measure, the probability distribution on $A$ which maximizes the entropy is the uniform distribution on $A$. 
    It is possible, using the coarea formula, to derive a precise expression for this limit, which is, however, of limited practical interest.
    
\end{example}


\begin{example}[Canonical ensemble]
    Isolated systems in thermal equilibrium are not typically those that we encounter in experiments. Instead, it is more common to observe systems which are in thermal equilibrium with respect to their environment, an ambient \textit{heat bath} at a fixed temperature.
    The total energy of such systems is not fixed: small fluctuations can occur as energy is exchanged back and forth between the heat bath and the system. However, the average energy $\bar E$ is fixed. 
    This is the macroscopic constraint that defines the canonical ensemble. For a fixed $N,V,\bar E$, define the density of the the canonical measure as the maximizer:

    $$ \underset{\rho \in \mathcal A}{\mathrm{argmax}}\ \mathfrak S(\rho)$$
    where $\mathcal A$ is the set of admissible densities
    $$\mathcal A=\{ \rho: \mathcal E \mapsto \R_+ |\ \int_{\mathcal E} \rho =1, \int_{\mathcal E} H(q,p)\rho(q,p)\text d q \text d p=E \}.$$

    Solving the Euler-Lagrange equation associated with this constrained optimization problem yields that the only admissible solution can be written under the form:

    $$ \rho^*(q,p)= \frac 1{Z} e^{-\beta H(q,p)}.$$

    Furthermore, one can show that $\rho^*$ is indeed the unique maximizer.
    Here, $-\beta$ and $1+\ln Z$ are the critical Lagrange multipliers associated respectively with the energy constraint and the normalization constraint. Thus
    $$Z=\int_{\mathcal E} e^{-\beta H(q,p)}\text d q \text d p$$
    is a normalization constant called the partition function, and $\beta$ is a tuning parameter related to the value of $\bar E$. The physical interpretation of $\beta$ is that of an inverse temperature,

    $$ \beta = \frac 1{k_B T},$$
    where $k_B=1.38 \times 10 ^{-23} \mathrm{J\cdot K^{-1}}$ is Boltzmann's constant. For obvious reasons, we prefer to refer to the canonical ensemble as the NVT ensemble (rather than the NV$\bar {\text E}$)
\end{example}

\begin{remark}
One can go further and observe that when observing a fixed volume of unconfined gas in thermal equilibrium, the total number of particles $N$ is not fixed. Instead, this fluctuates as particles are constantly exchanged with an ambient particle reservoir. Instead, the average number of particles $\bar N$ is fixed. The resulting ensemble is called the grand canonical or $\mu$VT ensemble. This, and many other constructions are possible, but we will restrict our attention to the NVE and NVT cases. 
\end{remark}

The main interest in obtaining such a description is that one can then express the macroscopic state of a system in terms of averages of microscopic observables with respect to the ensemble measure.

\section{From microscopic dynamics to macroscopic observables}
As we stated above, the aim of molecular dynamics is to infer from computer simulations at the microscopic level the macroscopic behavior physical systems.
Even if systems which are within the reach of computer simulations are much smaller than macroscopic systems, the hope is that one can derive from the observation of certain quantities 
\chapter{Sampling equilibrium properties}
In this chapter, we discuss techniques to sample static properties of systems in thermodynamic equilibrium.
Our focus will be on the microcanonical and canonical ensembles, for which the corresponding sampling dynamics are described by the Hamiltonian and Langevin equations.
Consequently, the bulk of this chapter is dedicated to theoretical discussions of their properties, as well as the description of numerical integration strategies.

\section{Microcanonical averages}
We start by describing methods to sample microcanonical averages, first describing a few qualitative properties of Hamiltonian dynamics. 
These will serve as criteria to determine viable candidate numerical schemes, which will be required to preserve, either exactly or asymptotically,
those qualitative properties.
We then turn to the description of such structure-preserving schemes, and their energy conservation properties, concluding by a numerical illustration.

\subsection{Elementary properties of Hamiltonian dynamics}
The Hamiltonian dynamics \eqref{eq:hamiltonian_dynamics} rewrites in matrix form, with $X_t=(q_t,p_t)$:
\begin{equation}\label{eq:hamiltonian_dynamics_matrix_form} \text d X_t=J\nabla H(X_t)\,\dt,\end{equation}
where $J$ is the symplectic matrix

$$J = \begin{pmatrix}
    0_{dN} & \Id_{dN}\\ -\Id_{dN} & 0_{dN}.
\end{pmatrix}$$

Applying the chain rule to any smooth function $\varphi: \mathcal E \to \R$, we obtain
        $$ \text d \varphi(X_t)= \text d X_t^{\intercal} \nabla \varphi(X_t)=(J \nabla H(X_t))^\intercal \nabla \varphi(X_t)\dt=(\nabla_p H \cdot \nabla_q - \nabla_q H \cdot \nabla_p)\varphi(X_t)\dt$$
        This motivates the following.
        \begin{definition}[Generator of the Hamiltonian dynamics]
            We define the generator associated with the Hamiltonian dynamics to be the operator $\cL_{\text{H}}$ acting on smooth functions as
        \begin{equation}
            \label{eq:hamiltonian_generator}
            \cL_{\mathrm{ham}}\varphi=(\nabla_p H \cdot \nabla_q - \nabla_q H \cdot \nabla_p)\varphi=\left(J\nabla H\right)^\intercal \nabla \varphi.
        \end{equation}
    \end{definition}
    We can split the generator as the sum of two elementary operators,
    $$\cLham=A+B,$$
    with
    \begin{equation}
        \label{eq:Lham_splitting}
        A=\left(M^{-1}p\right)\cdot \nabla_q \qquad B=-\nabla V(q)\cdot \nabla_p.
    \end{equation}
    The generator allows us to quantify the rate of change of an observable $\varphi$ under the evolution of the system. If we define, for $t\geq 0$, the evolution operators 
    $$P_t \varphi (q_0,p_0) = \varphi(\Phi_t(q_0,p_0)),$$
where $\Phi$ is the flow associated with the Hamiltonian dynamics, that is the collection of maps $(\Phi_t)_{t\geq 0}$, defined by
    $\Phi_t (q_0,p_0) = (q_t,p_t)$, the unique solution to \eqref{eq:hamiltonian_dynamics} with initial conditions  $(q_0,p_0)$, then we formally have:

    $$ \frac \partial{\partial t} P_t \varphi (q,p)= \partial_t \varphi(q_t,p_t)= \cLham \varphi(q_t,p_t)=\cLham P_t \varphi(q,p)=P_t\cLham \varphi(q,p).$$
    
    In the following result, we collect certain qualitative properties of Hamiltonian dynamics.
    \begin{prop}[Properties of Hamiltonian dynamics]
        \label{prop:hamiltonian_properties}
        Assume that the Hamiltonian $H$ \eqref{eq:hamiltonian} is $C^2$ on~$\mathcal E$ and that the flow $\Phi_t$ is globally defined for $t\in\R$. Then the following properties hold.
        \begin{enumerate}[i)]
            \item Group structure: \[\forall\, t,s\in \R,\qquad \Phi_t \circ \Phi_s = \Phi_{t+s},\, \Phi_0=\Id.\]
            \item Energy preservation: \[\frac{\mathrm{d}H(q_t,p_t)}{\dt}=0.\]
            \item Conservation of the Lebesgue measure: \[\forall \,\text{Lebesgue-measurable}\,D\subseteq \mathcal{E},\,\forall t\geq 0,\qquad|\Phi_t(D)|=|D|.\]
            \item Symplecticity: \[\forall t\geq 0,\qquad\nabla \Phi_t^\intercal J \nabla \Phi_t = J.\]
            \item Time reversibility: \[\Phi_t \circ \cR \circ \Phi_t=\mathcal R.\]
        \end{enumerate}
        The notation $\nabla$ corresponds to \eqref{eq:jacobian}, and the map $\mathcal R$ is the momentum-reversing involution
        \[\mathcal R(q,p)=(q,-p).\]
    \end{prop}
    \begin{proof}[Hints of proof.]
        \begin{enumerate}[i)]
            \item This property expresses the fact that the Hamiltonian evolution is autonomous, and follows from uniqueness in the Cauchy--Lipschitz theorem. This allows one to formally interpret the flow as a group action of $\R$ on $\mathcal E$.
            \item The energy conservation property simply follows from applying $\cLham$ to $H$.
            \item This property, known as Liouville's theorem, holds generally for any divergent-free flow. Its proof is based on a time differentiation of the determinant $\det\left(\nabla \Phi_t\right)$, and observing that the Hamiltonian vector field is divergence free: \[\mathrm{div}\left(J\nabla H\right)=\mathrm{div}_q\left(\nabla_p H\right)-\mathrm{div}_p\left(\nabla_q H\right)=0.\]
            \item The property is trivially satisfied at time $t=0$. A straightforward calculation shows that the time derivative of the symplecticity condition for $\Phi_t$ is $0$, which proves the claim. Let us remark that this property also implies property iii), since it shows $\det(\nabla\Phi_t)^2\det(J)=\det(J) \implies |\det(\nabla\Phi_t)|=1$.
            \item This again follows by uniqueness of trajectories: by observing that, for a fixed initial condition $(q_0,p_0)$, a time differentiation of the trajectory $(q_{-t},-p_{-t})=\mathcal R\, \circ\, \Phi_{-t}(q_0,p_0)$  shows it is Hamiltonian. Thus it must coincide with $\Phi_t(q_0,-p_0)=\Phi_t \circ \mathcal R (q_0,p_0)$. This can be restated as an equality of mappings, $\mathcal R \circ \Phi_{-t}=\Phi_t \circ \mathcal R$. Precomposing by $\Phi_t$ on each side yields the result using property i).
        \end{enumerate}
    \end{proof}

    \begin{remark}
        \label{rem:non_separable_hamiltonian}
        Properties \textit{i)} to \textit{iv)} above are still valid for any dynamics of the form \eqref{eq:hamiltonian_dynamics_matrix_form}, thus it is possible to consider dynamics 
        with more general Hamiltonians, which still obey them, disregarding issues of well-posedness. Property \textit{v)}, however uses the additional
        property that the classical Hamiltonian is separable into a kinetic and potential part, and that the kinetic part is an even function of $p$.
    \end{remark}   

    Property \textit{ii)} in the proposition above asserts that Hamiltonian trajectories remain on the constant energy manifold $S(H(q_0,p_0))$ defined in \eqref{eq:constant_energy_manifold}.
    Since this is the support of the microcanonical measure $\mu_{\mathrm{mc,E}}$, it is natural to ask whether the Hamiltonian dynamics can be used to sample the microcanonical measure, by means of ergodic averages.
    A minimum requirement for this to hold is that the measure is left invariant under Hamiltonian evolution. This is indeed the case. Consider a Hamiltonian trajectory $(q_t,p_t)$ such that $H(q_0,p_0)=E$. For $g$ and $f$ test functions,

    \begin{align*}
        \int_{\R} g(E)\int_{\mathcal E}f(q_t,p_t)\,\mathrm{d}\mu_{\mathrm{mc,E}}(q,p)\,\mathrm{d}E&=\frac1{Z_{\mathrm{E}}}\int_{\mathcal E} g(H(q,p))\left(f\circ \Phi_t\right) (q,p)\,\mathrm{d}q\,\mathrm{d}p\\
        &=\frac1{Z_{\mathrm{E}}}\int_{\mathcal E} g(H\circ \Phi_{-t}(\tilde q,\tilde p))f (\tilde q,\tilde p)\,\mathrm{d}\tilde q\,\mathrm{d}\tilde p\\
        &=\frac1{Z_{\mathrm{E}}}\int_{\mathcal E} g(H(\tilde q,\tilde p))f(\tilde q,\tilde p)\,\mathrm{d}\tilde q\,\mathrm{d}\tilde p\\
        &=\int_{\R}g(E)\int_{\mathcal E}f(\tilde q,\tilde p)\,\mathrm{d}\mu_{\mathrm{mc,E}}(q,p)\,\mathrm{d}E.
    \end{align*}
    the absence of a Jacobian determinant term in the change of variables from the second to the third line follows from property iii) in Proposition \ref{prop:hamiltonian_properties}, while the passage from the second to the third line is justified by the energy conservation property.
    This shows that the microcanonical measure is invariant under the Hamiltonian dynamics, but it is easy to construct examples where ergodic averages fail to converge to the correct value, as the following simple example shows.

    \begin{example}[A non-ergodic system]\label{ex:non_ergodic_system}
        We consider the following one-dimensional system with $\mathcal E= \mathbb T \times \R$, and the potential given by 
        \[V(q)=\cos\left(4\pi\left[q-\frac12\right]\right).\]
        
        The constant-energy manifold $S(0)$ consists of two disjoint compact connected components (see Figure \ref{fig:non_ergodic_system}). The observable
        \[\varphi(q,p)=\1_{q>\frac12}-\1_{q<\frac12}\]
        is constant on each component of $S(0)$, and has zero average with respect to $\mu_{\mathrm{mc,0}}$ by symmetry, but ergodic averages do not converge, since 
        \[\frac{1}{T} \int_0^T \varphi(q_t,p_t)\mathrm{d}t =  \frac{1}{T}\int_0^T \varphi(q_0,p_0)\mathrm{d}t=\varphi(q_0,p_0)\,\in\{\pm 1\}.\]
        \begin{figure}[htbp]
            \begin{center}
              \includegraphics[width=0.7\linewidth]{figures/chapter1/ergodicity.pdf}
              \caption{ \label{fig:non_ergodic_system}
                The Hamiltonian landscape for the potential $V$ of Example \ref{ex:non_ergodic_system} , with $S(0)$ plotted in red.
              }
            \end{center}
          \end{figure}
    \end{example}

    The issue of ergodicity is in general very hard to tackle for realistic systems, and in practice is often taken as a working hypothesis.
     Besides, it may happen that for observables satisfying some symmetries in the constant energy manifold converge to the correct microcanonical average even in the absence of ergodicity.
     This is, for example, the case for the kinetic energy observable in our Example \ref{ex:non_ergodic_system}.

    \subsection{Numerical schemes for Hamiltonian dynamics}

    It not impossible, except for a restricted class of systems, which do not typically arise in practice, to analytically integrate Hamilton's equation \eqref{eq:hamiltonian_dynamics}. For this reason, one must resort to numerical schemes, which provide approximations of the flow map over one timestep.
    More precisely, for a fixed timestep $\Delta t$, if one has an approximation of the flow 
    \[\widetilde{\Phi}_{\Delta t}\approx\Phi_{\Delta t},\]
    one can deduce discrete approximations of the evolution by iteration:
    \begin{equation}\label{eq:discrete_dynamics}(q^n,p^n)\defeq \widetilde{\Phi}_{\Delta t}^n(q_0,p_0)\approx (q_{n\Delta t},p_{n\Delta t}),\end{equation}
    which can then be used as sample points for the computation of empirical averages, discrete counterparts to the ergodic averages \eqref{eq:ergodic_averages},
    \begin{equation}
        \label{eq:discrete_ergodic_averages}
        \frac 1N_{\mathrm{iter}}\sum_{k=0}^{N_{\mathrm{iter}}} \varphi(q^k,p^k).
    \end{equation}
    In most common applications involving ordinary differential equations, the aim is to approximate the exact solution as precisely as possible over a given domain.
    In the case of sampling trajectory averages in molecular dynamics, however, the time domain is usually very large, because simulating long trajectories is a requirement to ensure that a representative portion of phase space is explored. As a consequence, it is in practice impossible to obtain precise solutions over a long time at the level of trajectories, because the evolution's sensitivity to initial conditions will cause small initial errors to rapidly blowup. 
    Furthermore, one does not really \textit{care} about the exact evolution, since the dynamics are merely used as a sampling device. Instead, one key requirement is that the dynamics stay on or close to the constant energy manifold associated with a given initial condition. It can be shown through eigenanalysis that for simple linear systems, this requirement is not satisfied by standard ODE numerical methods such as the explicit and implicit Euler schemes, or the RK4 method, for which the energy may exponentially increase or decrease.
    This has the practical effect that for reasonably sized atomic systems, numerical instabilities render the simulations nonsensical after only a few time steps, which is far from what is needed to obtain good estimates.
    One must then devise dedicated numerical methods, guided by the aim to preserve qualitative properties of the Hamiltonian evolution. It turns out that splitting schemes, based on operator splitting approximations of the Hamiltonian evolution operator over one timestep, preserve crucial qualitative properties of the Hamiltonian evolution.

    An important observation is that if one considers each part of \eqref{eq:Lham_splitting} as a generator in its own right, the corresponding elmentary dynamics are analytically integrable.
    \begin{remark}
        \label{rem:Lham_splitting_semigroups}
        Consider the two dynamics defined by
        \begin{equation}
            \label{eq:Lham_splitting_dynamics_A}
            \left\{\begin{aligned}
                \dif q_t^A&=M^{-1}p_t^A\dif t,\\
                \dif p_t^A&=0,
            \end{aligned}\right.
        \end{equation}
        and
        \begin{equation}
            \label{eq:Lham_splitting_dynamics_B}
            \left\{\begin{aligned}
                \dif q_t^B&=0,\\
                \dif p_t^B&=-\nabla V(q_t^B)\dif t.
            \end{aligned}\right.
        \end{equation}
        These can be analytically solved as
        \begin{equation}
            \label{eq:Lham_splitting_dynamics_solved}
            \left\{\begin{aligned}
                &\left(q_t^A,p_t^A\right)=\left(q_0^A+tp_0^A,p_0^A\right),\\
                &\left(q_t^B,p_t^B\right)=\left(q_0^B,p_0^B-t\nabla V\left(q_0^B\right)\right).
            \end{aligned}\right.
        \end{equation} 
        Moreover, these evolutions are of Hamiltonian form, with Hamiltonians corresponding respectively to the kinetic part and the configurational part only, and have corresponding generators $A$ and $B$.
        We denote by $\left(\Phi^A_t\right)_{t \in \R}$ and $\left(\Phi^B_t\right)_{t\in\R}$ their respective flow maps.
    \end{remark}
    This observation suggests the following general recipes to construct a class of numerical schemes for the Hamiltonian dynamics, named splitting schemes. We consider approximations of the form
    \begin{equation}\label{eq:flow_splitting_approximation}\Phi_{\Delta t}\approx \Phi^{G_k}_{\Delta t_k}\circ \dotsm \circ \Phi^{G_1}_{\Delta t_1},\end{equation}
    where $G_i\in\{A,B\}$ for all $i$ and $\sum_{G_i=A}\Dt_i=\sum_{G_i=B}\Dt_i=1$. We will be considering three schemes, the simplest of which are the symplectic Euler schemes.
    The symplectic Euler schemes are defined by the following update equations:
    
    \begin{equation}\label{eq:symplectic_euler_A}
    \left\{\begin{aligned}
         p^{n+1} &=p^n -\nabla V(q^n)\Delta t,\\
         q^{n+1} &=q^n + M^{-1}p^{n+1}\Delta t,
    \end{aligned}\right.
    \end{equation}
    and
    \begin{equation}\label{eq:symplectic_euler_B}
        \left\{\begin{aligned}
             q^{n+1} &=q^n +M^{-1}p^n\Delta t,\\
             p^{n+1} &=p^n - \nabla V(q^{n+1})\Delta t.
        \end{aligned}\right.
    \end{equation}

    These correspond respectively to the splittings
    $\Phi_{\Delta t}^A\circ\Phi_{\Delta t}^B\defeq \Phi_{\Delta t}^{BA}$
    and
    $\Phi_{\Delta t}^B\circ\Phi_{\Delta t}^A\defeq \Phi_{\Delta t}^{AB}$.
    The velocity Verlet scheme is based on the symmetric splitting
    $\Phi_{\Delta t/2}^B\circ\Phi_{\Delta t}^A\circ\Phi_{\Delta t/2}^B\defeq \Phi_{\Delta t}^{BAB}$,
    Its update equation is given by
    \begin{equation}\label{eq:verlet}
        \left\{\begin{aligned}
             p^{n+\frac12} &=p^n - \frac{\Delta t}{2}\nabla V(q^n)\\
             q^{n+1} &=q^n + \Delta t M^{-1}p^{n+\frac 12}\\
             p^{n+1} &= p^{n+\frac12}-\frac{\Delta t}{2}\nabla V(q^{n+1}).
        \end{aligned}\right.
    \end{equation}
    We have announced above that these numerical schemes preserve some qualitative properties of the Hamiltonian dynamics. We now turn to making this statement precise.
    Let us fix an evolution operator $\tilde\Phi_{\Delta t}$ corresponding to a splitting of the form \eqref{eq:flow_splitting_approximation} with a timestep $\Dt>0$.
    Recall the properties in \ref{prop:hamiltonian_properties}. Then analogous properties can be stated for each of the schemes. We will refer to these analogous using the same names and indexing, but where $\Phi_t$ is replaced by $\tilde\Phi_{\Delta t}$ regardless of $t$.
    We further say a splitting is \textit{symmetric} if the corresponding order of operators $A$ and $B$ is a palindrome. The velocity Verlet splitting is symmetric, while the symplectic Euler splittings are not.
    We may then go through the properties in Proposition \ref{prop:hamiltonian_properties}, listing those which apply, and those which have to be modified.
        \begin{enumerate}[i)]
            \item Group structure: the analogous statement is a group action of $\mathbb{Z}$ on $\mathcal E$. This holds if the splitting is symmetric: then, $\tilde\Phi_{\Dt}^{-1}=\tilde\Phi_{-\Dt}$.
            \item Energy preservation: this does not hold as is, but does in a weakened sense that we discuss below.
            \item Conservation of the Lebesgue measure: this holds for all splittings.
            \item Symplecticity: similarly, this holds for all splittings.
            \item Time-reversibility: this holds for all symmetric splittings.
        \end{enumerate}
    \begin{proof}[Hints of proofs]
        For every one of these properties, the strategy is the same. From Proposition \ref{prop:hamiltonian_properties}, each one of them holds for $\Phi_{\Dt}^A$ and for $\Phi_{\Dt}^B$.
        The aim is then to show that these properties are stable, at best under composition, and at worst under symmetric composition.
        \begin{enumerate}[i)]
            \item This simply follows from writing \[\left(g_1 \circ g_2 \circ \dotsm \circ g_2\circ g_1\right)^{-1}=g_1^{-1}\circ g_2^{-1}\circ \dotsm \circ g_2^{-1}\circ g_1^{-1},\] and using property \textit{i)} applied to $\Phi_{\Dt}^A$ and $\Phi_{\Dt}^B$.
            \item This is a crucial and slightly subtle point, to which we return in more detail below.
            \item This follows trivially by composition (or alternatively by symplecticity). For any measure preserving measurable maps $f$ and $g$, \[|f \circ g(D)|=|g(D)|=|D|.\] 
            \item This follows from the fact that any composition of symplectic maps is symplectic. This, in turn, follows from the multivariate chain rule below, and applying the symplectic property twice: \[\nabla(g\circ f)= \left((\nabla g)\circ f\right)\nabla f\implies \left[\left((\nabla g)\circ f\right)\nabla f\right]^\intercal J\left[\left((\nabla g)\circ f\right)\nabla f\right]=\nabla f^\intercal J \nabla f=J.\]
            \item Fixing $f \circ \mathcal R \circ f = g \circ \mathcal R \circ g = \mathcal R$, we write \[f\circ g \circ \mathcal R \circ g \circ f = f \circ \mathcal R \circ f = \mathcal R,\] and conclude by induction that the property holds for any symmetric splitting.
        \end{enumerate}
    \end{proof}

    We have shown that splitting schemes inherit some nice geometrical properties from the underlying Hamiltonian flow, and all the more for symmetric splittings.
    However our final aim is to sample from the microcanonical measure, hence we should aim to sample points which remain close to the constant energy manifold $S(E)$.
    Ideally, we would want to guarantee that the Hamiltonian is perfectly preserved under the discrete evolution induced by these schemes. This, it turns out, is too high a hope.
    It happens, however, that, for each of these schemes, a \textit{perturbed} Hamiltonian is (almost) exactly conserved, which we can interpret as the discrete dynamics \eqref{eq:discrete_dynamics} regularly  and (almost) exactly sampling points from a \textit{perturbed} dynamics corresponding to the new Hamiltonian.
    The order of this perturbation in the timestep $\Delta t$ then allows one to quantify the error over finite time intervals. The statements and proofs of this kind of results fall under the broad scope of backward numerical analysis. For a clear and more detailed introduction, one can consult \cite[Section 4]{HLG03}.
    The general idea of backward numerical analysis, given an evolution equation and an associated numerical method:
    \[\dot y = F_0(y),\qquad y^{n+1} = \tilde\Phi_{\Dt}(y),\]
    is to reinterpret the numerical solution $y^1$ not as an approximation of the exact solution $y_{\Dt}$, but as the exact solution $\tilde y_{\Dt}$ of an approximate evolution $\tilde y$, given by the ODE
    \[\dot{\tilde y}=\tilde F(y),\]
    where $\tilde F$ is given by a perturbative expansion
    \[\tilde F=F_0 + \Dt F_1 + \Dt^2 F_2 +\dots\]
    To avoid any convergence issue related to infinite expansions, precise statements usually truncate the expansion at some finite order $\alpha>0$, and require the exactness of the numerical trajectory up to order $\alpha+1$, say
    \[\tilde F=F_0 + \Dt F_1 + \dots + \Dt^\alpha F_\alpha,\qquad |\tilde \Phi_\Dt(y_0) - \tilde y_\Dt|= \mathrm{O}(\Dt^{\alpha+2}).\]
    Comparing Taylor expansions in powers $\Dt$ of $\tilde\Phi_\Dt$ and $\tilde y_{\Dt}$ then allows us to explicitly compute the terms in the expansion of $\tilde F$.
    As an example we compute the first correction term for the symplectic Euler scheme $\Phi_{\Dt}^{AB}$.
    \begin{example}[Leading-order correction for $\Phi_{\Dt}^{AB}$]
        \label{ex:modified_hamiltonian}
         Following our strategy, we Taylor-expand our scheme as
        \[\Phi_{\Dt}^{AB}(q,p)=\begin{pmatrix}q+\Dt M^{-1}p \\ p- \Dt \nabla V(q) - \Dt^2 \nabla^2 V(q) M^{-1}p\end{pmatrix}+\mathrm{O}(\Dt^3).\]
        Similarly, expanding the exact solution with initial condition $(q,p)=y_0$ to the second order yields
        \begin{align*} \Phi_\Dt(q,p)&= y_0 +\Dt J\nabla H(y_0)+\frac{\Dt^2}2\nabla\left[J\nabla H(y_0)\right]J\nabla H(y_0)+\mathrm{O}(\Dt^3)\\
        &=y_0+\Dt J\nabla H(y_0) + \displaystyle{\frac{\Dt^2}{2}}J\nabla^2 H(y_0)J \nabla H(y_0)+\mathrm{O}(\Dt^3)\\
        &=\begin{pmatrix}q+\Dt M^{-1}p -\displaystyle{\frac{\Dt^2}{2}}M^{-1}\nabla V(q) \\ p -\Dt \nabla V(q)-\displaystyle{\frac{\Dt^2}{2}}\nabla^2 V(q)M^{-1}p\end{pmatrix}+\mathrm{O}(\Dt^3).\end{align*}
        This shows the scheme is of order one, hence
        \[|\Phi_{\Dt}(q,p)-\Phi_{\Dt}^{AB}(q,p)|=\mathrm{O}(\Dt^2).\]
        Comparing the two expansions allows us to recover the discrepancy term, showing that the solution of the modified equation at time $\Dt$
        \[\left\{\begin{aligned}\frac{\mathrm d}{\dt} q &= M^{-1}p +\frac{\Dt}2M^{-1}\nabla V(q)\\
        \frac{\mathrm d}{\dt} p &= -\nabla V(q) +\frac{\Dt}2\nabla^2 V(q)M^{-1}p\end{aligned}\right.\]
        agrees with $\Phi_{\Dt}^{AB}(q,p)$ up to order $2$ in $\Dt$. Crucially, this equation is still of Hamiltonian form, for the modified Hamiltonian
        \begin{equation}\label{eq:modified_hamiltonian}\widetilde{H}(q,p)=H(q,p)-\frac{\Dt}2 \nabla V(q)^\intercal M^{-1}p.\end{equation}
    \end{example}
    In fact, it is a general fact that all truncations of the modified dynamics for a symplectic method are of Hamiltonian form.
    For a more thorough discussion of this fact, we refer to \cite[Section IX.4]{H13}, but for now we simply make note of the fact
    that given a numerical method, we can construct a modified Hamiltonian equation for which this numerical order is of arbitrarily high order of local consistency.
    This is important, because, as the following result \cite[Theorem IX.8.1]{H13} shows, the order of the numerical method is directly related to the long-time energy conservation properties along numerical trajectories.
    \begin{theorem}
        Let $H$ be an analytic Hamiltonian, and $\tilde\Phi_\Dt$ a symplectic numerical method of order $\alpha$.
        If there is a compact set $K\subset \mathcal E$ such that $(q^n,p^n)\in K$ for all $n\geq 0$, then there exists $\tau>0$ such that for $\Dt$ small enough,
        \begin{equation} \label{eq:long_time_energy_conservation}H(q^n,p^n)=H(q^0,p^0)+\mathrm{O}(\Dt^\alpha)\end{equation}
        for times $n\Dt\leq e^{\tau/\Dt}$.
    \end{theorem}
    The fact that that the order of the scheme is linked to the local conservation in time of the Hamiltonian is unsurprising.
    The main content of the above result is that this conservation is valid over very long times, provided the numerical trajectory does not explode and that the timestep is chosen to be small enough.
    
    We have already seen that symplectic Euler methods are of order one, and it can straightforwardly be shown that the Verlet scheme is of order 2 by a Taylor expansion.
    Hence we expect the fluctuation of the Hamiltonian to be of order $\Dt$ for the symplectic Euler methods and of order $\Dt^2$ for the Verlet scheme.
    Moreover, by construction, symplectic Euler methods are of order 2 for the first-order modified Hamiltonian dynamics, 
    so we expect the fluctuation of the first-order modified Hamiltonian computed in Example \ref{ex:modified_hamiltonian} to be of order 2 for the symplectic Euler methods. 
    The leading-order correction term for the other Euler symplectic scheme can be computed using the same method, and is given by the opposite of \eqref{eq:modified_hamiltonian}.

    In Figure \ref{fig:hamiltonian_conservation}, we verify this result numerically. 
    A Lennard--Jones system of $1000$ particles was simulated with a temperature $T=1.5$, for a time $\tau=1.0$, and at density $\rho=0.7$.
    The Lennard--Jones potential was cut off using a cubic spline interpolation between $r_a=2.0$ and $r_c=2.5$. 
    We plot the maximal fluctuation of the Hamiltonian
    \[\Delta H \defeq \underset{0 \leq n \leq \lceil \tau/\Dt\rceil}{\max} H(q^n,p^n)-\underset{0 \leq n \leq \lceil \tau/\Dt\rceil}{\min} H(q^n,p^n)\]
    as a function of the timestep $\Dt$ for each of the symplectic splitting schemes. We also plot the maximum fluctuation of the modified Hamiltonian \eqref{eq:modified_hamiltonian} for the symplectic Euler schemes, which we denote in the legend by a lowercase \underline{m}.
    The legend gives the order of the operators in the splitting, along with the slope of the least squares regression line in log-log space, which we superimpose in dotted line.
    In order for the trajectories to be comparable, every simulation was started from the same precomputed equilibrium starting configuration. The results concur with theoretical prediction.
    
    \begin{figure}[htbp]
        \begin{center}
          \includegraphics[width=0.85\linewidth]{figures/chapter1/energy_conservation.pdf}
          \caption{ \label{fig:hamiltonian_conservation}
            Maximal fluctuation of the Hamiltonians and leading-order modified Hamiltonians for the Symplectic Euler and Verlet schemes.
          }
        \end{center}
      \end{figure}

    From a practical point of view, we will always use the Verlet scheme, 
    since it offers the better energy conservation property without any computational overhead compared to the symplectic Euler schemes. 
    Indeed, the force calculation in the last step of \eqref{eq:verlet} can be used again in the first step of the next iteration, so that there is only one force calculation per iteration, as in the symplectic Euler method.
    Finally let us mention that the splitting based on the ordering $ABA$ of the elementary generators yields another symplectic method called the \textit{position} Verlet scheme, which enjoys the same conservation properties as standard (velocity) Verlet.
    It is, however, rarely used in practice.

\section{Canonical averages}
We now turn our attention to methods to compute canonical averages, which are expectations of observables with respect to the canonical measure \eqref{eq:canonical_measure}.
As we already alluded to, the sampling strategy will be based on the definition of a stochastic process under whose evolution the canonical measure is invariant. This is the Langevin dynamics.
We define it, and discuss some of its theoretical properties which are relevant to the sampling of canonical averages.
We then turn to describing a splitting strategy for the discretization of the continuous dynamics, which will serve as our effective sampling tools for canonical averages.
We finally give some numerical illustrations of the strategy.
\subsection{Langevin dynamics}
We first consider the inertial Langevin dynamics, defined by the following stochastic differential equation (SDE), where $\gamma, \beta$ are fixed positive constants:
\begin{equation}
    \label{eq:langevin}
    \left\{\begin{aligned}
        \text dq_t&=M^{-1}p_t\,\dt,\\
        \text dp_t&= -\nabla V(q_t)\,\dt -\gamma M^{-1}p_t\,\dt+\sqrt{\frac{2\gamma}\beta}\text \,dW_t,
    \end{aligned}\right.
\end{equation}
where $(W_t)_{t\geq 0}$ is a standard $dN$-dimensional Brownian motion.
This process is a combination of a Hamiltonian evolution with an additional action on the momenta which, if isolated, defines a $dN$-dimensional Ornstein-Uhlenbeck process.

This additional term be interpreted physically as the combination of two effects: a dissipation term 
$$-\gamma M^{-1}p_t\dt,$$
which can be understood as the effect of a viscous friction force on the particles, and a fluctuation term, 
$$\sqrt{\frac{2\gamma}\beta}\text dW_t,$$
which corresponds to the input of kinetic energy into the system as thermal agitation induced by a surrounding heat bath at temperature $1/(k_B\beta)$.\\

However, the physical meaning can be forgotten thanks to the fact that, \textit{in fine}, we only require that the canonical measure be invariant under this dynamic: as we shall shortly see, this is indeed the case.

\begin{remark}[Generalized Langevin dynamics]
    There are several ways to generalize this process: one is to consider a generic, possibly non-separable, Hamiltonians, as in Remark \ref{rem:non_separable_hamiltonian}, rather than the classical Hamiltonian used above.
    The other is to allow the fluctuation-dissipation term to be parametrized by coefficients $\gamma$ and $\sigma$ depending on the state variable, and which obey a relation ensuring the invariance of $\mu$.
    Hence in full generality, we could consider the following system of SDEs:
    \begin{equation}
        \label{eq:general_langevin}
        \left\{\begin{aligned}
            \text dq_t &=\nabla_p H(q_t,p_t)\dt,\\
            \text dp_t &= -\nabla_q H(q_t,p_t)\dt -\gamma(q_t,p_t)\nabla_pH(q_t,p_t)\dt+\sigma(q_t,p_t)\rm{d}W_t,
        \end{aligned}\right.
    \end{equation}
    where $\gamma$ and $\sigma$ are $dN \times dN$ matrix-valued functions.
    In fact one can also consider the case where $W_t$ is a $r$-dimensional Brownian motion and $\sigma$ is a $dN\times r$ matrix-valued function.
    The Dissipative Particle Dynamics (DPD, see \cite{EW95}) is a generalized Langevin equation of this form, where $\gamma$ and $\sigma$ are position-dependent and the Brownian motion is $dN(N-1)/2$-dimensional, which corresponds to the number of pairs of non-orthogonal momentum degrees of freedom.
\end{remark}

The generator of the Langevin dynamics is the operator
\begin{equation}
  \label{eq:langevin_generator}
\mathcal L_\gamma=M^{-1}p\cdot \nabla_q-\nabla V(q) \cdot \nabla_p- \gamma M^{-1} p \cdot \nabla_p+\frac\gamma\beta \Delta_p,
\end{equation}
and we denote the evolution operator using exponential notation:
\begin{equation}
    \label{eq:langevin_evolution_operator}
    \left(\e^{t\cL_\gamma}\varphi\right)(q,p) \defeq \E^{(q,p)}\left[\varphi(q_t,p_t)\right],
\end{equation}
where the expectation is over all trajectories of the dynamics \eqref{eq:langevin}, starting from $(q_0,p_0)=(q,p)$.

    \subsection{Invariance of the canonical measure}

    Using the generator, one can easily express the evolution of a probability distribution under the Langevin dynamics.
    We assume for simplicity that the solution $(q_t,p_t)_{t\geq 0}$ to \eqref{eq:langevin} has a distribution with a smooth density $\rho_t$ at time $t$ over $\mathcal E$.
    For any $C^\infty$ compactly supported observable $\varphi$, we have
    $$\int_{\mathcal E}\varphi(q,p)\rho_t(q,p)\,\dif q\,\dif p=\int_{\mathcal E}\E^{(q,p)}\left[\varphi(q_t,p_t)\right]\rho_0(q,p)\dif q\dif p=\int_{\mathcal E}\mathrm{e}^{t\cL_\gamma}\varphi(q,p)\rho_0(q,p)\dif q\dif p,$$
    where the superscript is as in \eqref{eq:invariant_measure}. Thus,
    $$\frac{\partial}{\partial t}\left(\int_{\mathcal E}\varphi(q,p)\rho_t(q,p)\,\dif q\,\dif p\right)=\int_{\mathcal E}\mathrm{e}^{t\cL_\gamma}\cL\varphi(q,p)\rho_0(q,p)\,\dif q\,\dif p=\int_{\mathcal E}\cL_\gamma\varphi(q,p)\rho_t(q,p)\,\dif q\,\dif p$$
    If we define $\cL_\gamma^\dagger$ as the adjoint of $\cL_\gamma$ on the flat space $L^2(\mathcal E)$, that is,
    \begin{equation}
        \label{eq:L_dagger}
        \int_{\mathcal E}\left(\cL_\gamma\varphi\right)\psi=\int_{\mathcal E}\varphi \left(\cL_\gamma^\dagger \psi\right)\qquad\text{for all }\phi,\,\psi
    \end{equation}
     compactly supported $C^\infty$ test functions, we have the Fokker--Planck equation,
    \begin{equation}\label{eq:fokker_planck}
        \frac{\partial}{\partial t}\int_{\mathcal E}\varphi(q,p)\rho_t(q,p)\dif q\dif p=\int_{\mathcal E}\varphi(q,p)\cL_\gamma^\dagger\rho_t(q,p)\dif q \dif p,
    \end{equation}
    which rewrites formally as 
    \begin{equation}
        \label{eq:fokker_planck_formal}
        \frac{\partial}{\partial t}\rho_t=\cL_\gamma^\dagger\rho_t.
    \end{equation}
    Using this equation, we can easily show that the canonical distribution is invariant under this dynamics, which is equivalent to the condition 
    $$\cL_\gamma^\dagger \mu=0.$$
    In fact, it is useful to reformulate this condition in the weighted space $L^{2}(\mu)$. Indeed, the stationary Fokker--Planck equation rewrites
    $$\int_{\mathcal E}\cL_\gamma \varphi \,\dif \mu=0\qquad \forall\,\varphi,$$
    or equivalently,
    $$\cL_\gamma^* \1_{\mathcal E}=0,$$
    where $\cL_\gamma^*$ is the adjoint of $\cL_\gamma$ in $L^2(\mu)$ for the scalar product
    $$(\varphi,\psi)\mapsto \int_{\mathcal E}\varphi \psi \,\dif \mu.$$
    This, in turn, follows easily from the following lemma.
    \begin{lemma}
        \label{lemma:star_adjoints_langevin}
        The $L^2(\mu)$ adjoints of the elementary differential operators are given by the formulae
        \begin{equation}
            \label{eq:star_adjoints_langevin}
            \left\{\begin{aligned}
            &\partial_{q_i}^*=-\partial_{q_i}+\beta\partial_{q_i}V,\\
            &\partial_{p_i}^*=-\partial_{p_i}+\beta\left(M^{-1}p\right)_i.
            \end{aligned}\right.
        \end{equation}
    \end{lemma}
    These are easily found by integration by parts. In particular, we find that 
    $$\partial_{q_i}\partial_{p_i}^*-\partial_{p_i}\partial_{q_i}^*=\beta\left((M^{-1}p)_i\partial_{q_i}-\partial_{q_i}V\partial_{p_i}\right),$$
    hence, by summing over $i$,
    \begin{equation}
        \label{eq:L_ham_antisymmetric}
        \cLham=\frac1\beta\left(\nabla_q\cdot\nabla_p^*-\nabla_p\cdot\nabla_q^*\right),
    \end{equation}
    which is an antisymmetric operator. Similarly,
    $$\partial_{p_i}\partial_{p_i}^*=\beta(M^{-1}p)_i\partial_{p_i}-\partial_{p_i}^2,$$
    hence
    \begin{equation}
        \label{eq:C_symmetric}
        C=-\frac{1}\beta\nabla_p\cdot\nabla_p^*,
    \end{equation}
    which is a symmetric operator. In summary, we have that 
    \begin{equation}
        \label{eq:L_star}
        \cL_\gamma^*=-\cLham+\gamma C=-(A+B)+\gamma C.
    \end{equation}
    It follows immediately that $\cL_\gamma^*\1_{\mathcal E}=0$. Notice that since $\cLham^*\1_{\mathcal E}=0$, the canonical measure is also invariant under the Hamiltonian dynamics. However, because the latter is restricted to a manifold with zero measure with respect to $\mu$, Hamiltonian ergodic averages cannot in general converge to the correct value.
    \begin{remark}[Fluctuation-dissipation relation for generalized Langevin dynamics]
        We come back to the general Langevin dynamics \eqref{eq:general_langevin}.
        In this case the generator is given by 
        \[\cL_{\gamma,\sigma}=\cLham-(\gamma \nabla_p H)\cdot \nabla_p +\frac12(\sigma \sigma^\intercal) : \nabla_p^2.\]
        The canonical measure is invariant under the action of the Hamiltonian part, so having
        \[\widetilde{\cL}_{\mathrm{FD}}\defeq -(\gamma \nabla_p H)\cdot \nabla_p +\frac12(\sigma \sigma^\intercal) : \nabla_p^2 \]
        such that  $\widetilde{\cL}_{\mathrm{FD}}^*\1_\mathcal{E}=0$ is enough to guarantee the invariance of the canonical measure.
        If $\gamma$ and $\sigma$ are momentum-dependent, then this condition is a complicated differential-in-$p$ relation between the coefficients of $\gamma$ and $\sigma$, which can be explicitly computed by integration by parts in the expression
        \[\int_{\mathcal E} (\cL_{\mathrm{FD}}\varphi)\psi\e^{-\beta H},\]
        with $\varphi, \psi \in C_c^\infty(\mathcal E)$ test functions. 
        In the case where $\gamma$ and $\sigma$ are only position-dependent however, as is often the case in practice, then the expression simplifies greatly, and becomes simply an algebraic relationship between $\gamma$ and $\sigma$, namely
        \begin{equation}
            \label{eq:general_fd_relation}
            \sigma \sigma^\intercal =\frac{2\gamma}{\beta}.
        \end{equation}
    \end{remark}

    \subsection{Overdamped limit of Langevin dynamics}
        As already pointed out, the fact that the kinetic marginal of $\mu$ is a Gaussian distribution makes sampling canonical momenta trivial. 
        Instead, the main problem is sampling from $\nu$. It follows directly from the invariance of $\mu$ under trajectories of the Langevin dynamics that $\nu$ is invariant under the configurational trajectories of the Langevin dynamics.
        It would be convenient, however, to have at our disposal a dynamics on $\mathcal D$ which has $\nu$ as an invariant measure.
        It turns out this is possible, by observing that the invariance of $\mu$ is independent of the parameter $\gamma$, and taking the limit $\gamma\to\infty$. This requires a bit of care.
        Notice the SDE on the momenta in \eqref{eq:langevin} rewrites 
        \[\dif p_t=-\nabla V(q_t)\dif t-\gamma \dif q_t+\sqrt{\frac{2\gamma}\beta}\dif W_t,\]
        thus a time integration gives
        \[q_t-q_0=\frac{p_0-p_t}\gamma -\frac{1}\gamma\int_0^t\nabla V(q_s)\dif s +\sqrt{\frac{2}{\gamma\beta}}W_t.\]
        The scaling invariance of the Brownian motion $(\sqrt{\alpha}W_{t/\alpha^2})_{t\geq 0}\sim (W_t)_{t\geq 0}$ suggests considering the timescale $\gamma\beta t$, thus
        \[q_{\gamma\beta t}-q_0=\frac{p_0-p_{\gamma\beta t}}\gamma-\frac1{\gamma}\int_0^{\gamma\beta t}\nabla V(q_s)\dif s +\sqrt{2}\widetilde{W_t},\]
        where $\widetilde W$ is again a Brownian motion. Using the change of variables $s=\gamma \beta u$ in the integral term yields
        \begin{equation}\label{eq:rescaled_langevin}q_{\gamma\beta t}-q_0=\frac{p_0-p_{\gamma\beta t}}\gamma-\beta\int_0^{t}\nabla V(q_{\gamma\beta u})\dif u +\sqrt{2}\widetilde{W_t},\end{equation}
        At this point, we formally take $\gamma\to\infty$, which suggests the following SDE for the rescaled in time process,
        \begin{equation}
            \label{eq:overdamped_langevin}
            \dif q_t=-\beta \nabla V(q_t)\,\dif t+\sqrt{2}\,\dif W_t.
        \end{equation}
        
        This equation defines the overdamped Langevin, or Brownian, dynamics.
        To justify the limit in a rigorous manner, one would hope to show that the rescaled process \eqref{eq:rescaled_langevin} converges in law to a weak solution of the SDE \eqref{eq:overdamped_langevin}, in some functional space.
        However, this is technical overkill, since, we only need to consider dynamics as sampling devices. In fact, the physical interpretation of this equation is not entirely clear in terms of the dimensions of the quantities involved.
        We can just as well take equation \eqref{eq:overdamped_langevin} as given, and be satisfied by the following fact.
    
        \begin{prop}
            The configurational Gibbs measure $\nu$ is invariant under the dynamics \eqref{eq:overdamped_langevin}.
        \end{prop}
    
        This follows along the same lines as for the Langevin dynamics. The generator (now acting on observables defined on $\mathcal D$) is the operator
        \begin{equation}
            \label{eq:overdamped_langevin_generator}
            \cL\varphi=-\beta\nabla V\cdot \nabla\varphi + \Delta \varphi.
        \end{equation}
        Again, we consider the weighted space $L^2(\nu)$. Adjoints of elementary differential operators are still given by the first line of \eqref{eq:star_adjoints_langevin}, and it is then easily seen that
        \begin{equation}
            \label{eq:L_overdamped_symmetric}
            \cL=-\nabla^*\cdot\nabla
        \end{equation}
        is a symmetric operator. Again we have $\cL^*\1_{\mathcal D}=0$, so $\nu$ satisfies the stationary Fokker--Planck equation under this dynamics.
    
        \begin{remark}
            Instead of rescaling time by $\beta\gamma$, we could have rescaled by $\gamma$, which would yield the dynamics
            \begin{equation}
                \label{eq:overdamped_langevin_alt}
                \dif q_t=-\nabla V(q_t)\dif t+\sqrt{\frac 2\beta}\dif W_t.
            \end{equation}
            Which formulation to choose is a matter of preference, since both yield a dynamics invariant under $\nu$, as seen from the identity (where we still write $\cL$ for the generator)
            \[\cL=-\frac1\beta\nabla^*\cdot\nabla.\]
        \end{remark}
        We end this chapter by stating some key properties of the continuous dynamics, regarding ergodicity and and statistical error, before moving to the description of splitting schemes. 

        \subsection{Convergence to equilibrium}
        Once we have shown that the Langevin dynamics and its overdamped limit are reasonable candidates to sample from the canonical measure and its configurational marginal,
        we should ensure that the target measure is actually sampled, and not simply invariant for the dynamics, as the cautionary example of the Hamiltonian dynamics shows.
        This is the question of ergodicity, and resuts of this nature mainly come in two flavors.
        \begin{enumerate}[(i)]
            \item Probabilistic ergodic theorems that ask about the convergence of random variables like \eqref{eq:ergodic_averages}, that give analogs of the Law of Large Numbers for the correlated process $\varphi(q_t,p_t)$.
            \item More analytic results which express the convergence of the law of the process at time $t$ towards a stationary solution to the Fokker--Planck equation. These can usually be expressed as decay estimates on the evolution Semigroup \eqref{eq:langevin_evolution_operator}, in a judiciously chosen functional setting.
        \end{enumerate}
        We will not get into details as these questions can quickly get quite technical, but rather give intuitive ideas of the setting, and point the reader to more thorough sources.
        Results such as (i) typically leverage the strong Law of Large Numbers by dissecting the trajectory into a discrete number of (loosely) \iid excursions through phase space, guaranteeing the almost sure convergence of trajectory averages.
        As such, proving the positive recurrence of the dynamics is crucial to showing that these excursions are well-behaved, while also implying that the invariant measure is unique.
        One idea, exploited by Kliemann in \cite{K87}, is to recast the Langevin dynamics as a control equation (where $W_t$ acts as the control). 
        He is thus able to leverage criteria on $\cL_\gamma$ from geometric control theory to ensure the almost sure convergence of trajectory averages.
        Results such as (ii) depend on the functional setting. Let us simply state the following result, based on hypocoercive estimates. 
        For an introduction to these ideas, and references, we point to Section 2.1.1 in \cite{LMS13}, and \cite{DMS15}.
        For any measure $\rho$ denote by $L^2_0(\rho)$ the space of square-integrable observables with respect to $\rho$ and zero mean.
        \begin{prop}[Exponential decay rate of the Semigroup]
            Assume that the potential $V$ is smooth and that the configurational marginal $\nu$ of $\mu$ satisfies a Poincaré inequality:
            there exists a constant $R>0$ such that for all $\varphi \in L^2_0(\nu)$ such that $\nabla \varphi \in L^2(\nu)^{dN}$,
            \begin{equation}
                \label{eq:poincare_inequality}
                \|\varphi \|_{L^2(\nu)}^2\leq \frac1R \| \nabla \varphi\|^2_{L^2(\nu)}.
            \end{equation}
            Then there exist constants $C_\gamma,\lambda_\gamma >0$ such that 
            \begin{equation}
                \label{eq:semigroup_exp_decay}
                \|\e^{t\cL_\gamma}\|_{\mathcal{B}(L_0^2(\mu))}\leq C_\gamma \e^{-t\lambda_\gamma}.
            \end{equation}
        \end{prop}
        For the overdamped case, exponential decay rates can also be obtained (much more directly) under the same conditions.
        Moreover the Poincaré inequality \eqref{eq:poincare_inequality} is automatically satisfied in the case of a compact configurational space $\mathcal D$.
        
        \subsection{Asymptotic variance for ergodic averages}\label{subsec:asymptotic_variance_cont}
        Since ergodic averages are computed over a finite time-interval, the corresponding random variable will have some variance. 
        Let us show how to relate this variance to the dynamics, provided a Central Limit Theorem holds.
        For simplicity, we assume that $(q_0,p_0)\sim \mu$, and let $\varphi \in H^1(\mu)$ be an observable of interest. We also denote by
        \begin{equation}
            \label{eq:equilibrium_projector}
            \Pi \varphi = \varphi -\int_\mathcal{E}\varphi\,\dif \mu
        \end{equation}
        the centering projector associated with $\mu$. The Central Limit Theorem asserts that the following convergence in law holds:
        \begin{equation}
            \label{eq:clt_diffusion}
            \frac1{\sqrt{T}}\int_0^T\Pi\varphi(q_t,p_t)\,\dif t \overset{\mathrm{law}}{\implies} \mathcal N(0,\sigma_{\varphi}^2),
        \end{equation}
        where $\sigma_{\varphi}^2$ is the asymptotic variance associated to $\varphi$ under the dynamics, which is thus given by the limit of the variance on the left-hand side. Let us compute:
        \[\sigma_{\varphi,T}^2\defeq \E_\mu\left[\left(\frac1{\sqrt{T}}\int_0^T\Pi\varphi(q_t,p_t)\,\dif t\right)^2\right]=\frac1T\int_0^T\int_0^T\E_\mu\left[\Pi\varphi(q_t,p_t)\Pi\varphi(q_s,p_s)\right]\,\dif s\,\dif t.\]
        By stationarity, for $t>s$, $\Pi\varphi(q_t,p_t)\Pi\varphi(q_s,p_s)\sim \Pi\varphi(q_{t-s},p_{t-s})\Pi\varphi(q_0,p_0)$, hence we may write, by Fubini's theorem, 
        \begin{align*}\sigma_T &= \frac2T\int_0^T\int_0^t\E_\mu\left[\Pi\varphi(q_{t-s},p_{t-s})\Pi\varphi(q_0,p_0)\right]\,\dif s\,\dif t\\
             &= \frac2T\int_0^T\int_s^T\E_\mu\left[\Pi\varphi(q_{s},p_{s})\Pi\varphi(q_0,p_0)\right]\,\dif t\,\dif s\\
             &=2\int_0^T\E_\mu\left[\Pi\varphi(q_{s},p_{s})\Pi\varphi(q_0,p_0)\right]\left(1-\frac{s}{T}\right)\,\dif s\\
             &=2\int_0^T\E_\mu\left[\left(e^{s\cL_\gamma}\Pi\varphi\right)\left(\Pi\varphi\right)\right]\left(1-\frac{s}{T}\right)\,\dif s
        \end{align*}
        Using a Cauchy--Schwarz inequality in $L^2(\mu)$ and an exponential decay estimate on the evolution semigroup like \eqref{eq:semigroup_exp_decay}, we obtain a bound of the form
        \[\int_0^T\left|\E_\mu\left[\left(e^{s\cL_\gamma}\Pi\varphi\right)\left(\Pi\varphi\right)\right]\frac{s}{T}\right|\,\dif s\leq \int_0^\infty C_\gamma\e^{-\lambda_\gamma s}\left\|\Pi\varphi\right\|^2_{L^2(\mu)}\frac{s}{T}\,\dif s\]
        for some positive constants $C$ and $\alpha$, which converges uniformly to $0$ as $T\to \infty$. It follows that 
        \begin{equation}
            \label{eq:asymptotic_variance_with_semigroup}
            \sigma_{\varphi}^2=\int_0^\infty \E_\mu\left[\left(e^{s\cL_\gamma}\Pi\varphi\right)\left(\Pi\varphi\right)\right]\,\dif s.
        \end{equation}
        We use the following equality of bounded operators on $\mathcal{H}^1(\mu)$, which is the operator:
        \begin{equation}
            \label{eq:resolvent_langevin}
            (-\cL_\gamma)^{-1}=\int_0^\infty \e^{s\cL_\gamma}\,\dif s,
        \end{equation}
        which again is justified by the exponential decay rate of evolution semigroup, and where the integral on the right is in the Bochner sense, that is a generalization of the Lebesgue integral to functions taking values in a general Banach space.
        Using this identity, we can write the asymptotic variance more concisely,
        \begin{equation}
            \label{eq:asymptotic_variance_with_Lm1}
            \sigma_{\varphi}^2=\E_{\mu}\left[(\Pi\varphi) (-\cL_\gamma)^{-1}(\Pi \varphi)\right].
        \end{equation}
        Note that the exact same computations can be performed for the overdamped Langevin dynamics. 
        It follows from \cite[Theorem 2.1]{B82}, that a sufficient condition for a CLT to hold is that $\Pi \varphi\in \operatorname{Ran}\mathcal L_\gamma$, which is the case provided we can express the inverse of $\cL_\gamma$ using \eqref{eq:resolvent_langevin}.
\subsection{Splitting schemes for the Langevin dynamics}\label{section:splitting_schemes_langevin}

Similarly to the Hamiltonian case, the generator \eqref{eq:langevin_generator} splits into three elementary generators, namely 
$$\mathcal L_\gamma= A+B+\gamma C=\cLham +\gamma C,$$
with
\begin{equation}
  \label{eq:C_definition}
C=-M^{-1}p\cdot \nabla_p +\frac1\beta \Delta_p.
\end{equation}
These generators individually give rise to dynamics which we can analytically integrate, defined by the following evolution operators:

\begin{equation}
  \label{eq:propagators}
  \left\{\begin{aligned}
    &\mathrm{e}^{tA}\varphi(q,p)=\varphi(q+tM^{-1}p,p),\\
    &\mathrm{e}^{tB}\varphi(q,p)=\varphi(q,p-t\nabla V(q)),\\
   &\mathrm{e}^{t\gamma C}\varphi(q,p)=\mathbbm E \left[\varphi \left(q, \mathrm e^{-\gamma M^{-1}t}p + \sqrt{\frac{M}{\beta}(1-\mathrm{e}^{-2\gamma M^{-1}t})}G\right)\right],
\end{aligned}\right.
\end{equation}
where $G$ is a standard $dN$-dimensional Gaussian. The third equality is a reformulation of an equality in law between an Itô integral and a Gaussian random variable, and follows by applying Itô's formula to the rescaled process
$$\e^{\gamma M^{-1}t}p_t,$$
where $p_t$ is the Ornstein--Uhlenbeck process:
\begin{equation}
    \dif p_t=-\gamma M^{-1}p_t\,\dif t+\sqrt{\frac{2\gamma}{\beta}}\,\dif W_t,
\end{equation}
then applying the following matrix form of Itô's isometric property 
\begin{equation}
    \label{eq:ito_isometry_matrix}
    \int_0^t A_s\,\dif W_s \overset{\mathrm{law}}{=} \left(\int_0^t A_s A_s^\intercal \,\dif s\right)^{\frac12}\mathcal G
\end{equation}
to obtain the equality in law.
 The dynamics associated with the $A$ and $B$ parts are deterministic Hamiltonian dynamics already identified in \eqref{eq:Lham_splitting_dynamics_solved}.
Just as in the Hamiltonian case, we can define schemes for the Langevin dynamics based on approximating the evolution operator \eqref{eq:langevin_evolution_operator} over one timestep by splitting the generator $\cL_\gamma$, and combining the corresponding evolution operators \eqref{eq:propagators} in a sequence.
We refer to such a splitting approximation by the sequence in which the individual propagators are composed. It is useful at this point to introduce the stochastic flow map associated with the Ornstein--Uhlenbeck dynamics.
\begin{equation}
    \label{eq:stochastic_flow_c}
    \Phi_t^C(q,p,\xi)=\left(q, \mathrm e^{-\gamma M^{-1}t}p + \sqrt{\frac{M}{\beta}(1-\mathrm{e}^{-2\gamma M^{-1}t})}\xi\right),
\end{equation}
where $\xi\in \R^{dN}$. The map is defined so that $\E[\varphi\left(\Phi_t^C(q,p,G)\right)]=\mathrm{e}^{t\gamma C}\varphi(q,p)$ when $G$ is a standard Gaussian random variable.
Given an ordering of operators, 
\begin{equation}\label{eq:splitting_ordering}(R_1,\dots,R_k)\in \{A,B,\gamma C\}^k,\end{equation}
we can consider the mapping obtained by composing the elementary flows, noted as
\begin{equation}\label{eq:markov_mapping}\Phi^{R_1,\dots R_k}\defeq\Phi_{\Delta t/n_{R_k}}^{R_k}\circ\dots\circ\Phi_{\Delta t/n_{R_1}}^{R_1},
\end{equation}
where
$$n_R\defeq \#\left\{1\leq j\leq n \middle| R_j=R\right\}$$
for $R\in\{A,B,\gamma C\}$, and which we may consider by a slight abuse to be the mapping which takes a point in phase space and $n_{\gamma C}$ vectors in $\xi_1,\dots,\xi_{n_{\gamma C}}\in\R^{dN}$, yielding a point in phase space by successively applying the flow, and, if need be, stochastic flow, mappings corresponding to the reverse ordering of \eqref{eq:splitting_ordering}.
Applying this mapping with a vector of independent standard Gaussians yields a stochastic mapping, which defines the update rule for the splitting scheme associated with the ordering \eqref{eq:splitting_ordering}.
An important property which follows from writing the update rule using the mapping \eqref{eq:markov_mapping} is that numerical trajectories formed by iterating this update rule with independent vectors of standard Gaussians form a Markov chain.
The hope is that the invariant measure corresponding to this Markov chain (provided it is unique) is a close approximation to the canonical measure, as well as being ergodic.
It is less cumbersome to write such schemes at the level of the evolution operator associated with the Markov chain,

\begin{equation}
    \label{eq:markov_evolution_operator}
    P_\Dt\varphi (q,p) = \E\left[\varphi(q^1,p^1)\middle|q^0=q,\,p^0=p\right].
\end{equation}
The evolution operator associated with the splitting \eqref{eq:splitting_ordering} is
\begin{equation}
    \label{eq:splitting_semigroup_expr}
    P_\Dt=\e^{\Dt/n_{R_1}R_1}\dotsm\e^{\Dt/n_{R_k}R_k}.
\end{equation}
Note that the order is reversed compared to the stochastic flow formulation.
While the formulation at the stochastic flow level is slightly complicated to write, it corresponds very closely to how a general splitting integrator for the Langevin dynamics can be implemented.
One simply has to compute the timestep corresponding to each operator in the ordering, and apply in succesive steps the corresponding flow maps, sampling independent Gaussian variables for each stochastic step.
We will refer to such schemes by the name obtained by concatenating the names of each operator appearing in the ordering, using $O$ instead of $\gamma C$.
For instance, the BAOAB scheme corresponds to the case $k=5$, $n_A=n_B=2$, and $n_{\gamma C}=1$, and the ordering $(B,A,\gamma C,A,B)$.
    \begin{example}[BAOAB scheme]
        The update rule is given by the following equations, with additional intermediate coordinate and momentum variables:
        \begin{equation}\label{baoab}
            \left\{\begin{aligned}
                 p^{n+\frac13} &=p^n -\frac{\Delta t}{2}\nabla V(q^n)\\
                 q^{n+\frac12} &=q^n + \frac{\Delta t}{2} M^{-1}p^{n+\frac 13}\\
                 p^{n+\frac23} &=\alpha_{\Delta t}p^{n+\frac13}+\sigma_{\Delta t}G^n\\
                 q^{n+1} &=q^{n+\frac12} + \frac{\Delta t}{2} M^{-1}p^{n+\frac 23}\\
                 p^{n+1} &= p^{n+\frac23}-\frac{\Delta t}{2}\nabla V(q^{n+1}),
            \end{aligned}\right.
        \end{equation}
        where, again, $G^n$ is a standard $dN$-dimensional Gaussian.
    \end{example}
     Now we have a recipe to make an infinite number of numerical schemes, which can easily be implemented in a computer.
     We could even go further and consider methods with an uneven distribution for the secondary timesteps, the introduction of negative secondary timesteps for the $A$ and $B$ steps, and so on.
     This room for creativity highlights the need for criteria to assess the quality of such schemes. A tradeoff between various considerations has to be found:

     \begin{enumerate}[(i)]
         \item Our aim is to compute long trajectories, which are needed to ensure that the phase space is properly explored, as well as to obtain better statistical properties for averages \eqref{eq:discrete_ergodic_averages}. Thus, for a fixed computational budget, we desire a scheme which allows us to take as large a timestep $\Delta t$ as possible. This is the issue of numerical stability.
         \item The use of a positive timestep $\Delta t$ implies in general that the invariant measure for the Markov chain corresponding to a given scheme is not the canonical measure. This issue is called systematic error, or bias, and one would desire a scheme which minimizes this bias.
         \item The main computational cost in computing iterates of these numerical schemes is the evaluation of the gradient of the potential used for the $B$ steps. As such, it is desirable to have a scheme which requires as few evaluations of this gradient as possible per iteration. Some care must be taken when implementing these, to ensure that already computed gradients are not re-computed: for instance, the gradient in the last step of the BOAB scheme, is equal to the one in the first step of the next iteration.
         \item Notice that the parameter $\gamma$ is free for the practitioner to choose. A natural question is to determine the properties of the marginal dynamics in $q$ in the limit $\gamma \to +\infty$, and in particular if we obtain a consistent discretization of the overdamped Langevin dynamics. Conversely, one could ask about properties of the dynamics as we take the Hamiltonian limit $\gamma\to 0$.
     \end{enumerate}

     \begin{remark}[Overdamped and Hamiltonian limits in splitting schemes]
        Concern (iv) is simple to address. Since
        \[\underset{\gamma \to 0}{\lim}\,\alpha_\Dt=1\,\qquad \underset{\gamma \to +\infty}{\lim}\alpha_\Dt=0,\]
        every $O$-step can simply be dropped as $\gamma\to 0$ from the sequence of operators defining the splitting. 
        This yields a symplectic scheme for the Hamiltonian parts of the dynamics, whose properties can be analyzed as before.
        For this reason, the ordering of the Hamiltonian parts of the splitting of most commonly used schemes corresponds to that of a velocity Verlet method.
         As $\gamma \to \infty$, 
        every $O$-step reduces to resampling the momentum according to the Maxwell--Boltzmann distribution. 
        The properties of the resulting scheme depend on the particular ordering of the splitting at hand. 
        For instance, if every $A$ step is preceded by an $O$ step, then the potential is entirely ignored by the evolution, and the trajectories form an isotropic random walk. This is for instance the case of the BOA scheme.
        On the other hand, it may happen that one obtains a discretization of the overdamped Langevin dynamics. For example, the update equation for the overdamped limit of the BAOA scheme in the case $M=\Id$ rewrites
        \[q^{n+1}=q^n -\frac{\Dt^2}2\nabla V(q^n)+\frac12\sqrt{\frac{\Dt^2}\beta}\left(G^n+G^{n+1}\right),\]
        with $(G^n)$ an \iid standard Gaussian sequence. Because of the two-step correlations in the Gaussian increments, the numerical trajectories are not Markovian as such.
        Nevertheless the scheme is reminiscent of a discretization of the overdamped equation \eqref{eq:overdamped_langevin_alt}, but with an effective timestep $\frac{\Dt^2}2$. 
        This quadratic rescaling of the timestep is common, and has to be related to the fact that the overdamped equation is defined through a diffusive rescaling.
        This discretization also corresponds to the overdamped limit of the BAOAB scheme, which is unsurprising in view of the results of the next chapter.
     \end{remark}

    \subsection{Error analysis for splitting schemes}
        We turn our attention to analyzing the error arising from the estimation of canonical expectations by discrete ergodic averages. We consider estimations of the form 
        \[\widehat{\varphi}_{N_{\mathrm{iter}}} \defeq \frac1{N_{\mathrm{iter}}}\sum_{k=0}^{N_{\mathrm{iter}}-1}\varphi(q^k,p^k),\]
        where $(q^n,p^n)_{n\geq 0}$ is a numerical trajectory of a Markov chain which is ergodic with respect to a unique stationary distribution $\mu_{\Dt}$ on $\mathcal E$, which approximates $\mu$.
        To show the existence of such an invariant distribution, one can rely on general results from the theory of Markov chains, combined with Lyapunov estimates. For a precise result, we refer the reader to \cite[Proposition 2.9]{LMS13}, for example.
        For simplicity, we will assume that $(q^0,p^0)$ is distributed according to $\pi_\Dt$, so that the whole numerical trajectory is stationary. This requires in practice that the system be equilibriated before starting to sample the observables of interest. We decompose the error as follows:
        \begin{equation}\label{eq:error_decomposition}\widehat{\varphi}_{N_{\mathrm{iter}}}-\int_\mathcal{E}\varphi \,\dif \mu= \widehat{\varphi}_{N_{\mathrm{iter}}} - \int_\mathcal{E}\varphi \,\dif \mu_{\Dt}+\int_\mathcal{E}\varphi\,\dif \mu_{\Dt}-\int_{\mathcal E}\varphi \,\dif \mu.\end{equation}
        Two terms contribute to the error.
        \subsubsection{Statistical Error}
        The first term
        \[\widehat{\varphi}_{N_{\mathrm{iter}}} - \int_\mathcal{E}\varphi \,\dif \mu_{\Dt}\]
        is the statistical error due to the truncation of computed trajectories to a finite time $T_{\mathrm{sim}}\defeq N_{\mathrm{iter}}\Dt$. 
        By the Central Limit Theorem for Markov Chains, asymptotically, the statistical error is of order
        $\sigma_{\varphi,\Dt}/\sqrt{N_{\mathrm{iter}}}$,
        where $\sigma_{\varphi,\Dt}$ is the asymptotic variance of the Markov chain, which is given by the following expression, under suitable assumptions, by
        \[\sigma_{\varphi,\Dt}^2=\mathrm{Var}_{\pi_\Dt}(\varphi)+2\sum_{k=1}^\infty \mathrm{Cov}_{\pi_\Dt}(\varphi(q^0,p^0)\varphi(q^k,p^k)).\]
        The proof of the above expression is in spirit the same as the one in the continuous case, exploiting the stationarity in law of the trajectory.
        Using the evolution and projection operators associated with the Markov chain, which are respectively defined by
        \begin{equation}
            \label{eq:markov_chain_operators}
            P_{\Dt}\varphi(q,p)=\E[\varphi(q^1,p^1)|(q^0,p^0)=(q,p)],\qquad \Pi_\Dt \varphi=\varphi - \int_{\mathcal E}\varphi \,\dif \mu_{\Dt},
        \end{equation}
        we can rewrite the asymptotic variance in a more analytic form, namely
        \begin{equation}
            \label{eq:asymptotic_variance_markov_chain}
            \sigma_{\varphi,\Dt}^2=2\int_{\mathcal E}(\Pi_{\Dt}\varphi)(\Id-P_\Dt)^{-1}(\Pi_\Dt \varphi)\,\dif\mu_{\Dt}-\int_{\mathcal E}(\Pi_\Dt \varphi)^2\,\dif \pi_\Dt,
        \end{equation}
        where we formally used the Neumann series
        \[\sum_{k=0}^\infty P_\Dt=(\Id-P_\Dt)^{-1},\]
        which has to justified rigorously, for instance using the geometric decay estimate for $P_\Dt^n$ from \cite[Proposition 2.9]{LMS13}.
        This implies that 
        \[\Dt\sigma_{\varphi,\Dt}^2=2\int_{\mathcal E}(\Pi_{\Dt}\varphi)\left(\frac{\Id-P_\Dt}{\Dt}\right)^{-1}(\Pi_\Dt \varphi)\,\dif\mu_{\Dt}+\mathrm{O}(\Dt).\]
        For reasonable discretizations of the Langevin dynamics, we expect that 
        \[\frac{\Id-P_\Dt}{\Dt}=-\cL_\gamma+\mathrm{O}(\Dt),\]
        which suggests that at dominant order in $\Dt$ approaching $0$, $\Dt\sigma_{\varphi,\Dt}^2$ behaves like
        \[2\int_{\mathcal E}(\Pi\varphi)\left(-\cL_\gamma\right)^{-1}(\Pi\varphi)\,\dif\mu = \sigma^2_\varphi,\]
        where we recall the asymptotic variance for the continuous dynamics \eqref{eq:asymptotic_variance_with_Lm1}
        Hence the statistical error can be estimated, for $\Dt$ small enough, by 
        \[\frac{\sigma_{\varphi,\Dt}}{\sqrt{N_{\mathrm{iter}}}}=\frac{\sigma_{\varphi,\Dt}\sqrt{\Dt}}{\sqrt{T_{\mathrm{sim}}}}\approx \frac{\sigma_\varphi}{\sqrt{T_{\mathrm{sim}}}}.\]
        Loosely speaking, the statistical error for the discrete ergodic estimator is governed at dominant order by the corresponding asymptotic variance for the underlying continous dynamics, as well as the \textit{physical} time of the simulation.
        This confirms that to minimize statistical error, and given a fixed budget of simulation steps, one should maximize $T_{\mathrm{sim}}$ and thus $\Dt$, as discussed in consideration (i).
        However, increasing the timestep comes at the following cost.

        \subsubsection{Systematic Error}
        The second term in \eqref{eq:error_decomposition}, namely
         \[\int_\mathcal{E}\varphi\,\dif \mu_{\Dt}-\int_{\mathcal E}\varphi \,\dif \mu,\]
        is independent of the simulation time, and expresses the fact, highlighted in consideration (ii), 
        that the invariant measure $\pi_\Dt$ for the discrete evolution will in general be different from $\mu$, in the sense that the average of an observable $\varphi$ under $\pi_\Dt$ can be expressed by an expansion of the form
        \begin{equation}
            \label{eq:expansion_of_avg_in_dt}
            \int_{\mathcal E}\varphi(q,p)\,\pi_\Dt(\dif q,\dif p)=\int_{\mathcal E}\varphi(q,p)\,\mu(\dif q,\dif p)+\Dt^{\alpha}\int_{\mathcal E}\varphi(q,p)f_{\alpha}(q,p)\,\mu(\dif q,\dif p)+\mathrm{O}(\Dt^{\alpha+1}),
        \end{equation}
        where $\alpha>0$ and $f_{\alpha}$ is the dominant correction term, which can be explicitly written down for splitting schemes, although computing them requires solving a high-dimensional partial differential equation.
        We postpone further discussion of these types of weak error estimates to the next chapter, in which we analyze the systematic error in the BAOA scheme.

    \subsection{Unbiased sampling}
     It turns out that one can devise schemes which have no systematic error: the Markov chain generating the trajectories has invariant measure exactly $\mu$.
     These methods are based on the Metropolis--Hastings algorithm, which gives a general method to sample a given target distribution.
    \subsubsection*{The Metropolis--Hastings algorithm}
     We aim to sample from a given target measure $\pi$ on some measurable space $\mathcal X$. We suppose we have at our disposal a way to generate proposal points from a given point $\in \mathcal X$. This amounts to defining a transition kernel, the \textit{proposal}, which we may take to be a map
     \[T:\mathcal X\times\mathcal X\longrightarrow\R_+,\]
     such that for any $x\in\R^d$, $T(x,\cdot)$ is a probability density on $\mathcal X$, and which is usually inexpensive to sample from (very often, if $\mathcal X=\R^d$, these are taken to be some form of Gaussian distribution). We also assume for simplicity that we always have $T(x,y)>0$. (This is always the case if the kernel is Gaussian).
     We also fix a function $r:\R_+\to(0,1]$, the acceptance rule, which satisfies the property
     \begin{equation}\label{eq:mh_rule}x\cdot r\left(\frac1x\right)=r(x)\end{equation}
     We then define a Markov chain by iterating the following algorithm, starting from an arbitrary point $q^0\in \mathcal X$.

     \begin{algorithm}[Metropolis--Hastings]
        Starting from an arbitrary point $q^0$, iterate on $n\geq 0$:
        \begin{enumerate}[(1)]
            \item Sample a proposal $\tilde q^{n+1}$ according to the probability law $T(q^n,\cdot)$.
            \item Compute\[R(\tilde q^{n+1},q^n)=r\left(\frac{\pi(\tilde q^{n+1})T(\tilde q^{n+1},q^n)}{\pi(q^n)T(q^n,\tilde q^{n+1})}\right).\]
            \item With probability $R(\tilde q^{n+1},q^n)$, set $q^{n+1}=\tilde q^{n+1}$, otherwise, set $q^{n+1}=q^n$. Concretely, this is done by sampling a uniform variable $U^n\sim \mathcal U([0,1))$, and setting $q^{n+1}=\tilde q^{n+1}$ if $U^n \leq R(\tilde q^{n+1},q^n)$, and $q^{n+1}=q^n$ otherwise.
            \item Go back to step (1) with $q^n\leftarrow q^{n+1}$.
        \end{enumerate}
     \end{algorithm}

    Since $T$ defines a Markov chain, we may always write $\tilde q^{n+1}=\Phi(q^n,\xi^n)$ for some family of \iid variables $(\xi^n)_{n\geq 0}$. We can then write $q^{n+1}$ in a concise form:
    \begin{equation}
        \label{eq:metropolis_hastings_algorithm}
        q^{n+1}=\Phi(q^n,\xi^n)+\1_{U^n>R\left(\Phi(q^n,\xi^n),q^n\right)}\left(q^n-\Phi(q^n,\xi^n)\right)=\Psi(q^n,\xi^n,U^n),
    \end{equation}
    where the $U^n$ are \iid uniform on $[0,1]$, such that the $(\xi^n,U^n)$ are an independent family. This expression is slightly formal since in general addition is not defined on $\mathcal X$, nevertheless, it shows that the algorithm defines a Markov chain. Moreover,
    \begin{equation}
        \begin{aligned}
            \pi(x)\mathbb P\left(q^1=y \middle| q^0=x\right)&=\pi(x)T(x,y)R(x,y)\\
            &=\pi(x)T(x,y)r\left(\frac{\pi(y)T(y,x)}{\pi(x)T(x,y)}\right)\\
            &=\pi(y)T(y,x)\frac{\pi(x)T(x,y)}{\pi(y)T(y,x)}r\left(\frac{\pi(y)T(y,x)}{\pi(x)T(x,y)}\right)\\
            &=\pi(y)T(y,x)r\left(\frac{\pi(x)T(x,y)}{\pi(y)T(y,x)}\right)\qquad\text{(Using \eqref{eq:mh_rule})}\\
            &=\pi(y)\mathbb P\left(q^1=x\middle|q^0=y\right).
        \end{aligned}
    \end{equation}
    Thus, the chain is reversible with respect to $\pi$, so that it is an invariant measure.
    Note that the algorithm is applicable even when we do not know how to evaluate $\pi$, but only the ratios $\pi(x)/\pi(y)$, which is in particular the case for Gibbs measures, which are known up to a normalization constant.

    The historical and most commonly used acceptance rule is the Metropolis rule:
    \begin{equation}
        \label{eq:metropolis_rule}
        r(x)=\min\left\{1,x\right\}.
    \end{equation}

    \begin{remark}[Alternative acceptance rules for Metropolis--Hastings]
        Other possible choices for $r$ are:
        \begin{enumerate}
            \item The Barker rule, \[r(x)=\frac{x}{1+x},\]
            \item Any combination of the Barker and Metropolis rules of the form, for $\gamma>0$, \[r(x)=\frac{x}{1+x}\left(1+2\left(\frac12\min\left(r,\frac 1r\right)\right)^\gamma\right).\]
        \end{enumerate}
    \end{remark}

    \begin{remark}[Issues with Metropolized-schemes]
    The Metropolis-Hastings algorithm provides a general recipe to define an unbiased Markov chain for the overdamped Langevin dynamics: one only needs to specify a proposition kernel, an acceptance rule, and a way to compute the ratio of the corresponding transition probabilities.
    For example, the so-called MALA scheme from \cite{RDF78} is very common, and combines a proposal function based on the Euler--Maruyama discretization of the dynamics \eqref{eq:overdamped_langevin} with a Metropolis acceptance rule.
    The use of the Metropolis-Hastings algorithm does not come for free. Because of the rejection rate, the correlations between samples decays at a slower rate, resulting in a higher asymptotic variance for estimators based on ergodic averages.
    In fact with regards to asymptotic variance, the choice of the Metropolis rule is optimal, as was shown in \cite{P73}.
    This implies that the use of a Metropolized scheme is only beneficial if the simulation time regime in which the systematic error overcomes the statistical error is computationally attainable. 
    For most systems of interest, this is not the case.
    Furthermore, the possibility of rejection effectively slows down the dynamics, which may degrade the quality of estimates for dynamical quantities.
    The effect of the Metropolization procedure with the MALA scheme on the estimation of self-diffusion properties has been analyzed in \cite{FHG15}, and improved rules and proposals for the computation of transport properties are discussed in detail in \cite{FG16}. 
    \end{remark}

    \subsection{Numerical illustrations}
    We now turn to a few numerical illustrations of the discussions above.
    In Figure \ref{fig:eos_argon}, we compute numerically compute the pressure as a function of the density for the Lennard--Jones fluid, using the parameters for Argon, at two different temperatures. 
    The resulting profile gives a numerical estimation of the equation of state of Argon, and comparison with experimental data shows that the agreement is very good at low densities and room temperature.
    The size of the systems was $N=2744$ atoms, using a sharp cutoff at $r_c=2.5$, and a BOAB splitting scheme with $\Dt=0.005$, and $\gamma=1.0$ was used for the friction parameter.
    We verified by using the block averaging procedure described in \cite{FP89} for the trajectories at both ends of the density range that the standard error bars are negligible.

    \begin{figure}[htbp]
        \begin{center}
          \includegraphics[width=0.7\linewidth]{figures/chapter1/argon_nvt_150K.pdf}
          \includegraphics[width=0.7\linewidth]{figures/chapter1/argon_nvt_300K.pdf}
          \caption{ \label{fig:eos_argon}
            Simulated equations of state of Argon at 150 K (liquid phase, top) and 300 K (supercritical phase, bottom). Experimental reference curves obtained from the NIST website \cite{NIST22} are plotted in red, simulated data points correspond to blue crosses.
          }
        \end{center}
      \end{figure}

    In Figures \ref{fig:configurational_bias} and \ref{fig:kinetic_energy_bias}, we highlight the systematic error in several observables. Simulations were run for systems of $N=27$ Lennard--Jones particles, at a reduced temperature $T=1.25$ and at a reduced density $\rho=0.25$. 
    A cutoff of the potential at a distance $r_c=2.0$ was imposed, with a linear correction term ensuring that $V$ be $C^1$. 
    Additionally, regression lines were added by extrapolating the behavior at small $\Dt$ based on the theoretical expansions \eqref{eq:expansion_of_avg_in_dt}, and constraining the regression lines to converge to the same value in the limit $\Dt\to 0$.
    This was achieved by a linear least squares regression procedure.
    The result also illustrate that a possibility to reduce the systematic error is to compute a desired average for several timesteps, and computing an extrapolated value at $\Dt=0$ based on the theoretical knowledge of the behavior of the systematic error as $\Dt\to 0$. This is known as Richardson extrapolation, see for instance \cite[Section 9.6]{QSS10}.
\begin{figure}[htbp]
    \begin{center}
      \includegraphics[width=0.49\linewidth]{figures/chapter1/potential_energy_bias.pdf}
      \includegraphics[width=0.49\linewidth]{figures/chapter1/virial_bias.pdf}
      \caption{ \label{fig:configurational_bias}
        Systematic error in configurational quantities for a Lennard--Jones fluid.
      }
    \end{center}
  \end{figure}

  \begin{figure}[htbp]
    \begin{center}
      \includegraphics[width=0.7\linewidth]{figures/chapter1/kinetic_energy_bias.pdf}
      \caption{ \label{fig:kinetic_energy_bias}
        Systematic error in kinetic energy for a Lennard--Jones fluid.
      }
    \end{center}
  \end{figure}
For the two configurational quantities we investigated, the virial and potential energy, we observe an overlap in the bias between the BAOAB and BAOA schemes.
It so happens that this overlap can be simply explained by a result relating the invariant measures of certain pairs of numerical schemes, see Lemma \ref{TU_like_lemma} in the next chapter.
In fact, a preprint \cite{KK22} was recently posted, showing that a certain widely used scheme in the molecular dynamics community was equivalent to the BAOA scheme, and that the latter sampled the same configurational marginal as the BAOAB scheme.
This prompted the need for a more thorough investigation of the properties of the BAOA scheme, which we discuss in the next chapter.
\chapter{Study of the BAOA scheme}
We consider time discretization schemes of underdamped Langevin dynamics known as the BAOA and BAOAB schemes, in order to compare the sampling bias induced by the timestep $\Delta t$ for these two methods.
Analysis of the timestep bias for the BAOAB scheme is given in \cite{LMS13}, however it does not appear that the BAOA bias is so well understood.
It has been observed numerically in \cite{KK22} (Section III.B) that the bias on the kinetic marginal distribution is much lower using the BAOA method. We attempt to explain this from a theoretical point of view, before illustrating our results in numerical examples.
 Building on known results for the BAOAB scheme, we show the following results.
\begin{enumerate}[(i)]
  \item In section \ref{relating invariant measures}, we express the invariant measure of the BAOA scheme in terms of the invariant measure of the BAOAB scheme (Proposition \ref{prop:mup_expression}), and using this expression, we show, as in \cite{KK22} (Section II.C), the equality between their respective configurational marginal distributions (Corollary \ref{corr marginal equality}). 
  \item In section \ref{BAOA first order estimate}, we show that the dominant error term for BAOA averages is only of order one in $\Delta t$, confirming that the BAOAB method is in general of higher order (Corollary \ref{corr2 baoa expansion}). 
  \item We show in section \ref{second order on the marginals} that for kinetic observables, however, the error is of second order in $\Delta t$, so that both marginal distibutions are second-order accurate (Corollary \ref{corr3 second order marginals}). 
  \item In section \ref{BAOA kinetic second order estimate}, we give an expression for the dominant error term in the kinetic marginal distribution of the BAOA scheme (Proposition \ref{prop:kappa_p_second_order}). In fact, we conjecture that, at least in dimension one, this term cancels, leading to an order of at least $\Delta t^3$ (Conjecture \ref{conjecture}).
  \item Lastly, in section \ref{discrepancy term kinetic}, we analyze the difference between the kinetic marginal distribution under the BAOA and BAOAB scheme (Proposition \ref{prop discrepancy term}), and explain why this difference leads to a systematic underestimation of the kinetic variance in BAOAB trajectories (Remark \ref{remark}).
\end{enumerate}

\subsection{Definitions and notations}\label{defs}
The underdamped Langevin dynamics is defined by the following stochastic differential equation:

\begin{equation}\label{Langevin}
  \left\{\begin{aligned}
      \mathrm{d}q_t&=M^{-1}p_t\dt,\\
      \mathrm{d}p_t&= -\nabla V(q_t)\dt -\gamma M^{-1}p_t\dt+\sqrt{\frac{2\gamma}\beta}\mathrm{d}W_t,
  \end{aligned}\right.
\end{equation}
where $M\in \R^{dN\times dN}$ is a symmetric mass matrix, $(q_t,p_t)\in \mathcal D\times \R^{dN}\defeq \mathcal E$, $(W_t)_{t\geq 0}$ is a standard $dN$-dimensional Brownian motion, 
and $\gamma, \beta$ are positive real parameters, respectively the friction coefficient and the inverse temperature. $\mathcal D$ is the configuration space, which is either the torus $(L(\R / \mathbb Z))^{dN}$ or the full space $\R^{dN}$.

The canonical measure, which we denote with the same symbol as its density, 

\begin{equation}
\label{canonical measure}
\int_{\mathcal E} \varphi\,\mathrm{d}\mu=\int _{\mathcal E}\varphi(q,p)\mu(q,p)\,\mathrm{d}q\,\mathrm{d}p =\frac 1 Z\int_{\mathcal E} \varphi(q,p) \mathrm{e}^{-\beta\left(\frac 12 p\cdot M^{-1}p + V(q)\right)}\,\mathrm{d}q\,\mathrm{d}p
\end{equation}
is invariant for this dynamics.

We can write $\mu$ as a product measure on $\R^{dN}\times \mathcal D$

\begin{equation}
\label{tensor form}
\mu(q,p)=\kappa(p)\nu(q), \qquad \nu(q)=Z_{\nu}^{-1}\mathrm{e}^{-\beta V(q)},\qquad \kappa(p)=\det\left(2\pi\beta M^{-1}\right)^{-\frac 12}\mathrm{e}^{-\frac\beta 2\langle M^{-1}p,p\rangle}.
\end{equation}

Notice $\kappa$ corresponds to a centered Gaussian law, the Maxwell-Boltzmann distribution.
The generator of the Langevin dynamics is the operator
\begin{equation}
  \label{Langevin generator}
\mathcal L_\gamma=M^{-1}p\cdot \nabla_q-\nabla V(q) \cdot \nabla_p- \gamma M^{-1} p \cdot \nabla_p+\frac\gamma\beta \Delta_p,
\end{equation}
which splits into three elementary generators, namely 
$$\mathcal L_\gamma= A+B+\gamma C,$$
with
\begin{equation}
  \label{Langevin generator splitting}
\qquad A=M^{-1}p\cdot \nabla_q,\qquad B=-\nabla V(q) \cdot \nabla_p,\qquad C=-M^{-1}p\cdot \nabla_p +\frac1\beta \Delta_p.
\end{equation}
These generators give rise to dynamics which we can express explicitly, defined by the following evolution operators:

\begin{equation}
  \label{propagators}
  \left\{\begin{aligned}
    &\mathrm{e}^{tA}\varphi(q,p)=\varphi(q+tM^{-1}p,p),\\
    &\mathrm{e}^{tB}\varphi(q,p)=\varphi(q,p-t\nabla V(q)),\\
   &\mathrm{e}^{t\gamma C}\varphi(q,p)=\mathbbm E \left[\varphi \left(q, \mathrm e^{-\gamma M^{-1}t}p + \sqrt{\frac{\gamma M^{-1}}{\beta}(1-\mathrm{e}^{-2M^{-1}t})}G\right)\right],
\end{aligned}\right.
\end{equation}
where $G$ is a standard $dN$-dimensional Gaussian. The dynamics associated with the A and B part are deterministic Hamiltonian dynamics, and the C part gives rise to an Ornstein-Uhlenbeck process on momentum coordinates. We consider two splitting schemes for \eqref{Langevin}, as defined by the following evolution operators:

\begin{equation}
  \label{schemes}
  \left\{\begin{aligned}
    &P_{\Delta t}=\e^{\Delta t B}\e^{\frac{\Delta t}2A}\e^{\Delta t \gamma C}\e^{\frac{\Delta t}2A},\\
    &Q_{\Delta t}=\e^{\frac{\Delta t}2B}\e^{\frac{\Delta t}2A}\e^{\Delta t \gamma C}\e^{\frac{\Delta t}2A}\e^{\frac{\Delta t}2B}.\\
\end{aligned}\right.
\end{equation}
These correspond respectively to the BAOA and the BAOAB scheme.
We also denote $\mu_{\Delta t,P}, \mu_{\Delta t, Q}$ the invariant measures for the Markov chains associated with \eqref{schemes}. We assume these have smooth densities which we also denote $\mu_{\Delta t,P}, \mu_{\Delta t,Q}$, and that a certain ergodicity condition holds (see Lemma 1).
Additionally we denote by $\nu_{\Delta t,P},\nu_{\Delta t,Q},\kappa_{\Delta t,P},\kappa_{\Delta t,Q}$ the associated marginals and densities with obvious notation inspired by \eqref{tensor form}.


\subsection{Relating invariant measures of discretization schemes}\label{relating invariant measures}
In this paragraph, we provide a formula for $\mu_{\Delta t,P}$ in terms of $\mu_{\Delta t,Q}$. This result allows one to very simply show the equality in the configurational marginals between these two measures, as noted in \cite{KK22}.
The main tool is the following result, which is a reformulation of the TU lemma (Lemma 9 from \cite{LMS13}).

\begin{lemma}
  Let $P_{\Delta t}, Q_{\Delta t}$ be bounded operators on $B^\infty(\mathcal E)$.
    Assume that, for any $n\geq 1$,
    $$ R_{\Delta t} P_{\Delta t}^n = Q_{\Delta t}^n S_{\Delta t},$$
    where $R_{\Delta t}$ and $S_{\Delta t}$ are bounded operators on $B^\infty (\mathcal E)$, such that $R_{\Delta t}\1=\1$, and that the following ergodic condition holds: for any $\varphi \in B^\infty(\mathcal E)$, and almost all $(q,p) \in \mathcal E$,
    $$ \underset{n\to\infty}\lim P_{\Delta t}^n\varphi (q,p) = \int_{\mathcal E} \varphi(q,p)\,\mu_{\Delta t,P}(\mathrm{d} q,\mathrm{d} p) $$
    $$ \underset{n\to\infty}\lim Q_{\Delta t}^n\varphi (q,p) = \int_{\mathcal E} \varphi(q,p)\,\mu_{\Delta t,Q}(\mathrm{d} q,\mathrm{d} p).$$
    Then  we have the relation $\mu_{\Delta t,P}$ and $\mu_{\Delta t,Q}$ via the following relation:

    \begin{equation*}
    \int_{\mathcal E} \varphi(q,p) \mu_{\Delta t,P}(\mathrm{d} q,\mathrm{d} p)=\int_{\mathcal E} \left(S_{\Delta t}\right)\varphi(q,p) \mu_{\Delta t,Q}(\mathrm{d} q,\mathrm{d} p)
    \end{equation*}
\end{lemma}

Applying Lemma 1 to \eqref{schemes} yields the following result.
\begin{prop}\label{prop:mup_expression}
  The following relation between the densities $\mu_{\Delta t,P}$ and $\mu_{\Delta t,Q}$ holds.

  \begin{equation}
    \label{mu P expression}
    \mu_{\Delta t,P}(q,p) = \mu_{\Delta t,Q}\left(q,p-\frac{\Delta t}2V(q)\right).
  \end{equation}
\end{prop}
\begin{proof}
  From the expressions \eqref{schemes}, we immediately get:

\begin{equation}
  \label{TU relation}
  P_{\Delta t}^n\e^{\frac{\Delta t}2B}=\e^{\frac{\Delta t}2B}Q_{\Delta t}^n,
\end{equation}
whereby applying Lemma 1, we get for any test function $\varphi$,
\begin{equation}
  \label{TU ccl}
  \int_{\mathcal E}\mathrm e^{\frac{\Delta t}2B}\varphi\, \mathrm{d}\mu_{P,\Delta t}=\int_{\mathcal E}\varphi\, \mathrm{d}\mu_{Q,\Delta t}
\end{equation}
Using equation \eqref{TU ccl} with $\psi=\e^{-\frac{\Delta t}2B}\varphi$ yields an exact expression for $\mu_{\Delta t,P}$ in terms of $\mu_{\Delta t,Q}$:
\begin{equation}
  \label{pi P expression}
  \int_{\mathcal E}\varphi\, \mathrm{d}\mu_{\Delta t,P} = \int_{\mathcal E}\e^{-\frac{\Delta t}2B}\varphi \mathrm{d}\mu_{\Delta t,Q}.
\end{equation}
Since $\varphi$ is arbitrary, we infer that at the level of densities,
\begin{equation}
  \mu_{\Delta t,P}(q,p)=\left(\mathrm{e}^{-\frac{\Delta t}2B}\right)^\dagger\mu_{\Delta t,Q}(q,p),
\end{equation}
where $\dagger$ denotes the adjoint on the flat space $L^2(\mathcal E)$. A simple computation shows that 
$$\mathrm e^{-\frac{\Delta t}2 B^\dagger}=\mathrm{e}^{\frac{\Delta t}2B},$$
since $B^\dagger=-B$.
Hence,
\begin{equation}\label{prop1 ccl}
  \mu_{\Delta t,P}(q,p) = \mathrm{e}^{\frac{\Delta t B}2}\mu_{\Delta t,Q}(q,p)=\mu_{\Delta t,Q}\left(q,p-\frac{\Delta t}2V(q)\right),
\end{equation}
which is the desired conclusion.
\end{proof}

Relation \eqref{prop1 ccl} is enough to show an equality between the configurational marginal distributions $\nu_{\Delta t,P}$ and $\nu_{\Delta t,Q}$, as noted in \cite{KK22}.
\begin{corollary}\label{corr marginal equality}
  The marginal distributions in the $q$ variable of $\mu_{\Delta t,P}$ and $\mu_{\Delta t,Q}$ coincide:
  \begin{equation}
    \label{marginal distributions equality}
    \nu_{\Delta t,Q}(q)= \nu_{\Delta t,P}(q).
  \end{equation}
\end{corollary}
\begin{proof}
  Write, for any $q\in \mathcal D$,
  \begin{equation*}
    \label{corrolary 1 proof}
    \begin{aligned}
    \nu_{\Delta t,Q}(q) &= \int_{\R^{dN}}\mu_{\Delta t,Q}(q,p)\mathrm{d}p\\
    &=\int_{\R^{dN}}\mu_{\Delta t,Q}\left(q,p-\frac{\Delta t}2 \nabla V(q)\right)\mathrm{d}p\\
    &=\int_{\R^{dN}}\mu_{\Delta t,P}(q,p)\,\mathrm{d}p\\
    &= \nu_{\Delta t,P}(q),
    \end{aligned}
  \end{equation*}
  which proves the claim.
\end{proof}


\subsection{Error estimate on the phase space measure}\label{BAOA first order estimate}

We now turn to obtaining the dominant order in the sampling bias of $\mu_{\Delta t,P}$, building on previously known results for $\mu_{\Delta t,Q}$, and the relation \eqref{mu P expression}.
Error estimates on $\mu_{\Delta t,Q}$ have been investigated in \cite{LMS13} (Section 1.4). In particular, the following expansion of $\mu_{\Delta t,Q}$ is derived, which will be central in our analysis.

\begin{theorem}[Theorem 13 in \cite{LMS13}]
  There exists a smooth function $f_2$ such that for any smooth $\psi$,

  \begin{equation}
    \label{BAOAB expansion}
   \int_{\mathcal E}\psi(q,p) \mu_{\Delta t,Q}(q,p) \text  dq \mathrm{d}p=\int_{\mathcal E}\psi(q,p)\mu(q,p)\mathrm{d}q \mathrm{d}p+\Delta t^2\int_{\mathcal E}\varphi(q,p) f_2(q,p)\mu(q,p)\mathrm{d}q \mathrm{d}p + \Delta t^4 r_{\psi,\gamma,\Delta t},
  \end{equation}
  where the remainder $r_{\psi,\gamma,\Delta t}$ is uniformly bounded for $\Delta t$ sufficiently small. Moreover, an expression for the dominant error term is obtained, 
  
  \begin{equation}
    \label{remainder term}
    \left\{\begin{aligned}
    &f_2=\tilde f_2-\frac18(A+B)g\\
    &\mathcal L^{*}_\gamma\tilde f_2=\frac1{12}(A+B)\left[\left(A+\frac B2\right)g\right]\\
    &g\defeq\beta(M^{-1}p)\cdot \nabla V(q)
    \end{aligned}\right.,
  \end{equation}
  where $\mathcal L^*_\gamma$ is the adjoint of $\mathcal L_\gamma$ on the weighted space $L^2(\mu)$.  
\end{theorem}
 Smoothness here is meant in a technical sense (see Definition 8 in \cite{LMS13}), which we refrain from detailing here. Using this expansion, one can derive the dominant order error for the BAOA scheme.


\begin{corollary}\label{corr2 baoa expansion}
  For any smooth observable $\varphi$,
  \begin{equation}
    \label{corr2 ccl}
    \int_{\mathcal E}\varphi(q,p)\mu_{\Delta t,P}(q,p)\mathrm{d}q \mathrm{d}p=\int_{\mathcal E}\varphi(q,p)\left(1+\frac{\Delta t}2g(q,p)\right)\mu(q,p)\mathrm{d}q \mathrm{d}p +O(\Delta t^2),
    \end{equation}
where $g$ is given by \eqref{remainder term}.
\end{corollary}
\begin{proof}
  Combining \eqref{BAOAB expansion} with \eqref{mu P expression}, we get the following estimation for averages with respect to $\mu_{\Delta t,P}$:
  \begin{equation}
    \label{}
   \int_{\mathcal E}\varphi(q,p)\mu_{\Delta t,P}(q,p)\,\mathrm{d}q\, \mathrm{d}p= \int_{\mathcal E}\varphi(q,p)\mu\left(q,p-\frac{\Delta t}2\nabla V(q)\right)\,\mathrm{d}q\, \mathrm{d}p+ \mathrm{O}(\Delta t^2).
  \end{equation}
  Taylor expanding $\mu$ gives
$$\mu\left(q,p-\frac{\Delta t}2\nabla V(q)\right)=\mu(q,p)\left(1+\frac{\Delta t}2\beta(M^{-1}p)\cdot \nabla V(q) +\mathrm{O}(\Delta t^2)\right)=\mu(q,p)\left(1+\frac{\Delta t}2g+\mathrm{O}(\Delta t^2)\right),$$
hence we get 
\begin{equation}
\label{BAOA first order}
\int_{\mathcal E}\varphi(q,p)\mu_{\Delta t,P}(q,p)\,\mathrm{d}q\, \mathrm{d}p=\int_{\mathcal E}\varphi(q,p)\mu(q,p)\left(1+\frac{\Delta t}2g(q,p)+\mathrm{O}(\Delta t^2)\right)\,\mathrm{d}q\, \mathrm{d}p,
\end{equation}
which proves the claim.
\end{proof}


\subsection{Error estimates on the kinetic marginal distributions}\label{second order on the marginals}
Equation \eqref{corr2 ccl} expresses the fact that the invariant measure $\mu_{\Delta t,P}$ is only exact at first order in $\Delta t$, which is one less that $\mu_{\Delta t,Q}$.
So in full generality, one can expect an error of order $\Delta t$ on averages obtained from BAOA trajectories, versus $\Delta t^2$ for averages computed from BAOAB trajectories. However, if we restrict ourselves to marginal observables, that is observables which only depend on the configurational coordinate or the kinetic coordinate, the first order error term vanishes.
Indeed, we have the following.
\begin{corollary}\label{corr3 second order marginals}
Let $\varphi(q,p)=\varphi(q)$ or $\varphi(q,p)=\varphi(p)$ be a marginal observable. Then
$$\int_{\mathcal E}\varphi\, \mathrm{d} \mu_{\Delta t,P}=\int_{\mathcal E}\varphi\, \mathrm{d}\mu +\mathrm{O}(\Delta t^2).$$
\end{corollary}
\begin{proof}
  By \eqref{corr2 ccl}, it is sufficient to show
  \begin{equation}\label{corr2 term}\int_{\mathcal E}\varphi g\, \mathrm{d}\mu=0.\end{equation}
  This follows from the following cancellations.
  \begin{equation}
    \label{g cancellations}
      \int_{\R^{dN}}g(q,p)\mu(q,p)\,\mathrm{d}p=\int_{\mathcal D}g(q,p)\mu(q,p)\,\mathrm{d}q=0.
  \end{equation}
  Indeed,
  $$\int_{\R^{dN}}\beta (M^{-1}p)\cdot \nabla V(q)\mu(q,p)\,\mathrm{d}p=0,$$
since the integrand is an odd function of $p$, and the marginal of $\mu$ in $p$ is a centered Gaussian density. Also,
$$\int_{\mathcal D}\beta(M^{-1}p)\cdot \nabla V(q)\mu(q,p)\,\mathrm{d}q=-\int_{\mathcal D}(M^{-1}p)\cdot \nabla_q\mu(q,p)\,\mathrm{d}q=0,$$
by an integration by parts. By first integrating \eqref{corr2 term} over the coordinate independent of $\varphi$, one of the cancellations \eqref{g cancellations} yields the result.
\end{proof}

Corollary 2 gives no new information concerning configurational observables, since we already know by Corollary 1 that these have the same averages under $\mu_{\Delta t,P}$ and $\mu_{\Delta t,Q}$, and that by Theorem 1, these have error of order $\Delta t^2$.
However, kinetic observables may yield different averages.

\subsection{Analysis of the second order error term for kinetic averages under $\mu_{\Delta t,P}$}\label{BAOA kinetic second order estimate}
It was observed numerically in \cite{KK22} that the averages of the kinetic and configurational temperatures computed with a BAOA scheme have a bias of order greater than $\Delta t$, as expected from the argument above.
In fact, for the kinetic temperature, the order appears to  be greater than $\Delta t^2$, in contrast to averages computed with the BAOAB method.
Understanding this behavior theoretically requires comparing second order error terms.

We show the following result, which identifies the second-order error term for kinetic observables.

\begin{prop}\label{prop:kappa_p_second_order}
  Let $\psi(q,p)=\psi(p)$ be a smooth kinetic observable. 
   Then,
  \begin{equation}
    \label{prop 2}
    \int_{\mathcal E} \psi(p) \mu_{\Delta t,P}(q,p)\,\mathrm{d}q\, \mathrm{d}p=\int_{\mathcal E}\psi(p)\mu(q,p)\,\mathrm{d}p\, \mathrm{d}q+\Delta t^2\int_{\mathcal E} \psi(p)\tilde f_2(q,p)\mu(q,p)\,\mathrm{d}q\,\mathrm{d}p + \mathrm{O}(\Delta t^3),
  \end{equation}
  where $\tilde f_2$ is given by \eqref{remainder term}.
\end{prop}
From Theorem 13 of \cite{LMS13}, this error term is identical to the dominant error term for OBABO averages, and minus the dominant error term for OABAO averages.
\begin{proof}
  By writing
  $$\int_\mathcal{E}\psi(q,p)\mu_{\Delta t,P}(q,p)\,\mathrm{d} q\,\mathrm{d} p-\int_\mathcal{E}\psi(q,p)\mu(q,p)\,\mathrm{d}p\,\mathrm{d}q=\int_\mathcal{E}\left(\psi(q,p)-\int_{\mathcal E}\psi\,\mathrm{d}\mu\right)\mu_{\Delta t,P}(q,p)\,\mathrm{d} q\,\mathrm{d} p,$$
  we may assume without loss of generality that $\psi$ has average 0 with respect to $\mu$.

  Using \eqref{BAOAB expansion}, we get 
  $$\int_{\mathcal E} \psi(p) \mu_{\Delta t,Q}(q,p)\,\mathrm{d}q\,\mathrm{d}p =\Delta t^2 \int_{\mathcal E}\psi(p)f_2(q,p)\mu(q,p)\,\mathrm{d}q\,\mathrm{d}p + \mathrm{O}(\Delta t^3),$$
  so that using our relation \eqref{mu P expression}, we get
  $$ \int_{\mathcal E} \psi(p) \mu_{\Delta t,P}(q,p)\,\mathrm{d}q\,\mathrm{d}p=\int_{\mathcal E}\psi(p)\mathrm e^{\frac{\Delta t}2B}\mu(q,p)\,\mathrm{d}q\,\mathrm{d}p + \Delta t^2 \int_{\mathcal E}\psi(p)\mathrm e^{\frac{\Delta t}2B}\left [ f_2(q,p)\mu(q,p)\right]\,\mathrm{d}q\,\mathrm{d}p + \mathrm{O}(\Delta t^3).$$
  This rewrites, at dominant order,
  $$\int_{\mathcal E}\psi(p)\mu\left(q,p-\frac{\Delta t}2\nabla V(q)\right)\,\mathrm{d}q\,\mathrm{d}p + \Delta t^2 \int_{\mathcal E}\psi(p)f_2(q,p)\mu(q,p)\,\mathrm{d}q\,\mathrm{d}p + \mathrm{O}(\Delta t^3).$$
  Expanding $\mu$ to the second order yields

\begin{align*}
  \label{mu 2-expansion}
  &\ \mu\left(q,p-\frac{\Delta t}2\nabla V(q)\right)+\mathrm{O}(\Delta t^3)\\
  &=\mu(q,p)\left[1+\beta\frac{\Delta t}2(M^{-1}p)\cdot \nabla V(q)+\frac {\Delta t^2}8\left[(\beta M^{-1}p)\otimes(\beta M^{-1}p)\nabla V(q)\right]\cdot \nabla V(q)-\beta\frac{\Delta t^2}8\left(M^{-1}\nabla V(q)\right)\cdot \nabla V(q)\right]\\
   &=\mu(q,p)\left[1+\beta\frac{\Delta t}2(M^{-1}p)\cdot \nabla V(q)+\frac{\Delta t^2}{8}\left(g^2(q,p)-\beta (M^{-1}\nabla V(q))\cdot \nabla V(q)\right)\right].
\end{align*}
Using $\int \psi \mathrm{d}\mu=0$ and the cancellation \eqref{g cancellations} on $q$ to remove the first order term, we obtain:

\begin{equation}
\label{BAOA second order}
\int_{\mathcal E} \psi(p) \mu_{\Delta t,P}(q,p)\,\mathrm{d}q\,\mathrm{d}p=\Delta t^2\int_{\mathcal E} \psi(p)\left(\frac 18\left(g^2(q,p)-\beta \left(M^{-1}\nabla V(q)\right)\cdot \nabla V(q)\right)+f_2(q,p)\right)\mu(q,p)\,\mathrm{d}q\,\mathrm{d}p+\mathrm{O}(\Delta t^3).
\end{equation}
Simplifications are possible. First, observe that 
$$-\beta\left(M^{-1}\nabla V(q)\right)\cdot \nabla V(q)=Bg(q,p),$$
so that, using the expression for $f_2$ given in \eqref{remainder term}, we get
\begin{equation}
  \int_{\mathcal E} \psi(p) \mu_{\Delta t,P}(q,p)\,\mathrm{d}q\,\mathrm{d}p=\Delta t^2\int_{\mathcal E} \psi(p)\left(\frac 18\left(g^2(q,p)-Ag(q,p)\right)+\tilde f_2(q,p)\right)\mu(q,p)\,\mathrm{d}q\,\mathrm{d}p+\mathrm{O}(\Delta t^3).
\end{equation}
Next, we examine the term
$$\left(g^2(q,p)-Ag(q,p)\right)\mu(q,p)=\left[\beta^2 \left( (M^{-1}p)\cdot \nabla V(q)\right)^2-\beta (M^{-1}p)\cdot(\nabla^2 V(q)M^{-1}p)\right]\mu(q,p),$$
by a straightforward calculation, where $\nabla^2$ denotes the Hessian matrix.
This expression is a finite sum of diagonal terms coming from both terms inside the brackets, and off-diagonal terms coming only from the rightmost term inside the bracket.
Importantly, these all vanish when integrated against the configurational marginal of $\mu$. To make this precise, we index $p$ and $q$ as 
$$p= (p_{i})_{1\leq i\leq dN},\ q= (q_{i})_{1\leq i\leq dN}.$$
Fixing indices $i\neq j$, the diagonal term corresponding to $i$ is 
\begin{equation}\label{diagonal term} \left[\beta^2\left(M^{-1}p\right)^2_i \left(\frac{\partial}{\partial q_i}V(q)\right)^2-\beta\left(M^{-1}p\right)^2_i\frac{\partial^2}{\partial q_i^2}V(q)\right]\mu(q,p)=\left(M^{-1}p\right)_i^2\frac{\partial^2}{\partial q_i^2}\mu(q,p), \end{equation}
and the off-diagonal term corresponding to $(i,j)$ is 
\begin{equation}
 \label{off diagonal term} -\beta \left(M^{-1}p\right)_i\left(M^{-1}p\right)_j\frac{\partial}{\partial q_i}V(q)\frac{\partial}{\partial q_j}V(q)\mu(q,p)=-\frac1{\beta}\left(M^{-1}p\right)_i\left(M^{-1}p\right)_j\frac{\partial^2}{\partial q_i\partial q_j}\mu(q,p).
\end{equation}
Factoring out the $q$-independent terms, and using the cancellations

\begin{equation}
  \label{second order cancellations on mu}
  \int_{\mathcal D}\frac{\partial^2}{\partial q_i^2}\mu(q,p)\,\mathrm{d}q=\int_{\mathcal D}\frac{\partial^2}{\partial q_i\partial q_j}\mu(q,p)\,\mathrm{d}q=0,
\end{equation}
which follow by integration by parts, we infer

\begin{equation}
  \int_{\mathcal E} \psi(p) \mu_{\Delta t,P}(q,p)\,\mathrm{d}q\,\mathrm{d}p=\Delta t^2\int_{\mathcal E} \psi(p)\tilde f_2(q,p)\mu(q,p)\,\mathrm{d}q\,\mathrm{d}p + \mathrm{O}(\Delta t^3),
\end{equation}
which concludes the proof.
\end{proof}

\begin{remark}
Using exponential decay estimates on the evolution semigroup $(\mathrm{e}^{t\mathcal L_\gamma})_{t\geq 0}$, (see \cite{LMS13}, paragraph 1.1.1 and references therein for more detail) one can show that the inverse operator $\mathcal L_\gamma^{-1}$ is well-defined for smooth centered observables.
Thus $\mathcal L_\gamma^{-1} \psi$ is well defined, say $\mathcal L_\gamma \Psi(q,p)=\psi(p)$.
Hence, \eqref{prop 2} rewrites 
\begin{align*}
  \int_{\mathcal E} \psi(p) \mu_{\Delta t,P}(q,p)\,\mathrm{d}q\,\mathrm{d}p&=\Delta t^2\int_\mathcal{E}\mathcal L_\gamma \Psi(q,p)\tilde f_2(q,p)\mu(q,p)\,\mathrm{d}q\,\mathrm{d}p+\mathrm{O}(\Delta t^3)\\
  &=\Delta t^2\int_\mathcal{E}\Psi(q,p)\mathcal L_\gamma^*\tilde f_2(q,p)\mu(q,p)\,\mathrm{d}q\,\mathrm{d}p+\mathrm{O}(\Delta t^3)\\
  &=\frac{\Delta t^2}{12}\int_\mathcal{E}\Psi(q,p)\left[\left(A+B\right)\left(A+\frac B2\right)g\right](q,p)\mu(q,p)\,\mathrm{d}q\,\mathrm{d}p+\mathrm{O}(\Delta t^3),
\end{align*}
using \eqref{remainder term}, which provides an alternative expression for the dominant error term.
Numerical evidence (see Figure \ref{fig:gamma_effect}) suggests that the error on BAOA and BAOAB averages is at dominant order independent of $\gamma$. Since the error term on BAOA given in \eqref{prop 2} depends on $\gamma$, this suggests that this term is zero, motivating the following conjecture.
\end{remark}

\begin{conjecture}\label{conjecture}
  For any smooth centered kinetic observable $\psi(p)$, we have
  
  \begin{equation}
    \int_{\mathcal E}\left(\mathcal L_\gamma ^{-1}\psi\right)(q,p)\left[\left(A+B\right)\left(A+\frac B2\right)g\right](q,p)\mu(q,p)\,\mathrm{d}q\,\mathrm{d}p=0.
  \end{equation}
  This would in particular imply that the kinetic marginal $\kappa_{\Delta t,P}$ is correct at order at least three in $\Delta t$, and is the subject of further investigation.
\end{conjecture}

\subsection{Analysis of the discrepancy between the dominant error terms on the kinetic marginals.}\label{discrepancy term kinetic}
Numerical evidence presented in \cite{KK22} shows a significant discrepancy between $\kappa_{\Delta t,P}$ and $\kappa_{\Delta t,Q}$.
Specifically, $\kappa_{\Delta t,Q}$ in the case $d=N=1$ tends to present a sharper peak than $\kappa_{\Delta t,P}$, thus underestimating the variance in the kinetic marginal.
 We show this in this paragraph this behavior is generic, in the sense that it does not, up to a shape parameter, depend on $V$.
The arguments above show that
\begin{equation}
  \begin{aligned}
    \int_{\mathcal E}\psi(p)\mu_{\Delta t,P}(q,p)\,\mathrm{d}q\,\mathrm{d}p&=\int_{\R^{dN}}\psi(p)\kappa_{\Delta t,P}(p)\,\mathrm{d}p\\
    &=\int_{\R^{dN}}\psi(p)\kappa(p)\,\mathrm{d}p+\Delta t^2\int_{\R^{dN}}\psi(p)\left(\int_{\mathcal D} \tilde f_2(q,p)\nu(q)\,\mathrm{d}q\right)\kappa(p)\,\mathrm{d}p +\mathrm{O}(\Delta t^3),  
  \end{aligned}
\end{equation}
where we used the product form \eqref{tensor form} for $\mu$. Similarly,
\begin{equation}
  \int_{\mathcal E}\psi(p)\kappa_{\Delta t,Q}(p)\,\mathrm{d}p=\int_{\R^{dN}}\psi(p)\kappa(p)\,\mathrm{d}p+\Delta t^2\int_{\R^{dN}}\psi(p)\left(\int_{\mathcal D}f_2(q,p)\nu(q)\,\mathrm{d}q\right)\kappa(p)\,\mathrm{d}p +\mathrm{O}(\Delta t^3),
\end{equation}
so that 

\begin{align*}
  \int_{\mathcal E}\psi(p)\left(\kappa_{\Delta t,P}(p)-\kappa_{\Delta t,Q}(p)\right)\,\mathrm{d}p&=\Delta t^2\int_{\R^{dN}}\psi(p)\left(\int_{\mathcal D}\left(\tilde f_2(q,p)-f_2(q,p)\right)\nu(q)\,\mathrm{d}q\right)\kappa(p)\,\mathrm{d}p +\mathrm{O}(\Delta t^3)\\
  &=\frac{\Delta t^2}8\int_{\R^{dN}}\psi(p)\left(\int_{\mathcal D}\left(A+B\right)g(q,p)\nu(q)\,\mathrm{d}q\right)\kappa(p)\,\mathrm{d}p +\mathrm{O}(\Delta t^3).
\end{align*}
Hence at the level of densities, we have at dominant order,

$$ \kappa_{\Delta t,P}(p)- \kappa_{\Delta t,Q}(p)=\frac{\Delta t^2\kappa(p)}8\int_{\mathcal D}\left(A+B\right)g(q,p)\nu(q)\,\mathrm{d}q + \mathrm{O}(\Delta t^3),$$
using the expressions for $f_2$ and $\tilde f_2$ given in \eqref{remainder term}. The following proposition gives an alternative expression for this discrepancy term.

\begin{prop}\label{prop discrepancy term}
  We have the following expression for the discrepancy term.

  \begin{equation}
    \label{discrepancy term}
    \kappa_{\Delta t,P}(p)- \kappa_{\Delta t,Q}(p)=\frac{\Delta t^2}8\mathrm{Tr}\left(\left(\left(\beta M^{-1}p\right)^{\otimes 2}-\beta M^{-1}\right)^\intercal\mathrm{Cov}_\nu(\nabla V)\right)\kappa(p) +\mathrm{O}(\Delta t^3).
  \end{equation}
\end{prop}
\begin{proof}
  For simplicity we assume $Z_\nu=\frac{\Delta t^2}8=\kappa(p)=1$. This has no incidence on our computations. We write:
  \begin{equation}
    (A+B)g(q,p)\nu(q)=\beta\left[\left(M^{-1}p\right)\cdot \left(\nabla^2V(q)M^{-1}p\right)-\left(M^{-1}\nabla V(q)\right)\cdot \nabla V(q)\right]\mathrm e^{-\beta V(q)}.
  \end{equation}
Setting $\tilde p= M^{-1}p$, we get 

\begin{equation}
  (A+B)g(q,M\tilde p)\nu(q)=\beta\left[\tilde p\cdot \left(\nabla^2V(q)\tilde p\right)-\left(M^{-1}\nabla V(q)\right)\cdot \nabla V(q)\right]\mathrm e^{-\beta V(q)}.
\end{equation}
This is a sum of terms of the form

$$T_{ij}(q,p)=\left[\beta \tilde p_i \tilde p_j \frac{\partial^2}{\partial q_i\partial q_j}V(q) - \beta M^{-1}_{i,j}\frac{\partial}{\partial q_i}V(q)\frac{\partial}{\partial q_j}V(q)\right]\mathrm e^{-\beta V(q)}.$$

Upon integrating this term over $\mathcal D$, we can integrate the left-most term by parts (boundary terms cancel out by periodicity or by growth conditions on $V$), to obtain
$$\int_{\mathcal D}T_{ij}(q,p)\,\mathrm{d}q=\int_{\mathcal D}\left[\beta^2\tilde p_i\tilde p_j \frac{\partial}{\partial q_i}V(q)\frac{\partial}{\partial q_j}V(q)-\beta M^{-1}_{i,j}\frac{\partial}{\partial q_i}V(q)\frac{\partial}{\partial q_j}V(q)\right]\mathrm e^{-\beta V(q)}\,\mathrm{d}q.$$
Hence,
$$\int_{\mathcal D}T_{ij}(q,p)\,\mathrm{d}q=\left(\beta^2\tilde p_i\tilde p_j -\beta M^{-1}_{i,j}\right)\int_{\mathcal D}\frac{\partial}{\partial q_i}V(q)\frac{\partial}{\partial q_j}V(q)\mathrm e^{-\beta V(q)}\,\mathrm{d}q,$$
so that
$$\int_{\mathcal D}(A+B)g(q,p)\nu(q)\,\mathrm{d}q=\sum_{i,j}\left(\beta^2\tilde p_i\tilde p_j -\beta M^{-1}_{i,j}\right)\int_{\mathcal D}\frac{\partial}{\partial q_i}V(q)\frac{\partial}{\partial q_j}V(q)\mathrm e^{-\beta V(q)}\,\mathrm{d}q,$$
which we rewrite

\begin{equation}
\left( \left(\beta M^{-1}p\right)^{\otimes 2}-\beta M^{-1}\right) : \int_{\mathcal D} \left(\nabla V \otimes \nabla V\right) (q) \nu(q)\,\mathrm{d}q=\mathrm{Tr}\left(\left(\left(\beta M^{-1}p\right)^{\otimes 2}-\beta M^{-1}\right)^\intercal\mathrm{Cov}_\nu(\nabla V)\right),
\end{equation}

using the fact that $\nabla V$ is a centered observable with respect to $\nu$, and concluding the proof.
\end{proof}

\begin{remark}\label{remark}
  This expression for the discrepancy term is not particularly wieldy, however it does explain the behavior observed in \cite{KK22}. In the case $d=N=\beta=M=1$, it becomes,

  $$\kappa_{\Delta t,P}(p)- \kappa_{\Delta t,Q}(p)=\frac{\Delta t^2}8(p^2-1)\mathrm{Var}_\nu(V')\kappa(p)+O(\Delta t^3),$$
  which is, up to a constant, the same correction term for any potential $V$. We plot this correction profile in figure \ref{fig:discrepancy_term}. The shape of this profile explains the higher peak observed in $\kappa_{\Delta t,Q}$.
\end{remark}

\begin{figure}[htbp]
  \begin{center}
    \includegraphics[width=0.49\linewidth]{/home/noeblassel/Documents/stage_CERMICS_2022/BAOA_tests/results/plots/discrepancy_term.pdf}
    \includegraphics[width=0.49\linewidth]{/home/noeblassel/Documents/stage_CERMICS_2022/BAOA_tests/results/plots/discrepancy_term2D.pdf}
    \caption{ \label{fig:discrepancy_term}
      Profile of the discrepancy term in one and two dimension, in the case of identity covariances for $\nabla V$.
    }
  \end{center}
\end{figure}

\section{Numerical results}
We propose illustrating our computations with numerical examples, on toy one dimensional systems.
\begin{enumerate}[(i)]
  \item In section \ref{models}, we define the potentials used for all the following experiments, and describe the sampling method used.
  \item In section \ref{nuP equals nuQ}, we verify numerically the relation \eqref{marginal distributions equality}.
  \item In section \ref{kappaP neq kappaQ}, we show that there is a significant discrepancy between the two kinetic marginal distributions. We also pinpoint the main, and possibly only source of this error, namely the $\gamma$-independent term \eqref{discrepancy term}.
  \item In section \ref{muP bad muQ good}, we numerically verify that the first order behavior \eqref{BAOA first order} is correct.
  \item In section \ref{why muP bad}, we give an explicit example of an observable for which the BAOA scheme yields a bias of order $\Delta t$.
  \item Finally, in section \ref{gamma does not count}, we show that the effect of the parameter $\gamma$ is undetectable at the level of the kinetic marginals, motivating Conjecture \ref{conjecture}.
\end{enumerate}


\subsection{Models}\label{models}
 We take $\beta=1$, $M=\mathrm{Id}$, and consider four potentials:
\begin{itemize}
  \item Periodic potential $$ \mathcal D = L(\R/\mathbb Z),\ L=1,\ V(q)=\sin(2\pi q/L), $$
  \item Quadratic potential $$ \mathcal D = \R,\ V(q)=\alpha \frac{q^2}2,\ \alpha=1,$$
  \item Double well potential $$ \mathcal D = \R,\ V(q)=\alpha \frac{q^2}2 +\beta\mathrm{e}^{-\frac{q^2}{2\sigma^2}},\ \alpha=1,\ \beta=4,\ \sigma=0.5,$$
  \item Tilted double well potential $$ \mathcal D = \R,\ V(q)=\alpha \frac{q^2}2 + \gamma q+\beta\mathrm{e}^{-\frac{q^2}{2\sigma^2}},\ \alpha=1,\ \beta=4,\ \gamma=1,\ \sigma=0.5.$$
\end{itemize}
Analytically unknown normalizing constants and reference quantities were obtained through numerical integration of $\mu$, using trapezoid rules with a mesh size of $10^{-6}$. For unbounded coordinates, we truncated the domain to the interval $[-5,5]$.
Approximations of $\mu_{\Delta t,P},\mu_{\Delta t,Q}$ were computed by recording the states of 10,000 independently evolving trajectories over $2\times 10^6$ timesteps in a $1000\times1000$ two-dimensional histogram on the truncated domain. The rare sample points outside of the truncated domain were discarded.

\subsection{Equality of marginal configurational distributions}\label{nuP equals nuQ}
On Figures  \ref{fig:marginal_q_periodic} and \ref{fig:marginal_q_wells}, we verify numerically the equality \eqref{marginal distributions equality} between the configurational marginal distributions $\nu_{\Delta t,Q}$ and $\nu_{\Delta t,P}$, which holds for any $\Delta t$. This point was demonstrated in \cite{KK22}.
\begin{figure}[htbp]
  \begin{center}
    \includegraphics[width=0.49\linewidth]{/home/noeblassel/Documents/stage_CERMICS_2022/BAOA_tests/results/plots/marginal/marginal_q_PERIODIC_0.1.pdf}
    \includegraphics[width=0.49\linewidth]{/home/noeblassel/Documents/stage_CERMICS_2022/BAOA_tests/results/plots/marginal/marginal_q_PERIODIC_0.2.pdf}
    
    \caption{ \label{fig:marginal_q_periodic}
      Marginal configurational distributions for the periodic potential. Left: $\Delta t=0.1$. Right: $\Delta t=0.2$. Even for large timesteps, the distributions coincide perfectly.
    }
  \end{center}
\end{figure}

\begin{figure}[htbp]
  \begin{center}
    \includegraphics[width=0.49\linewidth]{/home/noeblassel/Documents/stage_CERMICS_2022/BAOA_tests/results/plots/marginal/marginal_q_DOUBLE_WELL_0.4.pdf}
    \includegraphics[width=0.49\linewidth]{/home/noeblassel/Documents/stage_CERMICS_2022/BAOA_tests/results/plots/marginal/marginal_q_TILTED_DOUBLE_WELL_0.4.pdf}
    
    \caption{ \label{fig:marginal_q_wells}
      Marginal configurational distributions for $\Delta t=0.4$. Left: double well potential. Right: tilted double well potential.
    }
  \end{center}
\end{figure}


\subsection{Comparison of marginal kinetic distributions}\label{kappaP neq kappaQ}
We observe, as in \cite{KK22}, that the kinetic marginal distribution $\kappa_{\Delta t,Q}$ departs from the reference at a faster rate than $\kappa_{\Delta t,P}$, and more precisely appears to underestimate the variance, leading to a sharper distribution.
Additionally we observe that removing the part of the bias on BAOAB due to the discrepancy term \eqref{discrepancy term} leads to a significant improvement. These corrected marginals are plotted under the label "correction".
See Figures \ref{fig:marginal_p_periodic} and \ref{fig:marginal_p_double_well}.

\begin{figure}[htbp]
  \begin{center}
    \includegraphics[width=0.49\linewidth]{/home/noeblassel/Documents/stage_CERMICS_2022/BAOA_tests/results/plots/marginal/marginal_p_PERIODIC_0.1.pdf}
    \includegraphics[width=0.49\linewidth]{/home/noeblassel/Documents/stage_CERMICS_2022/BAOA_tests/results/plots/marginal/marginal_p_PERIODIC_0.2.pdf}
    
    \caption{ \label{fig:marginal_p_periodic}
      Marginal kinetic distributions for the periodic potential. Left: $\Delta t=0.1$. Right: $\Delta t=0.2$.
    }
  \end{center}
\end{figure}

\begin{figure}[htbp]
  \begin{center}
    \includegraphics[width=0.49\linewidth]{/home/noeblassel/Documents/stage_CERMICS_2022/BAOA_tests/results/plots/marginal/marginal_p_DOUBLE_WELL_0.3.pdf}
    \includegraphics[width=0.49\linewidth]{/home/noeblassel/Documents/stage_CERMICS_2022/BAOA_tests/results/plots/marginal/marginal_p_DOUBLE_WELL_0.4.pdf}
    
    \caption{ \label{fig:marginal_p_double_well}
      Marginal kinetic distributions for the double well potential. Left: $\Delta t=0.3$. Right: $\Delta t=0.4$.
    }
  \end{center}
\end{figure}


\subsection{Verification of the first-order expansion}\label{muP bad muQ good}
We verify the correctness first-order expansion of $\mu_{\Delta,P}$ obtained in \eqref{BAOA first order}, by comparing the joint distributions obtained from Monte-Carlo simulations with a reference calculation of the first-order expansion for $\mu_{\Delta t,P}$. Additionally, we plot the empirical estimate of $\mu_{\Delta t,Q}$ and $\mu$.
The plots show joint likelihoods as a function of the state, using a color mapping. Empirical joint distributions for BAOA and BAOAB trajectories are plotted on the top row of each figure.
On the bottom row, a reference computation of $\mu$ is plotted on the right, as well as a reference computation of
$$\left(1+\frac{\Delta t}2g\right)\mu$$
on the left. The results visually confirm our result, while suggesting that, as a whole, $\mu_{\Delta t,Q}$ is the superior approximation of $\mu$.
See Figures \ref{fig:joint_periodic}, \ref{fig:joint_quadratic} and \ref{fig:joint_double_well}.


\begin{figure}[htbp]
  \begin{center}
    \includegraphics[width=0.45\linewidth]{/home/noeblassel/Documents/stage_CERMICS_2022/BAOA_tests/results/plots/joint/joint_BAOA_PERIODIC_0.1.pdf}
    \includegraphics[width=0.45\linewidth]{/home/noeblassel/Documents/stage_CERMICS_2022/BAOA_tests/results/plots/joint/joint_BAOAB_PERIODIC_0.1.pdf}
    \includegraphics[width=0.45\linewidth]{/home/noeblassel/Documents/stage_CERMICS_2022/BAOA_tests/results/plots/joint/joint_theoretical_PERIODIC_0.1.pdf}
    \includegraphics[width=0.45\linewidth]{/home/noeblassel/Documents/stage_CERMICS_2022/BAOA_tests/results/plots/joint/joint_reference_PERIODIC.pdf}
    \caption{ \label{fig:joint_periodic}
      Joint distributions for the periodic potential, $\Delta t=0.1$.
    }
  \end{center}
\end{figure}

\begin{figure}[htbp]
  \begin{center}
    \includegraphics[width=0.45\linewidth]{/home/noeblassel/Documents/stage_CERMICS_2022/BAOA_tests/results/plots/joint/joint_BAOA_QUADRATIC_0.4.pdf}
    \includegraphics[width=0.45\linewidth]{/home/noeblassel/Documents/stage_CERMICS_2022/BAOA_tests/results/plots/joint/joint_BAOAB_QUADRATIC_0.4.pdf}
    \includegraphics[width=0.45\linewidth]{/home/noeblassel/Documents/stage_CERMICS_2022/BAOA_tests/results/plots/joint/joint_theoretical_QUADRATIC_0.4.pdf}
    \includegraphics[width=0.45\linewidth]{/home/noeblassel/Documents/stage_CERMICS_2022/BAOA_tests/results/plots/joint/joint_reference_QUADRATIC.pdf}
    \caption{ \label{fig:joint_quadratic}
      Joint distributions for the quadratic potential, $\Delta t=0.4$.
    }
  \end{center}
\end{figure}

\begin{figure}[htbp]
  \begin{center}
    \includegraphics[width=0.45\linewidth]{/home/noeblassel/Documents/stage_CERMICS_2022/BAOA_tests/results/plots/joint/joint_BAOA_DOUBLE_WELL_0.4.pdf}
    \includegraphics[width=0.45\linewidth]{/home/noeblassel/Documents/stage_CERMICS_2022/BAOA_tests/results/plots/joint/joint_BAOAB_DOUBLE_WELL_0.4.pdf}
    \includegraphics[width=0.45\linewidth]{/home/noeblassel/Documents/stage_CERMICS_2022/BAOA_tests/results/plots/joint/joint_theoretical_DOUBLE_WELL_0.4.pdf}
    \includegraphics[width=0.45\linewidth]{/home/noeblassel/Documents/stage_CERMICS_2022/BAOA_tests/results/plots/joint/joint_reference_DOUBLE_WELL.pdf}
    \caption{ \label{fig:joint_double_well}
      Joint distributions for the double well potential, $\Delta t=0.4$.
    }
  \end{center}
\end{figure}

\subsection{Example of first-order bias in a BAOA average}\label{why muP bad}
We demonstrate that for certain observables, BAOA is drastically outperformed by BAOAB, by calculating the average of $g$ for increasing timesteps. Note by \eqref{g cancellations}, the true average is 0. Figures \ref{fig:double_well_bias} and \ref{fig:tilted_double_well_bias} show the estimated averages as a function of the timestep on the left, and the same data on a log-log plot on the right. The order of the error on the BAOAB averages suggest that the second order error term
$$\int_{\mathcal E}gf_2\,\mathrm{d}\mu$$
given in \eqref{BAOAB expansion} cancels out, yielding a fourth-order bias in $\Delta t$ for BAOAB averages of $g$.

\begin{figure}[htbp]
  \begin{center}
    \includegraphics[width=0.49\linewidth]{/home/noeblassel/Documents/stage_CERMICS_2022/BAOA_tests/results/plots/bias/DOUBLE_WELL_bias_g.pdf}
    \includegraphics[width=0.49\linewidth]{/home/noeblassel/Documents/stage_CERMICS_2022/BAOA_tests/results/plots/bias/DOUBLE_WELL_bias_loglog_g.pdf}
    \caption{ \label{fig:double_well_bias}
      Averages of $g$ for the double well potential.
    }
  \end{center}
\end{figure}

\begin{figure}[htbp]
  \begin{center}
    \includegraphics[width=0.49\linewidth]{/home/noeblassel/Documents/stage_CERMICS_2022/BAOA_tests/results/plots/bias/TILTED_DOUBLE_WELL_bias_g.pdf}
    \includegraphics[width=0.49\linewidth]{/home/noeblassel/Documents/stage_CERMICS_2022/BAOA_tests/results/plots/bias/TILTED_DOUBLE_WELL_bias_loglog_g.pdf}
    \caption{ \label{fig:tilted_double_well_bias}
      Averages of $g$ for the tilted double well potential.
    }
  \end{center}
\end{figure}


\subsection{Effect of the friction parameter}\label{gamma does not count}
All experiments shown above used a value of $\gamma=1$ for the friction parameter. In this final experiment, we examine the effect of changing $\gamma$
We show the marginal kinetic distributions for three values of $\gamma\in \{0.1,1,10\}$.
The results show that there is no visually discernable effect of the parameter $\gamma$: all $\kappa_{\Delta t,P}$s are superposed close to the reference curve, and all $\kappa_{\Delta t,Q}$s are superposed above. This suggest that most of the error on $\kappa_{\Delta t,Q}$ arises from the additional term
$$-\frac{\Delta t^2}8\int_{\mathcal E} \varphi (A+B)g\, \mathrm{d}\mu,$$
which is the dominant discrepancy term in \eqref{discrepancy term}, and which is independent of $\gamma$. This is the fact we observed numerically on figures \ref{fig:marginal_p_double_well} and \ref{fig:marginal_p_periodic}.
See figure \ref{fig:gamma_effect}.

\begin{figure}[htbp]
  \begin{center}
    \includegraphics[width=0.90\linewidth]{/home/noeblassel/Documents/stage_CERMICS_2022/BAOA_tests/results/plots/bias/gamma_effect_DOUBLE_WELL_0.3.pdf}
    \caption{ \label{fig:gamma_effect}
      Kinetic marginal distributions for $\Delta t=0.3$ on the double well potential.
    }
  \end{center}
\end{figure}
\chapter{Non-equilibrium Molecular Dynamics}
\section{Non-equilibrium dynamics}
In the previous chapter, we investigated various numerical strategies to sample states from thermodynamic ensembles.
The object now is to go beyond the computation of average observables, to consider dynamical behavior of molecular systems.
One way to define question we may ask is how the system responds to small perturbations of the equilibrium.
For instance we may think of applying a small non-gradient force $\eta F$ to the equilibrium force $-\nabla V(q)$, which amounts to considering the following equation:
\begin{equation}
    \label{eq:nemd_langevin}
    \left\{\begin{aligned}
        \dif q_t &= M^{-1}p_t\dif t\\
        \dif p_t &= -\nabla V(q_t)\dif t +\eta F\dif t -\gamma M^{-1}p_t\dif t +\sqrt{\frac{2\gamma}{\beta}}\dif W_t
    \end{aligned}\right.
\end{equation}
We can think of this as effectively tilting the potential landscape so that we expect the steady-state of this perturbed dynamics to feature a measurable flux of particles in the direction $F$,
which we measure by looking at the average velocity in the direction $F$,
\begin{equation}
    \label{eq:mobility}
    \E_\eta\left[F\cdot \left(M^{-1}p\right)\right],
\end{equation}
where $\E_\eta$ denotes the average with respect to the steady-state for the dynamics \eqref{eq:nemd_langevin}, which we take to be a probability measure on $\mathcal E$,
 for which we have no closed form, in contrast to the equilibrium setting.
\section{Linear response theory}
\section{Numerical schemes}
\chapter{Norton dynamics}
We have considered so far two general methods for the computation of transport coefficients. The Green-Kubo method, which relies on analysis of autocorrelations in the fluctuation of zero-average equilibrium quantities, 
and the NEMD method, which relies in measuring the ratio in the average of a given equilibrium-centered response observable under a driven steady-state and the magnitude of the driving force.
In the case of mobility, we apply a small constant force, and measure the resulting particle flux in the direction of the perturbation. We can thus think of the mobility $\rho_F$ as measuring how \textit{responsive} the flux is to the forcing on the system.
A natural question to ask is whether it is possible to measure the dual quantity, that is, how \textit{resistive} the system is to a given flux. One possible strategy to answer this question, in loose terms, would be to \textit{constrain} the response to be constant, and measure the average magnitude of the forcing needed to maintain it.
In the limit of a small response, the linear dependency between these quantities can be hoped to provide an equivalent and reciprocal measure of the transport coefficient. By analogy with the Thévenin and Norton circuit theorems, we will from now on refer to the standard, constant-forcing method as the Thévenin method,
and the dual, constant-response method as the Norton method. We will again be using the mobility and the shear viscosity as our examples, so as to leverage our previous calculations as ways to validate our method.


\section{General framework}
We first express the method in full generality, before specializing to the non-equilibrium molecular dynamics context. We consider the stochastic differential equation:
\begin{equation}
    \label{eq:norton_general_sde}
    \left\{\begin{aligned}
        \dif X_t &= b(X_t)\,\dif t +\sigma(X_t)\,\dif W_t + \dif \Lambda_t F(X_t)\\
        R(X_t)&= r,
    \end{aligned}\right.
\end{equation}
where $b ,\sigma$ and $F$ are respectively $R^D,\, \R^{D\times E}$ and $\R^D$-valued functions, and $W$ is a $E$-dimensional Brownian motion.
We interpret $F$ as a perturbation direction for an \textit{equilibrium SDE}, and $\Lambda$ is the perturbation intensity, which is determined in order to maintain the response observable $R$ constant equal to $r>0$ along the dynamic's trajectories.
The Norton dynamics \eqref{eq:norton_general_sde} should be thought of as the constant-response counterpart to a corresponding Thévenin dynamics
\begin{equation}
        \label{eq:thevenin_general_sde}
    \left\{\begin{aligned}
        \dif X_t &= b(X_t)\,\dif t +\sigma(X_t)\,\dif W_t + \eta F(X_t),\\
        \eta>0,
    \end{aligned}\right.
\end{equation}
which is perturbed in the same direction, but with a constant forcing intensity $\eta$. Both these dynamics have a corresponding reference dynamics, respectively for $r=0$ and $\eta=0$,
with the reference Thévenin dynamics simply being the equilibrium Langevin dynamics.
For notational simplicity, let us introduce the following definitions:
\begin{equation}
    \label{eq:norton_notation}
    b_t = b(X_t),\qquad \sigma_t = \sigma(X_t),\qquad F_t= F(X_t),\qquad \nabla R_t=\nabla R(X_t),\qquad \nabla^2 R_t = \nabla^2 R(X_t).
\end{equation}
For $A,B\in\R^{D}$, such that $A\cdot B\neq 0$, we denote by $P_{A,B}$ the projector onto $A$ orthogonally to $B$
\begin{equation}
    \label{eq:non_orthogonal_projector}
    P_{A,B}=\frac{A\otimes B}{A\cdot B}.
\end{equation}
Finally, for any projector $P$, we denote by $\overline{P}$ the complementary projector
\begin{equation}
    \label{eq:complementary_projector}
    \overline{P}=\Id-P,
\end{equation}
We can now state the following result, which follows by a simple application of Itô calculus, and gives an expression for the dynamics \eqref{eq:norton_general_sde} without reference to $\Lambda$.
\begin{prop}\label{prop:norton_sde}
   Provided it is well-posed, the solution of the following SDE solves \eqref{eq:norton_general_sde}
    \begin{equation}
        \label{eq:norton_general_sde_solved}
        \dif X_t = \overline{P}_{F_t,\nabla R_t}\left[b_t\,\dif t + \sigma_t\,\dif W_t\right] - \frac{1}{2}\frac{\nabla^2 R_t : \left(\overline{P}_{F_t,\nabla R_t}\sigma_t \sigma_t^\intercal\overline{P}_{\nabla R_t,F_t}\right)}{F_t\cdot \nabla R_t}F_t\,\dif t,
    \end{equation}
    with $\Lambda_t$ taken to be an Itô process defined by
    \begin{equation}
        \label{eq:norton_sde_multiplier_solved}
        \dif \Lambda_t = -\frac{\nabla R_t \cdot b_t}{F_t\cdot \nabla R_t}\,\dif t-\frac{\nabla R_t\cdot \sigma_t\,\dif W_t}{F_t\cdot \nabla R_t}-\frac12\frac{\nabla^2 R_t : \left(\overline{P}_{F_t,\nabla R_t}\sigma_t\sigma_t^\intercal\overline{P}_{\nabla R_t,F_t}\right)}{F_t\cdot \nabla R_t}.
    \end{equation}
\end{prop}
\begin{proof}
The proof follows by analysis-synthesis.
We first assume that $\Lambda$ admits a decomposition as an Itô process,
\begin{equation}
    \label{eq:norton_multiplier_decomposition}
    \dif \Lambda_t = \lambda_t \dif t + \dif \widetilde{\Lambda}_t,
\end{equation}
where $\lambda$ is a bounded variation process and $\widetilde{\Lambda}$ is a martingale, and that $R$ sufficiently regular (say $C^2$ and bounded).
We can apply Itô's formula to the constant response condition, to get
\begin{equation}
    \label{eq:constant_response_condition}
     \nabla R_t \cdot \dif X_t + \frac12\nabla^2 R_t:\left\langle\sigma_t\,\dif W_t + \dif\widetilde{\Lambda}_tF_t\right\rangle=0,
\end{equation}
where the brackets denote the quadratic covariation of an Itô process. By identifying the martingale and bounded variation parts of the above process with $0$, we deduce an SDE for $\Lambda_t$:
\begin{equation}
    \label{eq:norton_multiplier_sde}
    \left\{\begin{aligned}
        &\nabla R_t \cdot \left( \sigma_t\,\dif W_t + F_t\,\dif \widetilde{\Lambda}_t\right)=0\\
        &\nabla R_t \cdot \left( b_t + \lambda_t F_t\right)+\frac12\nabla^2 R_t: \left\langle\sigma_t\,\dif W_t + \dif\widetilde{\Lambda}_tF_t\right\rangle=0.
    \end{aligned}\right.
\end{equation}
From the first equation, we deduce the expression of the martingale part of the forcing:
\begin{equation}
    \label{eq:norton_multiplier_martingale}
    \dif \widetilde{\Lambda}_t = - \frac{\nabla R_t\cdot \sigma_t\,\dif W_t}{\nabla R_t\cdot F_t},
\end{equation}
and as a consequence, we may compute the quadratic covariation bracket in \eqref{eq:constant_response_condition},
\begin{equation}
    \label{eq:constant_response_condition_bracket_solved}
    \left\langle\sigma_t\,\dif W_t + \dif\widetilde{\Lambda}_tF_t\right\rangle=\left\langle\left(\Id-\frac{F_t\otimes \nabla R_t}{F_t\cdot \nabla R_t}\right)\sigma_t\,\dif W_t\right\rangle = \overline{P}_{F,_t\nabla R_t}\sigma_t\sigma_t^\intercal\overline{P}^\intercal_{F_t,\nabla R_t}\,\dif t,
\end{equation}
where we use $P^\intercal_{A,B}=P_{B,A}$. Inserting equation \eqref{eq:constant_response_condition_bracket_solved} into \eqref{eq:norton_multiplier_sde} yields the full SDE for the forcing $\Lambda_t$ \eqref{eq:norton_sde_multiplier_solved} and the result follows by inserting the latter in \eqref{eq:norton_general_sde}.
\end{proof}

As stated above, we propose the Norton dynamics as a possible and alternative means of computing the linear response to a small perturbation of the equilibrium dynamics. 
We finish this general introduction by evoking a few theoretical questions related to these dynamics.
\begin{enumerate}[(i)]
    \item Existence of a unique steady-state for the dynamics \eqref{eq:norton_general_sde} and \eqref{eq:thevenin_general_sde}. As mentioned in the previous chapter, some results already exist in the Thévenin case.
    \item As before, proving the convergence of trajectory averages.
    \item With respect to the steady states, whose corresponding expectations we denote by $\E_\eta$ in the Thévenin case, and $\E_r$ in the Norton case, equality of the transport coefficients \[\underset{\eta \to 0}{\lim}\frac{\E_\eta[R]}{\eta}\] and \[\underset{r\to 0}{\lim}\frac{r}{\E_r[\lambda]}.\] This is the question of equivalence of the linear responses.
    \item Another natural question is that of the equivalence of the Norton and Thévenin equilibrium ensembles. This question asks about the existence of an asymptotic regime, or thermodynamic limit as $D\to\infty$, for which averages under the two equilibrium steady-states converge to a common value, for a sufficiently rich class of observables.
    \item On a related note, investigating the equivalence of the non-linear responses, that is the existence of a well-defined $r(\eta)$ such that the graphs $(\eta,\E_\eta[R])$ and $(\E_{r(\eta)}[\lambda],r(\eta))$ either agree or converge to a common value in some asymptotic regime when $D\to\infty$.
    \item Even more ambitiously, full equivalence of the non-equilibrium ensembles.
    \item Finally, we mention the question of finding a relation between the Norton equilibrium fluctuations of $\lambda$ and the linear response, in an analogy with the Green-Kubo formula.
\end{enumerate}
Let us mention that the question of the equivalence of of non-equilibrium ensembles has already been investigated by Evans in \cite{E93}, although the setting is purely deterministic, and the proof is formal.
The purpose of this chapter is to make somewhat rigorous a setting in which these questions are susceptible to find an answer, albeit a probably a difficult one to obtain.
Furthermore, the numerical results which we present in the remainder of this report show that these questions do not have obviously negative answers, and thus should be worthy of investigation.

\section{Norton dynamics for transport coefficients}
We now turn to specialising the setting to the Norton counterpart of the dynamics \eqref{eq:general_nemd_dynamics}. It reads
\begin{equation}
    \label{eq:norton_dynamics_general}
    \left\{ 
        \begin{aligned}
            &\dif q_t=m^{-1}p_t\,\dif t,\\
            &\dif p_t=-\nabla V(q_t)\,\dif t-\gamma M^{-1}p_t\,\dif t +\sqrt{\frac{2\gamma}\beta}\,\dif W_t +\dif \Lambda_t F(q_t),\\
            &R(q_t,p_t)=r,
        \end{aligned}
    \right.
\end{equation}
where $\Lambda_t$ is the magnitude of the perturbation, which is defined by the constraint $R(q_t,p_t)=r>0$.
We may assume $\gamma$ is a positive semi-definite diagonal matrix, as in the Thévenin case. $R$ is the response observable, which we take of the form
\[R(q,p)=G(q)\cdot p,\]
with $G$ a smooth vector field. Note in particular that $\nabla_p R(q,p)=G(q)$, and $\nabla_p^2 R(q,p)=0$. Finally note that the forcing acts solely on the momenta.
A SDE for $\Lambda_t$ can be obtained by applying Proposition \ref{prop:norton_sde}.
It immediately implies 
\begin{equation}
    \label{eq:norton_dynamics_general_solved}
    \left\{
        \begin{aligned}
            \dif q_t &= M^{-1}p_t \,\dif t\\
            \dif p_t &= \overline{P}_{F_t,G_t}\left[-\nabla V(q_t)-\dif t\gamma M^{-1}p_t\,\dif t + \sqrt{\frac{2\gamma}{\beta}}\,\dif W_t\right] - \frac{\nabla_q R_t\cdot M^{-1}p_t}{F_t\cdot G_t}F_t\,\dif t,
        \end{aligned}\right.
\end{equation}
with an expression for the forcing term
\begin{equation}
    \label{eq:norton_forcing_term}
    \left\{\begin{aligned}
        \dif \Lambda_t &= \lambda_t \,\dif t + \dif\widetilde{\Lambda}_t,\\
        \lambda_t &= \frac{G_t\cdot\left[\nabla V(q_t) + \gamma M^{-1}p_t \right] - \nabla_q R_t\cdot M^{-1}p_t}{F_t\cdot G_t},\\
        \dif \Lambda_t &= -\frac{G_t\cdot\sqrt{\frac{2\gamma}\beta}\,\dif W_t}{F_t\cdot G_t}.
    \end{aligned}\right.
\end{equation}

We use notations identical to \eqref{eq:non_orthogonal_projector}, \eqref{eq:complementary_projector} and \eqref{eq:norton_notation} in the expressions above. The bounded and martingale parts of the forcing term can be identified in \eqref{eq:norton_forcing_term}, and are given respectively by $\lambda_t$ and $\widetilde{\Lambda}_t$. 
We record the infinitesimal generator of the dynamics \eqref{eq:norton_dynamics_general_solved}, which is given by

\begin{equation}
    \label{eq:norton_general_generator}
    \cL_{\gamma}^{\mathrm{Norton}}\varphi = A \varphi - \left[\overline{P}_{F,G}\left(\nabla V+\gamma M^{-1}p\right)\right] \cdot \nabla_p\varphi - \frac{\nabla_q R \cdot M^{-1}p}{F\cdot G} F\cdot \nabla_p \varphi + \frac1\beta\nabla^2_p\varphi : \left[\overline{P}_{F,G}\gamma\overline{P}_{G,F}\right].
\end{equation}
Note $\overline{P}_{F,G}$ is a function of the coordinate variable in the equation above, and $A$ is as in \eqref{eq:Lham_splitting}.
\begin{remark}
    \label{rem:norton_sv_well_posedness}
    Note that the $G(q_t)\cdot F(q_t)$ term in the denominator may pose a question of well-posedness of the dynamics. Let us always suppose in our computations that $G(q_t)\cdot F(q_t)>0$, but this is by no means automatic.
    Indeed, thinking of the extreme case when $F$ and $G$ are orthogonal everywhere and $V=0$ highlights the fact that this is an issue of controlability: in that case, by isotropy, the component of the momentum in the direction $G$ will diffuse according to an Ornstein-Uhlenbeck process independent from any forcing applied in the direction $F$. In this case there is no way to control the response, and thus the dynamics is ill-defined. 
\end{remark}

Writing the Norton dynamics under the form \eqref{eq:norton_dynamics_general_solved} is instructive, since its structure appears clearly: its stochastic increments are comprised of the projected increments of an equilibrium Langevin dynamics, 
with an additional term involving $\nabla_q R$, which can be interpreted as a correction term enforcing the constant response constraint, in reaction to the configurational dynamics.
This fact will become clearer in the following section, where we describe splitting schemes for the Norton dynamics.

\subsection{Splitting schemes for numerical integration}
As in every case we considered so far, we will again rely on the fact that the Norton dynamics \eqref{eq:norton_dynamics_general_solved} can be split into three simpler, structure-preserving dynamics, to construct a variety of numerical schemes.
\begin{definition}[Splitting of the Norton dynamics]
    Consider the following dynamics
    \begin{equation}
        \label{eq:norton_A_dynamics}
        \left\{
            \begin{aligned}
                \dif q_t &=M^{-1}p_t\,\dif_t\\
                \dif p_t &= -\frac{\nabla_q R(q_t,p_t)\cdot M^{-1}p_t}{F(q_t)\cdot G(q_t)}F(q_t)\,\dif t
            \end{aligned}
        \right.,
    \end{equation}

    \begin{equation}
        \label{eq:norton_B_dynamics}
        \left\{
            \begin{aligned}
                \dif q_t &= 0\\
                \dif p_t &= -\overline{P}_{F_t,G_t}\nabla V(q_t)\,\dif t
            \end{aligned}
        \right.,
    \end{equation}
    and finally
    \begin{equation}
        \label{eq:norton_O_dynamics}
        \left\{
            \begin{aligned}
                \dif q_t &= 0\\
                \dif p_t &= \overline{P}_{F_t,G_t}\left[\gamma M^{-1}p_t\,\dif t +\sqrt{\frac{2\gamma}\beta}\,\dif W_t\right],
            \end{aligned}
        \right.
    \end{equation}
    which additively combine to form the Norton dynamics \eqref{eq:norton_general_sde_solved}. By analogy with the equilibrium setting, we will respectively refer to \eqref{eq:norton_A_dynamics}, \eqref{eq:norton_B_dynamics} and \eqref{eq:norton_O_dynamics} as the (Norton) $A$,$B$ and $O$ dynamics.
    At the level of the generator \eqref{eq:norton_general_generator}, this corresponds to a splitting into three operators,
    \[\cL_\gamma^{\mathrm{Norton}}=\cL_{A}^{\mathrm{Norton}}+\cL_{B}^{\mathrm{Norton}}+\cL_{\gamma,O}^{\mathrm{Norton}},\]
    with
    \begin{equation}
        \label{eq:norton_generator_splitting}
        \left\{
            \begin{aligned}
                \cL_{A}^{\mathrm{Norton}}\varphi &= A\varphi -\frac{\nabla_q R\cdot M^{-1}p}{F\cdot G}F\cdot \nabla_p \varphi,\\
                \cL_{B}^{\mathrm{Norton}}\varphi &= -(\overline{P}_{F,G}\nabla V)\cdot \nabla_p \varphi,\\
                \cL_{\gamma,O}^{\mathrm{Norton}}\varphi &= -(\overline{P}_{F,G}\gamma M^{-1}p)\cdot \nabla_p+\frac1\beta \left(\overline{P}_{F,G}\gamma\overline{P}_{G,F}\right):\nabla_p^2\varphi.
            \end{aligned}
        \right.
    \end{equation}
\end{definition}

The claim that these dynamics are structure-preserving is summed up in the following straightforward result.
\begin{lemma}
    Let $(q_t,p_t)$ refer to the solution of any of the Norton $A$,$B$ or $O$ dynamics. Then if $R(q_0,p_0)=r$, $R(q_t,p_t)=r$ for all $t>0$.
\end{lemma}
\begin{proof}
    The result can be proven by three thoughtless applications of the chain rule or Itô's formula.
    A quicker and more instructive proof follows from observing that for any $X,q,p\in \R^{dN}$, 
    \begin{align*}
        G(q) \cdot \overline{P}_{F(q),G(q)} X &= \overline{P}_{F(q),G(q)}^{\intercal}G(q)\cdot X\\
        &=\overline{P}_{G(q),F(q)}G(q)\cdot X\\
        &=0,
    \end{align*}
    since $P_{G(q),F(q)}$ is a projector onto the one-dimensional subspace spanned by $G(q)$ and $\overline{P}_{G(q),F(q)}$ is its complement. Hence any dynamics of the form 
    \[\dif q_t =0,\qquad \dif p_t = \overline{P}_{G_t,F_t}\left[b(q_t,p_t)\,\dif t +\sigma(q_t,p_t)\,\dif W_t\right]\]
    preserves $R(q_t,p_t)=G(q_t)\cdot p_t$, by applying the Itô formula, since $\nabla_p R(q,p)=G(q)$ and $\nabla_p^2 R=0$.
    This implies the result for the $B$ and $O$ dynamics. However, since the Norton dynamics also preserves $R(q_t,p_t)$ by construction, 
    the preservation property for the $A$ dynamics follows immediately, from the response conservation property of the Norton dynamics in the case $V=0$ and $\gamma=0$.
    For the more skeptical reader, we write, with $(q_t,p_t)$ a solution to the $A$ dynamics,
    \begin{align*}
        \frac{\dif}{\dif t} R(q_t,p_t) &= \nabla_q R(q_t,p_t)\cdot (M^{-1}p_t) -\nabla_p R(q_t,p_t)\cdot \left(\frac{\nabla_q R(q_t,p_t)\cdot M^{-1}p_t}{F(q_t)\cdot G(q_t)}F(q_t)\right)\\
        &= \nabla_q R(q_t,p_t)\cdot (M^{-1}p_t)\left(1-\frac{G(q_t)\cdot F(q_t)}{F(q_t)\cdot G(q_t)}\right)\\
        &=0
    \end{align*}
\end{proof}

As before, this splitting allows us to define a variety of numerical schemes for the Norton dynamics, following the same strategy as in Section \ref{section:splitting_schemes_langevin}.
In order to do so, we must prescribe a way to integrate each of the A,B and O steps individually. This is what we now turn to.
In the following, let us fix a timestep $\Dt>0$, and a response intensity $r>0$. Similar to the equilibrium setting, the B step can be integrated analytically, since its trajectories are ballistic.
Thus we have the following algorithm.
\begin{algorithm}[Norton B step]
    \begin{equation}
        \label{eq:norton_B_step}
        \left\{\begin{aligned}
            \widetilde{p}^{1} &= p^0 - \nabla V(q^0)\\
            p^{1} &= \widetilde{p}^{1} +\Dt \lambda^{1} F(q^0),
        \end{aligned}\right.
    \end{equation}
    where $\lambda^{1}$ is determined in order to conserve the response:
    \begin{equation}
        \label{eq:norton_B_step_lambda}
        G(q^0)\cdot p^{1} = r \iff \Dt\lambda^{1} = \frac{r-G(q^0)\cdot \widetilde{p}^{1}}{F(q^0)\cdot G(q^0)}.
    \end{equation}
\end{algorithm}
We next rely on the fact that the Norton O dynamics is in Ornstein--Uhlenbeck form to compute its exact solution over one timestep, using the following result.
\begin{prop}
    \label{prop:projected_ou_process}
    We consider the following general form of the Norton Ornstein--Uhlenbeck process in $\R^{dN}$. 
    \begin{equation}
        \label{eq:projected_ou_process}
        \dif p_t = - P \gamma M^{-1} p_t \,\dif t + P\sqrt{\frac{2\gamma}\beta}\,\dif W_t,
    \end{equation}  
    where $P^2=P$ is a projector (not necessarily orthogonal), $\gamma$ and $M^{-1}$ are positive definite symmetric matrices, and $W$ is a standard Brownian motion. We denote by $\overline P$ the complementary projector 
    $\overline P=\Id - P$. Assume also that $P$ and  $\gamma M^{-1}$ commute. Then, the analytical solution of \eqref{eq:projected_ou_process} writes
    \begin{equation}
        \label{eq:projected_ou_process_solved}
        p_t=\overline{P}p_0 + Pp_t^{\mathrm{eq}},
    \end{equation}
    where $p_t^{\mathrm{eq}}$ is the solution of the coupled equilibrium Ornstein--Uhlenbeck process,
    \begin{equation}
        \left\{\begin{aligned}
            \dif p_t^{\mathrm{eq}} &= - \gamma M^{-1} p_t^{\mathrm{eq}} \,\dif t + \sqrt{\frac{2\gamma}\beta}\,\dif W_t,\\
            p_0^{\mathrm{eq}} &= p_0.   
        \end{aligned}\right.
    \end{equation}
\end{prop}
\begin{proof}
    The proof follows the standard strategy of applying Itô's formula to the process rescaled by $\e^{tP\gamma M^{-1}}$. It yields
    \begin{equation}
        \dif \left(\e^{tP\gamma M^{-1}} p_t\right)=\e^{tP\gamma M^{-1}}\left(P\gamma M^{-1} p_t\,\dif t - P\gamma M^{-1}p_t \,\dif t+ P\sqrt{\frac{2\gamma}{\beta}}\,\dif W_t\right),
    \end{equation}
    whence, integrating in $t$ and multiplying both sides by $\e^{-tP\gamma M^{-1}}$,
    \begin{equation}
        p_t = \e^{-tP\gamma M^{-1}}p_0 +\int_0^t \e^{-P\gamma M^{-1}(t-s)}P\sqrt{\frac{2\gamma}{\beta}}\,\dif W_s.
    \end{equation}
    This yields an expression for the solution whatever the particular properties of $\gamma$, $M^{-1}$ and $P$. However, since $P$ is a projector, we can make use of the following useful formula: if $A$ and $P$ are square matrices with $AP=PA$, 
    \begin{equation}
        \label{eq:rodrigue_formula}
        \e^{PA}=\sum_{k=0}^\infty \frac{(PA)^k}{k!}=\sum_{k=0}^\infty \frac{P^kA^k}{k!}=\Id + P\sum_{k=1}^\infty \frac{A^k}{k!}=\Id + P\left(\e^A -\Id\right)= \overline{P}+P\e^A.
    \end{equation}
    The second equality follows from the fact that $A$ and $P$ commute, while the third follows from a repeated application of the projector identity $P=P^2=P^3=\dots$. This is sometimes referred to as Rodrigue's formula (see for instance \cite{LM15}{chapter 8}).
    Applying's Rodrigue's formula to our analytical form yields
    \begin{equation}
        p_t = \overline{P}p_0 +P\e^{-t\gamma M^{-1}}p_0 +\int_0^t \left(\overline{P}+ Pe^{-\gamma M^{-1}(t-s)}\right)P\sqrt{\frac{2\gamma}{\beta}}\,\dif W_s.
    \end{equation}
    expanding the product inside the integral, and using $\overline P P=0$, we get 
    \begin{equation}
        p_t = \overline{P}p_0 +P\e^{-t\gamma M^{-1}}p_0 +\int_0^t Pe^{-\gamma M^{-1}(t-s)}P\sqrt{\frac{2\gamma}{\beta}}\,\dif W_s.
    \end{equation}
    By our commutativity assumption, we can factor out the $P$s from the integral sign  and obtain
    \begin{equation}
        p_t = \overline{P}p_0 +P\e^{-t\gamma M^{-1}}p_0 +P^2\int_0^t e^{-\gamma M^{-1}(t-s)}\sqrt{\frac{2\gamma}{\beta}}\,\dif W_s= \overline{P}p_0 +P\left(\e^{-t\gamma M^{-1}}p_0 +\int_0^t e^{-\gamma M^{-1}(t-s)}\sqrt{\frac{2\gamma}{\beta}}\,\dif W_s\right),
    \end{equation} 
    regrouping the terms in $P$ and using again $P^2=P$. The result follows by simply recognizing the parenthesized term as $p_t^{\mathrm{eq}}$.
\end{proof}
\begin{remark}
    The commutativity assumption may seem overly restrictive, but in fact it is enough in the cases we consider. 
    For instance, in the case of shear viscosity computations with anisotropic friction, $M$ is a scalar multiple of the identity and $\gamma$ is diagonal and constant with respect to longitudinal coordinates, so that the commutativity condition is indeed verified.
\end{remark}

As a consequence of this computation, we may define a numerical strategy to integrate the fluctuation-dissipation part of the Norton dynamics.

\begin{algorithm}[Norton O step]
    \begin{equation}
        \label{eq:norton_O_step}
        \left\{
            \begin{aligned}
                \widetilde{p}^{1} &= \alpha_{\Dt}p^0 +\sigma_{\Dt}\mathcal G^1,\\
                p^{1} &= \widetilde{p}^{1} + \Dt\lambda^{1} F(q^{0}),
            \end{aligned}
        \right.
    \end{equation}
    where $\mathcal G^1$ is a standard $dN$-dimensional Gaussian, $\alpha_\Dt,\,\sigma_\Dt $ are given by \eqref{eq:alpha_sigma}, and $\lambda^{1}$ is again determined by equation \eqref{eq:norton_B_step_lambda} in order to conserve the response.
\end{algorithm}

Unfortunately, the A dynamics \eqref{eq:norton_A_dynamics} cannot be solved analytically. 
However, its general form is that of an infinitesimal Hamiltonian increment on the position coordinate, with an additional correction term in the direction $F$ on the momentum coordinate. 
Furthermore, we know the response is conserved by the A dynamics. This naturally suggests the following scheme.

\begin{algorithm}[Norton A step]
    \begin{equation}
        \label{eq:norton_A_step}
        \left\{\begin{aligned}
            q^1 &= q^0 + \Dt M^{-1}p^0, \\
            p^1 &= p^0 +\Dt\lambda^{1} F(q^1),
        \end{aligned}\right.
    \end{equation}
where $\lambda^{1}$ is determined by 
\begin{equation}
    \label{eq:norton_A_step_lambda}
    G(q^1)\cdot p^{1} = r \iff \Dt\lambda^{1} = \frac{r-G(q^1)\cdot p^0}{F(q^1)\cdot G(q^1)}
\end{equation}
to enforce the constant response constraint.
\end{algorithm}

An advantage of schemes for the Norton dynamics formed by chaining and iterating steps of \eqref{eq:norton_A_step}, \eqref{eq:norton_B_step} and \eqref{eq:norton_O_step} 
is that they immediately yield a discretization of the forcing intensity $\dif\Lambda_t$, and thus an estimation of the average $\E_r[\lambda]$, which is the quantity of prime interest to us.
Indeed, observe from equation \eqref{eq:norton_forcing_term} that we can write
\begin{equation}
    \label{eq:norton_forcing_splitting}
    \dif \Lambda_t = \lambda^{\mathrm{A}}_t \dif t + \lambda^{\mathrm{B}}_t\,\dif t + \dif \Lambda_t^{\mathrm{O}},
\end{equation}
with 
\begin{equation}
    \label{eq:norton_forcing_splitting_details}
    \left\{
    \begin{aligned}
        \lambda^{\mathrm{A}}_t &= -\frac{\nabla_q R(q_t,p_t)\cdot G(q_t)}{F(q_t)\cdot G(q_t)},\\
        \lambda^{\mathrm{B}}_t &= -\frac{\nabla V(q_t)\cdot G(q_t)}{F(q_t)\cdot G(q_t)},\\
        \dif \Lambda_t^{\mathrm{O}} &= -\frac{\left[-\gamma M^{-1}p_t\,\dif t +\sqrt{\frac{2\gamma}\beta}\,\dif W_t \right]\cdot G(q_t)}{F(q_t)\cdot G(q_t)}.
    \end{aligned}
    \right.
\end{equation}
These individual parts can be interpreted individually as the forcing intensities in the direction $F(q_t)$ applied to each of the A,B and O dynamics, relative to the corresponding equilibrium dynamics.
This yields a natural interpretation of the $\lambda^n$ terms along numerical trajectories, as discretizations of \eqref{eq:norton_forcing_splitting_details}, 
and thus one can use these ergodic averages for the average forcing. As a full example, we record the BABO-like scheme we used in our simulations, as well as the corresponding estimator of $\E_r[\lambda]$.
This procedure can of course be generalized to other orderings of the operators in the splitting. The impact of such a choice on the discretization error of the mean force should be the subject of future investigation.
\begin{example}[Estimation of the mean force using a BABO scheme.]
    The numerical scheme is implemented, for a fixed timestep $\Dt>0$, by iterating
    \begin{equation}
        \label{eq:BABO_norton_scheme}
        \left\{
        \begin{aligned}
            \widetilde{p}^{n+\frac14} &= p^n -\frac{\Dt}2\nabla V(q^n)\\
            \frac{\Dt}2\lambda^{n+\frac14} &= \frac{r-G(q^n)\cdot \widetilde{p}^{n+\frac14}}{F(q^n)\cdot G(q^n)}\\
            p^{n+\frac14} &= \widetilde{p}^{n+\frac14}+\frac{\Dt}2\lambda^{n+\frac14}F(q^n)\\
            q^{n+1} &= q^n +\Dt M^{-1} p^{n+\frac14}\\
            \Dt\lambda^{n+\frac12} &= \frac{r-G(q^{n+1})\cdot p^{n+\frac14}}{F(q^{n+1})\cdot G(q^{n+1})}\\
            p^{n+\frac12} &= p^{n+\frac14}+\Dt\lambda^{n+\frac12}F(q^{n+1})\\
            \widetilde{p}^{n+\frac34} &= p^n -\frac{\Dt}2\nabla V(q^{n+1})\\
            \frac{\Dt}2\lambda^{n+\frac34} &= \frac{r-G(q^{n+1})\cdot \widetilde{p}^{n+\frac34}}{F(q^{n+1})\cdot G(q^{n+1})}\\
            p^{n+\frac34} &= \widetilde{p}^{n+\frac34}+\frac{\Dt}2\lambda^{n+\frac34}F(q^{n+1})\\
            \widetilde{p}^{n+1} &= \alpha_\Dt p^{n+\frac34} +\sigma_\Dt\mathcal G^{n+1}\\
            \Dt\lambda^{n+1} &= \frac{r-G(q^{n+1})\cdot \widetilde{p}^{n+1}}{F(q^{n+1})\cdot G(q^{n+1})}\\
            p^{n+1}&=\widetilde{p}^{n+1}+\Dt\lambda^{n+1}F(q^{n+1}).
        \end{aligned}
        \right.
    \end{equation}
    If the scheme is iterated $N_{\mathrm{iter}}$ times, the mean force can be estimated by
    \begin{equation}
        \label{eq:norton_mean_force_estimator_general}
        \widehat{\lambda}_{N_\mathrm{iter}}'=\frac1{N_{\mathrm{iter}}}\sum_{k=0}^{N_{\mathrm{iter}}-1} \left(\frac{\lambda^{k+\frac14}+\lambda^{k+\frac34}}2+\lambda^{k+\frac12}+\lambda^{k+1}\right).
    \end{equation}
    Note that the $\lambda^{k+1}$ also account for the part of the correction due to the Gaussian increment $\mathcal G^{k+1}$. In practice, it may be possible to replace $\lambda^{k+1}$ with an estimation of the bounded variation part of 
    $\dif \Lambda^{\mathrm{O}}$,
    \[\lambda_t^{\mathrm{O}}=\frac{\gamma M^{-1}p_t\cdot G(q_t)}{F(q_t)\cdot G(q_t)},\]
    to obtain an estimator with a lower variance. For example, in our simulations, we took $M=m\Id$ and a constant $\gamma$, so that using the constant-response condition, 
    \[\lambda_t^{\mathrm{O}}=\frac{\gamma m^{-1}}{F(q_t)\cdot G(q_t)},\]
    so that the mean force estimator we considered in practice was 
    \begin{equation}
        \label{eq:norton_mean_force_estimator_practical}
        \widehat{\lambda}_{N_\mathrm{iter}}=\frac1{N_{\mathrm{iter}}}\sum_{k=0}^{N_{\mathrm{iter}}-1} \left(\frac{\lambda^{k+\frac14}+\lambda^{k+\frac34}}2+\lambda^{k+\frac12}+\frac{\gamma m^{-1}}{F(q^k)\cdot G(q^k)}\right).
    \end{equation}
\end{example}

\subsection{Physical interpretation}
We conclude this theoretical discussion by pointing out a property of the Norton dynamics \eqref{eq:norton_dynamics_general_solved} in the deterministic case $\gamma=0$, and when the forcing direction $F$ is proportional to the response direction $G$.
Without loss of generality, we assume $F=G$, since changing the proportionality constant amounts to changing the scale of the response.
We also assume that the system is homogeneous, so that we may take $M=\Id$.
In this case the projectors $P_{G_t,F_t}$ and the like all become orthogonal projectors onto the one-dimensional subspace spanned by $F_t$, or its orthogonal complement.
We show that in this particular case there is a physical interpretation for the dynamics

\begin{prop}[Gauss's principle of least constraint]\label{prop:constrained_dynamics}
    Consider the deterministic version of the Norton dynamics,
    \begin{equation}
        \label{eq:norton_dynamics}
        \left\{\begin{aligned}
        \dot{q} &= p,\\
        \dot{p}&= -\overline{P}_{F(q)}\nabla V(q) - \frac{\nabla F(q)p\cdot p}{|F(q)|^2},
        \end{aligned}\right.
    \end{equation}
    where the constraint is given by 
    \[F(q_t)\cdot p_t = F(q_0)\cdot p_0=r,\]
    and where we write $\overline{P}_F$ for $\overline{P}_{F,F}$.
    The rightmost term in the equation for $\dot p$ comes from writing $\nabla_q R(q,p)=\nabla G(q)p$.
    Then Gauss's principle of least constraint is satisfied:

    \begin{equation}
        \label{eq:GPOLC_norton}
        \left\{\begin{aligned}
            \dot{q} &= p,\\
            \dot{p}&= \underset{f\cdot F(q)+\nabla F(q)p\cdot p =0}\argmin\,|-\nabla V(q)-f|^2.
            \end{aligned}\right.
    \end{equation}
    In other words, the force applied to the system undergoing the Norton dynamics minimizes at each point in time the Euclidean distance to the force applied to the same system undergoing the Hamiltonian dynamics \eqref{eq:hamiltonian_dynamics}, subject to the constant response constraint.
\end{prop}
\begin{proof}
    Firstly, let us make clear that the constraint in the minimization problem actually expresses the constant response constraint.
    Since \[r =F(q)\cdot p,\]
    a differentiation in time yields 
    \begin{equation}
        \label{eq:hidden_constraint}
        0=\nabla F(q)\dot{q}\cdot p + F(q) \cdot \dot{p}=\nabla F(q)p\cdot p +F(q)\cdot \dot{p}.
    \end{equation}
    Since the force applied to the system is equal to $\dot{p}$, this effectively yields a constraint on the set of possible forces applied to the Norton system.
    This is analogous to the case of a Hamiltonian system subject to a set of holonomic constraints, which depend only on the position $q$. 
    In this case, a so-called hidden or shadow constraint on the set of admissible forces can be obtained by twice-differentiating the constraint condition. 
    For a pedagogical discussion of this case, we refer the reader to \cite{LM15}, section 4.3.
    In our non-holonomic case (the constraint also depends on $p$), a single time-differentiation is enough.
    We can then compute the minimizer 
    \begin{equation}
        \underset{f\cdot F(q)+\nabla F(q)p\cdot p =0}\argmin\,|-\nabla V(q)-f|^2.
    \end{equation}
    Since the constraint is affine in $f$, the unique solution, provided $F(q)\neq 0$, is given by an orthogonal projection of $-\nabla V(q)$ onto the hyperplane (in the variable $f$)
    \[f \cdot F(q) +\nabla F(q)p\cdot p=0,\]
    whose normal direction is given by $F(q)$. Hence the minimizer $f^*$ can be written 
    \begin{equation}
        \label{eq:f_star_minimizer}
        f^*=-\nabla V(q) -\lambda F(q)
    \end{equation}
    for some Lagrange multiplier $\lambda$, which is determined by 
    \[f^* \cdot F(q) +\nabla F(q)p\cdot p=0.\]
    Substituting our tentative expression for $f^*$, we get 
    \[\left(-\nabla V(q) - \lambda F(q)\right)\cdot F(q)+\nabla F(q)p\cdot p=0.\]
    This gives 
    \begin{equation}
        \label{eq:lambda_gpolc_expression}
        \lambda=-\frac{\nabla V(q)\cdot F(q)}{|F(q)|^2}+\frac{\nabla F(q)p\cdot p}{|F(q)|^2},
    \end{equation}
    and finally 
    \[f^* = -\overline{P}_{F(q)} \nabla V(q) -\frac{\nabla F(q)p\cdot p}{|F(q)|^2}F(q),\]
    which concludes the proof.
\end{proof}
This result gives a nice physical interpretation of the deterministic Norton dynamics: it corresponds to a \textit{physical} trajectory of a Hamiltonian system subject to the constant-response non-holonomic constraint.
Indeed, Gauss's principle of least constraint \cite{G29} is equivalent to the Legendre-d'Alembert principle, which can be use to derive the equations of motion for constrained classical systems.
The use of Gauss's principle of least constraint to enforce thermodynamic constraints for non-equilibrium molecular dynamics was already discussed in \cite{EHFML83}.
It is not yet clear to us how to extend this result to the case where $F$ and $G$ are non-colinear, and when the system is inhomogeneous, although it seems likely in this case Gauss's principle should have to be stated with respect to some modified scalar product.
Another interesting direction would be to extend the principle to the properly Langevin case $\gamma>0$. At any rate, we now turn to presenting our numerical results.
As before, we will be using the two examples of mobility and shear viscosity computations.

\section{Numerical results}

\subsection{Mobility}
The Norton dynamics for mobility is recovered by setting $F$ to be constant
\[F \in \{F_{\mathrm{C}},F_{\mathrm{S}}\},\]
and taking 
\[G =M^{-1}F\]
indeed, the response
\[R(q,p) = M^{-1}F\cdot p = F\cdot M^{-1}p,\]
since $M$ is symmetric, which corresponds to the response observable for the mobility. Since $F$ is constant in the $q$ variable, the correction term in the A dynamics \eqref{eq:norton_A_dynamics} is $0$.
Hence the A step \eqref{eq:norton_A_step} is analytically correct, with a vanishing Lagrange multiplier.
 We thus estimate the mean force through the estimator \eqref{eq:norton_mean_force_estimator_practical} with $\lambda^{k+\frac12}=0$.
An estimator of \[\underset{r\to 0}{\lim}\,\frac{\E_r[\lambda]}{r}\] 
can then be obtained in the same way as \eqref{eq:rho_F_nemd_estimator}, by 

\begin{equation}
    \label{eq:norton_rho_F_estimator}
    \widehat{\rho}_F= |\widehat{\lambda}|^{-2}\widehat{\lambda}\cdot r= \underset{\rho \in \R}{\mathrm{argmin}} \left| \rho\widehat{\lambda} - r\right|^2,
\end{equation}
where $\widehat{\lambda} \in \R^k$ is a vector of $k$ ergodic averages of the form \eqref{eq:norton_mean_force_estimator_practical}, and $r$ is a vector containing the corresponding response intensities.
Assuming, for simplicity's sake, that all ergodic averages are computed with the same number $N_{\mathrm{iter}}$ of time steps, and that $\widehat{\lambda}$ has a (diagonal) matrix of asymptotic variances $\Sigma_\lambda^2$, applying the delta-method to \eqref{eq:norton_rho_F_estimator} yields
the following estimator for the asymptotic variance of $\widehat{\rho}_F$
\[
\sigma^2_{\rho_F}= \nabla g(\widehat{\lambda})^\intercal \Sigma^2_\lambda\nabla g(\widehat{\lambda}),
\]
where 
\[g(y)=|y|^{-2}y\cdot r,\]
which yields, after computation,

\begin{equation}
    \label{eq:norton_sv_av_estimator}
    \sigma^2_{\rho_F} = \sum_{i=1}^k \Sigma_{\lambda,ii}^2\left(\frac{r_i|\widehat{\lambda}|^2-2\widehat{\lambda}_i(\widehat{\lambda}\cdot r)}{|\widehat{\lambda}|^4}\right)^2.
\end{equation}
In the case $k=1$, this becomes
\begin{equation}
    \label{eq:norton_sv_av_estimator_fd}
    \sigma^2_{\rho_F}=\sigma_{\lambda}^2 r^2\widehat{\lambda}^{-4}.
\end{equation}
In Figure \ref{fig:norton_mobility_linear}, we plot the response-forcing diagram in the linear regime. 
The estimated transport coefficient from the color drift is consistent with the results obtained from the Thévenin method. For the single drift, we observe a discrepancy in the linear response.
In Figure \ref{fig:norton_mobility_full}, we plot the response-forcing curve in the non-linear regime, and superimpose the data obtained from the Thévenin method. 
For the color drift, the two computed equations of state coincide perfectly. For the single drift, we again observe a small discrepancy.
This leads us to speculate that the equivalence of non-linear responses may only hold in the thermodynamic limit, and for forcings which act on the bulk of the system, contrarily to the single drift forcing.
In Figure \ref{fig:norton_avs} , we show the computed asymptotic variance for estimators of $\rho_F$ based on the Thévenin methods and the Norton methods. 
Asymptotic variances for estimators of the response and mean force were computed using block averages as in \cite{FP89}, and the corresponding asymptotic variance for the Norton estimator of $\rho_F$ was obtained using \eqref{eq:norton_sv_av_estimator_fd}.
It appears that the two methods are roughly equivalent in terms of asymptotic variance, and that the variance for the finite-difference estimator scales like the inverse square of the forcing intensity, consistent with the result expected in the Thévenin case, as noted in Remark \ref{rem:linear_regime}.

The final estimates for the transport coefficients using the Norton method are 
\[\rho_{F_{\mathrm{C}}} = 0.1211 \pm 0.0006,\qquad \rho_{F_{\mathrm{S}}}=0.1089\pm0.0005.\]
Estimates of the transport coefficients were obtained using \eqref{eq:norton_rho_F_estimator}, and statistical error was estimated using \eqref{eq:norton_sv_av_estimator}

\begin{figure}[htbp]
    \begin{center}
      \includegraphics[width=0.9\linewidth]{figures/norton_mobility_plot.pdf}
      \caption{ \label{fig:norton_mobility_linear}
        Mobility response versus average forcing intensity in the linear regime. Least squares linear regression lines are plotted in dotted line, and the estimated transport coefficients are indicated in the legend.
      }
    \end{center}
  \end{figure}

  \begin{figure}[htbp]
    \begin{center}
      \includegraphics[width=0.7\linewidth]{figures/norton_mobility_full.pdf}
      \caption{ \label{fig:norton_mobility_full}
        Comparison of the Thévenin and Norton mobility equations of state.
      }
    \end{center}
  \end{figure}

  \begin{figure}[htbp]
    \begin{center}
      \includegraphics[width=0.7\linewidth]{figures/asymptotic_vars_mobility.pdf}
      \caption{ \label{fig:norton_avs}
       Plot of the asymptotic variance as a function of the forcing intensity for the different methods, on a log-log plot.
       Results corresponding to a Thévenin estimator are denoted with a T, and those corresponding to the Norton estimator are denoted with a N. 
      }
    \end{center}
  \end{figure}

\subsection{Shear viscosity}
The NEMD response observable \eqref{eq:nemd_shear_viscosity_response} for shear viscosity can be recovered in the form 
\[R_k(q,p)= \widetilde{G}_k(q)\cdot p\]
 by setting
\begin{equation}
    \label{eq:shear_viscosity_response_alt}
    \forall\, 1\leq i\leq N,\,\forall\, 2\leq \alpha\leq d,\qquad \widetilde{G}_k(q)_{i1}=\frac{1}{m}\exp\left(\frac{2ik\pi q_{i2}}{L}\right),\qquad \widetilde{G}_k(q)_{i\alpha}=0.
\end{equation}
Thus we can apply the Norton method. In practice, to avoid dealing with a complex exponential, we rather define
\begin{equation}
    \label{eq:shear_viscosity_response}
    \forall\, 1\leq i\leq N,\,\forall\, 2\leq \alpha\leq d,\qquad G_k(q)_{i1}=\frac{1}{m}\sin\left(\frac{2k\pi q_{i2}}{L}\right),\qquad G_k(q)_{i\alpha}=0.
\end{equation}
This can be achieved at no cost to generality by considering an appropriate translation of the forcing profile with respect to the results of Figure \ref{tab:fourier_coefficients}.
Estimators for the transport coefficient \eqref{eq:shear_viscosity_tranpsort_coefficient} and estimates of the statistical error can be obtained identically to the mobility case.
All simulations were run in the reference thermodynamics condition \eqref{eq:reference_thermo_condition_sv} for a minimum length of $8\times 10^7$ timesteps.

In Figure \ref{fig:norton_sv_linear_response}, we plot the normalized response \[\frac{R_1}{c_1(f_y)}\] against the estimated average forcing $\langle \lambda \rangle$ in the linear regime, for the different forcing profiles.
The estimated transport coefficients are consistent with one another, and agree with those obtained in the Thévenin and Green--Kubo setting.
In Figure \ref{fig:norton_sv_nonlinear_response}, we compare the non-linear response profiles between the Thévenin and Norton methods, for each of the forcing profiles. In every case, the responses coincide perfectly throughout.
In Figure \ref{fig:norton_avs_sv} we compare asymptotic variances of the finite difference estimators of the normalized response $\rho_{F,1}/c_1(f_y)$ using the Thévenin and Norton method. The asymptotic variance was estimated directly from the response time series in the Thévenin case, using a block averaging procedure,
and for the Norton method, a block averaging procedure on the forcing time series was combined with a delta-method. We find that the Norton estimators consistently outperform their Thévenin counterparts. Furthermore, it appears not all forcing profiles are equal in terms of asymptotic variance.
\begin{figure}[htbp]
  \begin{center}
    \includegraphics[width=0.9\linewidth]{figures/norton_sv_lin.pdf}
    \caption{ \label{fig:norton_sv_linear_response}
      Normalized Fourier response versus forcing intensity for the three transverse forcing profiles.
      The size of the linear response regime is roughly the same for every type of forcing. Least squares linear regression lines on the 10 first values are plotted in dotted line, and estimated normalized transport coefficients are indicated in the legend.
    }
  \end{center}
\end{figure}
\begin{figure}[htbp]
    \begin{center}
      \includegraphics[width=0.9\linewidth]{figures/norton_sv_avs.pdf}
      \caption{ \label{fig:norton_avs_sv}
        Scaling of the asymptotic variance for the finite difference estimator of the normalized transport coefficient for shear viscosity}
    \end{center}
  \end{figure}

  \begin{figure}[htbp]
    \begin{center}
      \includegraphics[width=0.75\linewidth]{figures/sinusoidal_joint.pdf}
      \includegraphics[width=0.75\linewidth]{figures/constant_joint.pdf}
      \includegraphics[width=0.75\linewidth]{figures/linear_joint.pdf}
      \caption{ \label{fig:norton_sv_nonlinear_response}
        Norton and Thévenin equations of state for the shear-viscosity Fourier response, for each type of forcing profile.
      }
    \end{center}
  \end{figure}


\appendix
\chapter{Implementation details}
We conclude this report by a short appendix dedicated to providing some details about the implementation of the methods we described in the previous chapters,
both from a historical point of view, meaning that we will be documenting the process, as well as of implementing the methods we described, as well as filling in some methodological gaps.
Throughout this document, the reader will notice that little is said about $\nabla V$, meaning the methods we described simply invoked $\nabla V$ where it is needed, 
without getting into the specifics of \textit{how} it is computed.
For a pair interaction potential of the form \eqref{eq:lennard_jones}, the naive method of summing over all pairs of particles incurs a cost of
\[\frac{N(N-1)}2 =\mathrm{O}(N^2)\]
evaluations of the pair interaction function $v$.
Another point we alluded to is that the statistical error on the computation of averages usually dissipates at a rate $\mathrm{O}(1/\sqrt{\Dt N_{\mathrm{iter}}})$.
Furthermore, for averages to be physically meaningful, $N$ must be taken large enough for the thermodynamic limit to be considered reached.
In practice, a quadratic scaling of the computational cost of computing $\nabla V$ may render essentially impossible the computation of long numerical trajectories.
Instead, one has to rely on approximate evaluations of $\nabla V$. For instance, in the case of a Lennard-Jones interaction, the pair interaction potential decays quickly, in $\mathrm{O}(r^{-6})$ as a function of the interparticular distance $r$.
In this case, a reasonable approximation consists on imposing a fixed maximal interaction distance $r_c$, called the \textit{cutoff} distance, and asserting that particles which are further apart that $r_c$ do not interact.
This amounts mathematically to replacing the pair interaction function $v(r)$ by a cut-off version $v(r)\1_{r<r_c}$, effectively modifying the potential.
There are various ways to alter this procedure in order to make the modified potential continuous or $C^1$, for instance using spline interpolation. 
Of course, this procedure alone does not solve anything, since one still needs to determine \textit{which} pairs of particles are neighbors, meaning they are closer apart than $r_c$, which still amounts to $\mathrm{O}(N^2)$ computations of the (squared) Euclidean norm.
To make this viable, one has to store a list of neighbor pairs in an efficient data structure which only needs to be updated once every few simulation iterations, since the set of neighboring pairs tend not to drastically change over short time intervals.
In practice, and depending on the timestep, the meaning of the word \textit{few} can sometimes be understood to be as large as $50$.
There are various strategies to implement such a data structure, such as Verlet lists or cell lists, which are described in detail in (citer Allen and Tildesley eg).
At any rate, since the number of neighbors to a given particle should on average be close to $4\pi\rho r_c^3/3=\mathrm{O}(1)$, where $\rho$ is the particle density, the cost of computing $\nabla V$ becomes (ideally) \textbf{linear} in $N$,
which makes long time simulations a practical possibility. 
We mention the issue of short-range interactions since all our molecular simulations used $V$ of the Lennard-Jones form, but other methods exist in the case where $V$ decays slowly like the Coulomb electrostatic potential, relying in this case on a fast decay property in Fourier space.
Of course, many more tricks exists, which collectively form the whole craftsmanship of Molecular Dynamics (MD) simulations: finding efficient, approximate ways to compute $V$ and $\nabla V$.

The process of implementing from scratch these methods is time consuming, and unnecessary given the number of freely available MD packages. 
However, some of these packages are quite ancient and heavily optimized so that the process of implementing new methods comes with quite a heavy learning overhead, as well as being harder to debug.
Instead, we chose to make use of a Julia package for MD simulation, thus taking this opportunity to learn the programming language Julia \cite{bezanson2017julia}, which has been gaining a lot of traction in the scientific computing user base, 
in part because it offers the shared benefit of performance comparable to that of C or C++, as well as possessing a rich ecosystem of highly intercompatible packages, powered by a strong community and a design philosophy centered around generic, type-agnostic, interfaces.

The first step in the internship consisted in choosing a package between two contenders, NBodySimulator \cite{NBS}, and Molly \cite{Molly}. 
NBodySimulator, in summary, acts as a wrapper around the native differential equations Julia framework, constructing a differential equation or stochastic differential equation from the specification of a molecular system.
As such, it aims to compute \textit{exact} solutions to the dynamics, or at least exact evaluations of $V$. 
On the other hand, Molly offers a number of neighbor-finding strategies, such as the cell list method, a parallelized naive iteration over all pairs of neighbors, as well as a tree-search based method. 
In Figure \ref{fig:linear_scaling}, we show the results of a comparison of the scaling of the simulation time as a function of the system size, for NBodySimulator and Molly's various neighbor finding strategies.
We find that, if for very small systems, NBodySimulator outperforms Molly, the advantage quickly disappears, for $N \geq 750$ approximately, in the case of the cell list method.
Interestingly, even the naive double loop for Molly overtakes NBodySimulator in performance. This is because by default, Molly parallelizes the force computation, while there is no parallelization in NBodySimulator.

\begin{figure}[htbp]
    \begin{center}
      \includegraphics[width=0.7\linewidth]{figures/appendix/scaling_comparison_full.png}
      \includegraphics[width=0.7\linewidth]{figures/appendix/scaling_smallN.png}
      \caption{ \label{fig:linear_scaling}
        Comparison of world clock time for the simulation of 10000 steps of a Lennard-Jones system. The "no nf" corresponds to the naive double loop. Top: full curve. Bottom: zoom on the small system regime.
      }
    \end{center}
  \end{figure}

On this basis, we chose to implement our methods in Molly. Along the way, some of our implementations were integrated in the Molly source, in large part during a one-week stay in the MRC Laboratory of Molecular Biology in Cambridge, where the main author of Molly's source code, Joe Greener, works.
These integrations include:
\begin{enumerate}[i)]
    \item A cutoff strategy based on a cubic spline interpolation,
    \item A general Langevin integrators, which works for all splitting orderings \eqref{eq:splitting_ordering} and in the (non)-equilibrium setting,
    \item A logger to efficiently estimate time correlations functions of the form \eqref{eq:time_correlation},
    \item A logger to track the self-diffusion coordinates \eqref{eq:self_diffusion_process} for Einstein mobility computations,
    \item A refactoring of Molly's general logging architecture, for more flexibility and extensibility,
    \item The addition of a parameter for Boltzmann's constant, allowing for fully consistent sets of user-defined custom systems of units,
    \item An online ergodic average and asymptotic variance estimator for a user-specified observable.
\end{enumerate}
These were additionally documented, and non-regression tests were implemented for each of them. We refer the reader to the Molly documentation and source code for more details.
Other implementations were too use-specific, not general or not mature enough to be integrated. These include
\begin{enumerate}[i)]
    \item A pressure observable based on \eqref{eq:pressure},
    \item The implementation of NEMD force fields for mobility and shear viscosity computations,
    \item The implementation of an integrator for the Norton method,
    \item Various logging and visualization utilities.
\end{enumerate}
These implementations are available in the repository \cite{myrepo}, although we apologize in advance to the interested reader for the poorly organized, largely uncommented mess that lies within.
We expect, especially for the pressure observable, that a future integration in Molly will be possible, for instance once long-range force contributions are implemented, as well as a more flexible interface for force computations.
This will be a necessary step in the implementation of barostats, which are important in the context of biomolecular simulation, a use case Molly is particularly attuned to.
On the other hand, non-equilibrium methods are unlikely to have a place in a general-purpose MD package.
Instead, we might consider publishing in the future a standalone package that extends Molly for transport coefficient computations, once our understanding of the Norton method has cohered.


\printbibliography
%-------\begin{thebibliography}{10}
    \bibitem{LMS13}
    B.~Leimkuhler, C.~Matthews, G.~Stoltz.
    \newblock The computation of averages from equilibrium and nonequilibrium Langevin molecular dynamics
    \newblock{\\ \url{https://arxiv.org/abs/1308.5814v1}}
  
    \bibitem{KK22}
    S. ~Kieninger, B.G. ~Keller
    \newblock GROMACS Stochastic Dynamics and BAOAB are equivalent configurational sampling algorithms
    \newblock{\\ \url{https://arxiv.org/abs/2204.02105}}
  
    \bibitem{HLG03}
    E. ~Hairer, C. ~Lubich, G. ~Wanner,
    \newblock Geometric numerical integration illustrated by the Störmer-Verlet method,
    \newblock Acta Numerica 12 (2003) 399-450

  \end{thebibliography}
\end{document}