\section{Hamiltonian averages}
\subsection{Classical Hamiltonian dynamics}
    \begin{definition}[Phase space coordinates]
    We describe the state of a system of $N$ particles by a tuple 
    $$(q,p)\in \cM^N\times \R^{dN} \defeq \cLs$$
    Where $\cM$ is a $d$-dimensional manifold on which lives the coordinate variable $q$, and $p$ is the momentum coordinate.\\
    The momentum of a particle is its velocity multiplied by its mass, thus we may write 
    $$v=M^{-1}p$$
    where $M$ is a diagonal matrix encoding the masses of each particle, and $v$ is the velocity coordinate.\\
    As we are interested in the evolution of systems through time, we will be interested in trajectories through the phase space $\cLs$, that is functions
    $$\begin{cases}\R_+\mapsto \cLs\\
        t\mapsto (q_t,p_t)
    \end{cases}$$
    \end{definition}
        \begin{definition}[Hamiltonian]
            The Hamiltonian of a system is the function defined on $\cLs$ by\\

            \begin{equation}
                \label{Hamiltonian}
                H(q,p)=\color{blue}{\frac12p^\intercal M^{-1}p}+\color{purple}{V(q)}
            \end{equation}
            
            It is the total energy of the system in state $(q,p)$, sum of its \textcolor{blue}{kinetic energy} and \textcolor{purple}{potential energy}
        \end{definition}

        Under this description, Newton's second law admits the following reformulation, which defines Hamiltonian dynamics.

        \begin{equation}
            \label{Hamiltonian dynamics}
            \begin{cases}
                \text d q_t=M^{-1}p_t\dt=\nabla_p H(q_t,p_t)\dt\\
                \text d p_t=-\nabla V(q_t)\dt=-\nabla_q H(q_t,p_t)\dt
            \end{cases}
        \end{equation}


        \begin{remark}
        Equation (\ref{Hamiltonian dynamics}) concisely writes, for $X_t\defeq (q_t,p_t)$, 
        \begin{equation}\label{general hamiltonian equation} \text d X_t=J\nabla H(X_t)\dt\end{equation}
        Where
        $$J=\begin{pmatrix}
            0_{dN} & I_{dN}\\
            -I_{dN} & 0_{dN}
        \end{pmatrix}$$
        is the symplectic matrix. By the chain rule, we have
        $$ \text d H(X_t)=\text d X_t^{\intercal}\nabla H(X_t)=(J\nabla H(X_t))^\intercal \nabla H(X_t)\dt=0$$
        This relation expresses the fact that the total energy is conserved along trajectories of the system under Hamilton's equation.
        Note this property is only due to the form of $J$, and not to the particular expression (\ref{Hamiltonian}) of the classical Hamiltonian.
        In full generality, we may consider any dynamics of the form (\ref{general hamiltonian equation}), where we are free to choose $H$, and note that the conservation property holds.\\
        However, one key property of the classical Hamiltonian is that it is a sum of two terms each involving only one of the coordinate and momentum variable. Such Hamiltonian are called separable, and we shall see that separability proves a useful property to construct numerical schemes dedicated to the integration of such dynamics.
        \end{remark}
        
        \begin{equation}
            \label{Hamiltonian generator}
        \end{equation}

        \label{non-separable hamiltonian}
        \label{evolution operator exponential notation}

    \subsection{The Lennard-Jones Model}

    \subsection{Numerical integration of Hamiltonian dynamics}

    \subsection{Examples of instantaneous observables}
        
    \subsection{Shortcomings of the Hamiltonian approach}

\section{Canonical averages}
    \subsection{The notion of ensemble}

\subsection{Langevin dynamics}
We consider a special case of the inertial Langevin dynamics, defined by the following stochastic differential equation (SDE), where $\gamma, \beta$ are set real constants.

\begin{equation}
    \label{Langevin}
    \begin{cases}
        \text dq_t=M^{-1}p_t\dt \\
        \text dp_t= -\nabla V(q_t)\dt \textcolor{blue}{-\gamma M^{-1}p_t\dt}+\textcolor{purple}{\sqrt{\frac{2\gamma}\beta}\text dW_t}
    \end{cases}
\end{equation}

Where $(W_t)_{t\geq 0}$ is a standard $dN$-dimensional Brownian motion.\\
This process is a combination of a Hamiltonian evolution with an additional action on the momenta which, if isolated, defines a $dN$-dimensional Ornstein-Uhlenbeck process.\\
This additional term be interpreted physically as the combination of two effects: a \textcolor{blue}{dissipation term} which can be understood as the effect of a viscous friction force on the particles, and a \textcolor{purple}{fluctuation term}, which corresponds to the input of kinetic energy into the system as thermal agitation induced by a surrounding heat bath at temperature $1/(k_B\beta)$.\\
However, the physical meaning can be forgotten thanks to the fact that, \textit{in fine}, we only require that the canonical measure be ergodic under this dynamic: as we shall shortly see, this is indeed the case.

\begin{remark}
    There are several ways to generalize this process.\\
    One way is to consider more general, possibly non-separable, Hamiltonians, as in (\ref{non-separable hamiltonian}), rather than the classical Hamiltonian used above.\\
    The other is to allow the fluctuation-dissipation term to be parametrized by coefficients $\gamma$ and $\sigma$ depending on the state variable, and which obey a relation ensuring ergodicity.\\
    Hence in full generality, we could consider the following Langevin dynamic:
    
    \begin{equation}
        \label{general Langevin}
        \begin{cases}
            \text dq_t=\nabla_p H(q_t,p_t)\dt \\
            \text dp_t= -\nabla_q H(q_t,p_t)\dt -\gamma(q_t,p_t)\nabla_pH(q_t,p_t)\dt+\sigma(q_t,p_t)\text dW_t
        \end{cases}
    \end{equation}
\end{remark}



\subsection{Properties of the Langevin dynamics}

To investigate some of the properties of the dynamics, it is useful to introduce the notion of a generator for a process defined by a possibly inhomogeneous SDE.
\subsubsection{The generator}
We consider a general process defined by a SDE of the form:

\begin{equation}
    \label{SDE}
    \text dX_t=b(t,X_t)\text \dt + \sigma(t,X_t)\text dW_t
\end{equation}

Where $b$ is a $\R^n$-valued function, $W$ is a standard $d$-dimensional Brownian motion and $\sigma$ is a $n\times d$ matrix-valued function.\\
For $\varphi$ a smooth bounded function, Itô's lemma allows us to compute:
\begin{align*}
    d\varphi(t,X_t) &=\frac{\partial \varphi}{\partial t}(t,X_t)\text dt + \nabla^\intercal \varphi(t,X_t) \text dX_t + \frac12 \operatorname{Tr}(\nabla^{2\intercal}\varphi(t,X_t)\text d\langle X ,X\rangle _t)\\
    &=\left( \frac{\partial \varphi}{\partial t} + \nabla^{\intercal}\varphi b + \frac12 \operatorname{Tr}(\nabla^2\varphi \sigma\sigma^\intercal) \right)(t,X_t) \text dt+ (\nabla^\intercal \varphi\sigma )(t,X_t)\text dW_t
\end{align*}

Where $\nabla$, $\nabla^2$ are with respect to the spatial coordinates. In other words,\\


\begin{equation}
    \label{Ito}
    \varphi(t,X_t)= \varphi(0,X_0)+\int_0^t\left( \frac{\partial \varphi}{\partial t} + \nabla^\intercal\varphi b + \frac12 \operatorname{Tr}(\nabla^2\varphi\sigma\sigma^\intercal) \right)(s,X_s) \text ds + \int_0^t\nabla^\intercal \varphi\sigma(s,X_s)\text dW_s
\end{equation}
%----invariance of $\mu$

    \begin{definition}[Generator of an Itô process]
        Let $X_t$ be a $\R^n$-valued process defined by \ref{SDE}. We define its generator at time $t$ as the operator defined by

        \begin{equation}
            \label{Generator}
            \cL_t \varphi(x)=\left(\frac{\partial \varphi}{\partial t} + \nabla^\intercal\varphi b + \frac12 \operatorname{Tr}(\nabla^2\varphi \sigma\sigma^\intercal)\right)(t,x)
        \end{equation}

    \end{definition}

    In view of (\ref{Ito}), we have, provided regularity conditions on $\sigma$ and $\varphi$,
    $$\E\left[\varphi(t,X_t)\right|X_s=x]=x+\int_s^t \E[\cL_u\varphi(X_u)] \text du$$
    so that, at least formally,
    \begin{equation}
    \label{generator formal motivation}
    \frac{\partial}{\partial t}\E[\varphi(t,X_t) | X_{s}=x]=\E[\cL_{t} \varphi(X_{t})|X_{s}=x]=\cL_{t}\E[\varphi(t,X_{t})|X_{s}=x]
    \end{equation}

    If we define a family of evolution operators $(P_{s,t})_{s\leq t}$ by the formula
    $$P_{s,t} \varphi (x)= \E[ \varphi (t,X_t) | X_s=x] $$
    (\ref{generator formal motivation}) rewrites
    $$ \frac{\partial}{\partial t} P_{s,t} \varphi (x)=P_{s,t}\cL_t\varphi(x)=\cL_t P_{s,t}\varphi(x)$$
    
    An important special case occurs when $b,\ \sigma$ and $\varphi$ do not depend on time. In this case the generator is a single operator $\cL$, defined by
    $$\cL \varphi = \nabla^\intercal \varphi b + \frac12 \operatorname{Tr}(\nabla^2 \varphi \sigma \sigma ^\intercal)$$ 
    The evolution operators $P_t \defeq P_{0,t}$ ($ = P_{s,s+t}\ \forall s$ by stationarity) form a semi-group, and act on the space of smooth functions as
    $$ P_t \varphi(x)= \E[\varphi(X_t)| X_0=x]$$
    The formal derivative is given by
    $$ \frac{\partial}{\partial t}P_t=P_t\cL=\cL P_t$$
    
    As in (\ref{evolution operator exponential notation}), we may write, by analogy with the finite-dimensional setting,
    $e^{t\cL}\defeq P_t$


    \subsubsection{Invariance of the canonical measure}

    The Langevin dynamics (\ref{Langevin}), when written under the form (\ref{SDE}), corresponds to the case 

    $$ b(q,p)= \begin{pmatrix} M^{-1}p \\ -
    \nabla V(q)-\gamma M^{-1}p\end{pmatrix},\ \sigma(q,p)= \sqrt{\frac{2\gamma}\beta}\begin{pmatrix} 0_{dN} &  0_{dN} \\  0_{dN} & \text{I}_{dN} \end{pmatrix}$$
    Hence, applying \ref{Generator}, we obtain the generator for the dynamics
    $$ \cL=  \cL_q + \cL_p $$
    where
    $$ \cL_q \varphi= \nabla^\intercal_q\varphi M^{-1}p$$
    $$ \cL_p\varphi =-\nabla^\intercal_{p}\varphi (\nabla V(q)+\gamma M^{-1}p) +\frac{\gamma}{\beta}\Delta_p $$
    which we may rewrite, recognizing the generator for the Hamiltonian dynamics \ref{Hamiltonian generator}
    $$ \cL=\cL_{H}+\gamma \cL_{\text{ou}}$$
    where $\cL_{\text{ou}}\varphi =-M^{-1}\nabla^\intercal_p\varphi p + \frac1{\beta}\Delta_p\varphi$

    \begin{lemma}
        Let $\varphi,\psi$ be smooth compactly supported functions on $\cLs$. Then
    \end{lemma}

\subsection{Numerical integration of the Langevin dynamics}
    \subsubsection{Splitting methods}
    \subsubsection{Implementation}

\subsection{Illustration: the equation of state of Argon}
\subsection{The Metropolis method}